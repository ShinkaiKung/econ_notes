
\begin{definition}[Commodity Bundles and Constraints]
    Budget Set = consumption set + economic constrains \[
        \beta(p,w) = \{x \in X: px \leq w\}
    \] where:
    \begin{itemize}
        \item $p = (p_1, p_2,\dots,p_L) \in \mathbb{R}^{L}$ is a price vector,
        \item $w \in \mathbb{R}^{+}$ is wealth,
        \item Consumption Plans: $x = (x_1, x_2, \dots, x_L) \in \mathbb{R}^{L}$,  $\left\{\begin{array}{l}
                      x_l > 0: \text{receive}   \\
                      x_l < 0: \text{give away} \\
                  \end{array}\right.$
        \item Consumption Set: $X \in \mathbb{R}^{L}$, in general we assume $X = \mathbb{R}^{L}_{+}\equiv\{x\in \mathbb{R}^{L}: x \geq 0\}$
        \item  Commodity Space $\mathbb{R}^{L}$: finite number of goods, perfect divisibility.
    \end{itemize}

\end{definition}
\begin{remark*}
    Implicit Assumptions: \begin{itemize}
        \item Price-taking consumers
        \item Perfect Information
        \item Complete Markets
        \item Exogenous Wealth
        \item Linear Budget Constraint
    \end{itemize}

    Normally, budget set is \textbf{compact} and \textbf{convex}, and $p \gg 0$.
    A counterexample: quantity discount.
\end{remark*}

\section{Preference}
\begin{definition}[Preference]
    \textbf{Preference} is represented by $\succeq$, a binary relationship on $X$: \[
        (X, \succeq) \subset X \times X: x^{1} \succeq x^{2} \iff(x^{1}, x^{2}) \in (X, \succeq)
    \] which means $x^{1}$ is at least as good as $x^{2}$.
    \begin{remark*}
        $X$ represents the consumption set, $X \times X$ is the \underline{Cartesian Product}, which represents all the possible combo pairings of commodities. $(x^{1}, x^{2})$ represents an \underline{Ordered Pair}.
    \end{remark*}

    \textbf{Strict Preference} $\succ$: \[
        x^{1} \succ x^{2} \iff x^{1} \succeq x^{2} \wedge \neg(x^{2} \succeq x^{1})
    \]

    \textbf{Indifference} $\sim$: \[
        x^{1} \sim x^{2} \iff x^{1} \succeq x^{2} \wedge x^{2} \succeq x^{1}
    \]
\end{definition}

Properties of \textbf{rational preferences}: \begin{enumerate}
    \item Transitivity: $\forall x^{1}, x^{2}, x^{3} \in X$: $ x^{1} \succeq x^{2} \wedge  x^{2} \succeq x^{3} \Longrightarrow  x^{1} \succeq x^{3}$.
    \item Completeness: $\forall x^{1}, x^{2} \in X \Longrightarrow x^{1} \succeq x^{2} \vee x^{2} \succeq x^{1}$.
\end{enumerate}
Other preference properties:
\begin{enumerate}
    \setcounter{enumi}{2}
    \item Continuity: $\forall x^{0} \in X$, the sets $\{x \in X: x \succeq x^{0}\}$ and $\{x \in X: x^{0} \succeq x\}$ are closed. \begin{itemize}
              \item Upper Contour Set and Lower Contour Set are closed.
              \item $\succeq$ is continuous $\iff \forall x^{0} \in X, $ an arbitrary convergent sequence $(x^{n})^{\infty }_{n=1}$, \[
                        \Big[\forall n, x^{n} \succeq x^{0}\Big] \Longrightarrow \Big[\lim_{n \to \infty} x^{n} = x \succeq x^{0}\Big]
                    \] \[
                        \Big[\forall n, x^{0} \succeq x^{n}\Big] \Longrightarrow \Big[\lim_{n \to \infty} x^{n} = x,\, x^{0} \succeq x\Big]
                    \]
                    A counterexample is the \underline{Lexicographic Order} in $\mathbb{R}^{2}_{+}$: \[
                        x \succeq x^{0} \iff (x_1 > x_1^0) \wedge (x_1=x_1^0 \wedge x_2 \geq x_2^0)
                    \], e.g., $x^{0}=(5,2), x^{n} = (5+1/n,0)$ for any $n$, we have $5+1/n > 5 \Longrightarrow x^{n} \succeq x^{0}$, however, $\lim_{n \to \infty} x^{n} = (5,0) \nsucceq x^{0}$.
              \item Closed Set: A set $A$ is closed iff all convergent sequence in $A$ has their limits in $A$.
          \end{itemize}
    \item Local nonsatiation: $\forall x^{0} \in X, \forall \varepsilon > 0, \exists x^{1} \in X \Longrightarrow ||x^{0}-x^{1}||<\varepsilon \wedge x^{1} \succ x^{0}$ \begin{itemize}
              \item It means we can always find a direction to improve.
          \end{itemize}
    \item[(4')] Monotonicity: $\forall x^{0}, x^{1}: x^{1} \gg x^{0} \Longrightarrow x^{1} \succ x^{0}$. (All goods more than initial set)
    \item[(4'')] Strong Monotonicity: $\forall x^{0}, x^{1}, x^{1} \neq x^{0}: x^{1} \geq x^{0} \Longrightarrow x^{1} \succ x^{0}$. (Some goods more than initial set) \begin{itemize}
              \item We can prove (4'') $\Longrightarrow$ (4') $\Longrightarrow$ (4).
          \end{itemize}
    \item Convexity: $\forall x \in X, \{x \in X: x \succeq x^0\}$ is convex. \begin{itemize}
              \item The upper contour set is convex.
              \item Convexity also gives the information about how the MRS changes while increasing one good.
              \item $(3)+(4)+(5) \Longrightarrow$ a nice indifference curve: \begin{enumerate}
                        \item Continuity gives us a solid, unbroken line.
                        \item Monotonicity makes that line slope downwards from left to right.
                        \item Convexity bends that downward-sloping line so it's bowed in toward the origin.
                    \end{enumerate}
          \end{itemize}
    \item[(5')] Strict Convexity: $\forall \lambda \in (0,1)\, \forall x^0,x^1,x^2 \in X: x^1 \neq x^2, x^1 \succeq x^0 \Longrightarrow \lambda x^1 + (1-\lambda)x^2 \succ x^0$. \begin{itemize}
              \item It's just a strict version of $(5)$.
          \end{itemize}
\end{enumerate}

\section{Utility Function}

\begin{definition}
    Given a rational preference $\succeq $ on $X$. A \textbf{utility function} $u: X \longrightarrow \mathbb{R}$ that represents $\succeq $ is such that \[
        \forall x^0,x^1 \in X, x^0 \succeq x^1 \iff u(x^0) \geq u(x^0).
    \]
\end{definition}

\begin{proposition}
    \[
        \left\{\begin{array}{l}
            x^0 \succ x^1 \iff u(x^0) > u(x^1) \\
            x^0 \sim x^1 \iff u(x^0) = u(x^1)  \\
        \end{array}\right.
    \]
\end{proposition}

\begin{proposition}[ordinality]
    Given $u: X \longrightarrow \mathbb{R}$ representing $\succeq $ and $f: \mathbb{R} \longrightarrow \mathbb{R}$ any strict increasing function, then $v = f \circ u$ also represents $\succeq $.
\end{proposition}

\begin{proof}
    $\forall x^0,x^1 \in X, x^0 \succeq x^1$.

    $u$ represents $succeq$: $u(x^0) \geq u(x^1)$

    $f$ is strictly increasing: $f(u(x^0)) \geq f(u(x^1)) \iff v(x^1) \geq v(x^0)$
\end{proof}

\begin{remark*}
    The "ordinality" implicitly means the scale of the utility function means nothing.
\end{remark*}

\begin{theorem}[Debreu's]
    Given a \textbf{rational preference} $\succeq$ on $X \subset \mathbb{R}^{L}_{+}$. If $\succeq $ is \textbf{continuous}, then there \textbf{exists} a \underline{continuous utility function} $u: X \longrightarrow \mathbb{R}$ that represents $\succeq $.
\end{theorem}

\begin{proof}
    To give a simple and intuitive proof, we assume $X = \mathbb{R}^{L}_{+}$, and that the preference $\succeq$ is also \textbf{monotone}.

    The proof proceeds by construction. We define a function $u(x)$ by mapping each bundle $x$ to a specific amount of a reference bundle, $e=(1,\dots,1)$. We then show this function is well-defined, represents the preferences, and is continuous.

    \textbf{Step 1:} The utility function $u(x)$ is well-defined

    For any bundle $x \in \mathbb{R}^L_+$, we must show there exists a \textbf{unique} scalar $\lambda_x \ge 0$ such that $\lambda_x e \sim x$.

    To do this, we define the set $F_x = \{\lambda e : \lambda \geq 0, \lambda e \preceq x \}$. This set contains all bundles on the main diagonal that are weakly inferior to $x$. We will show this set is compact (closed and bounded) and non-empty.

    \begin{itemize}
        \item[\textit{(1)}] \textit{\underline{Proof of non-empty:}} The zero bundle $\mathbf{0} = 0 \cdot e$. By monotonicity, $x \succeq \mathbf{0}$ for any $x \in \mathbb{R}^L_+$. Thus, $0 \cdot e \in F_x$, and $F_x$ is not empty.

        \item[\textit{(2)}] \textit{\underline{Proof of bounded:}} By monotonicity, for any $x \in \mathbb{R}^L_+$, we can find a bundle $z$ such that $z \gg x$ (e.g., $z_i = x_i + 1$ for all $i$), which implies $z \succ x$. Now, for a sufficiently large scalar $\bar{\lambda}$, we will have $\bar{\lambda} e \gg z$. Monotonicity then implies $\bar{\lambda} e \succ z$. By transitivity, $\bar{\lambda} e \succ x$. This means that any vector $\lambda e \in F_x$ (where $\lambda e \preceq x$) must have its components bounded by those of $\bar{\lambda}e$. Thus, the set of vectors $F_x$ is bounded.

        \item[\textit{(3)}] \textit{\underline{Proof of closed:}} The set $F_x$ is the intersection of the ray $R = \{\lambda e : \lambda \ge 0\}$ and the lower contour set $LCS_x = \{y \in \mathbb{R}^L_+ : y \preceq x\}$. The ray $R$ is a closed set. Because the preference relation $\succeq$ is \textbf{continuous}, the set $LCS_x$ is also closed. The intersection of two closed sets is closed, therefore $F_x$ is closed.
    \end{itemize}

    Since $F_x$ is a non-empty, closed, and bounded subset of $\mathbb{R}^L_+$, it is \textbf{compact}.

    Now, consider the function $f: F_x \to \mathbb{R}$ defined by $f(\lambda e) = \lambda$. This function is continuous. By the \textbf{Weierstrass Extreme Value Theorem}, a continuous function on a compact set attains its maximum. Let this maximum value be $\lambda^*$, achieved at the point $\lambda^* e \in F_x$.

    By construction, since $\lambda^* e \in F_x$, we know $\lambda^* e \preceq x$.

    We must also show $\lambda^* e \succeq x$. We prove this by contradiction. Assume $\lambda^* e \prec x$. This means $\lambda^* e$ is in the \textbf{strict lower contour set} of $x$, $SLCS_x = \{y \in X : y \prec x\}$. Since $\succeq$ is continuous, the set $SLCS_x$ is \textbf{open}.

    Because $\lambda^* e \in SLCS_x$ and $SLCS_x$ is open, there exists an $\varepsilon > 0$ such that the bundle $(\lambda^* + \varepsilon)e$ is also in $SLCS_x$. This means $(\lambda^* + \varepsilon)e \prec x$, which in turn implies $(\lambda^* + \varepsilon)e \preceq x$. By definition, this means $(\lambda^* + \varepsilon)e \in F_x$.

    However, this implies that the function $f(\lambda e) = \lambda$ attains the value $\lambda^* + \varepsilon$ on the set $F_x$. This contradicts $\lambda^*$ being the maximum value of $f$ on $F_x$.

    Therefore, the assumption $\lambda^* e \prec x$ must be false. Since we have both $\lambda^* e \preceq x$ and $\neg(\lambda^* e \prec x)$, we conclude that $\lambda^* e \sim x$. Uniqueness of $\lambda^*$ follows from strict monotonicity.

    We can now define the utility of $x$ as this unique scalar: $u(x) \doteqdot \lambda^*$.

    \textbf{Step 2:} $u(x)$ represents $\succeq$

    We must prove that for any $x, y \in X$, we have $x \succeq y \iff u(x) \geq u(y)$.

    ($\Longrightarrow$) Assume $x \succeq y$. By definition of our function, we have $x \sim u(x)e$ and $y \sim u(y)e$.
    By transitivity, the relations $x \succeq y$ and $x \sim u(x)e$ imply $u(x)e \succeq y$.
    Again by transitivity, $u(x)e \succeq y$ and $y \sim u(y)e$ imply $u(x)e \succeq u(y)e$.
    By monotonicity, for bundles on the main diagonal, $u(x)e \succeq u(y)e$ holds if and only if $u(x) \geq u(y)$.

    ($\Longleftarrow$) Assume $u(x) \geq u(y)$.
    By monotonicity, this implies $u(x)e \succeq u(y)e$.
    By definition, we have $x \sim u(x)e$ and $y \sim u(y)e$.
    Using transitivity on the entire relation: $x \sim u(x)e \succeq u(y)e \sim y$. This implies $x \succeq y$.

    \textbf{Step 3:} $u(x)$ is continuous

    \textbf{A real-valued function is continuous if and only if the inverse images of all closed intervals are closed sets.} ("A real-valued function is continuous if and only if the inverse images of all open intervals are open sets." is also correct.) This is equivalent to showing that for any scalar $\bar{\lambda}$, the sets $\{x \in X : u(x) \geq \bar{\lambda}\}$ and $\{x \in X : u(x) \leq \bar{\lambda}\}$ are both closed.

    (a) Consider the set $A = \{x \in X : u(x) \geq \bar{\lambda}\}$. From Step 2, the condition $u(x) \geq \bar{\lambda}$ is equivalent to $x \succeq \bar{\lambda}e$. Thus, $A = \{x \in X : x \succeq \bar{\lambda}e\}$. This is, by definition, the \textbf{upper contour set} of the bundle $\bar{\lambda}e$. By the initial assumption that the preference relation $\succeq$ is continuous, all its upper contour sets are closed. Therefore, $A$ is a closed set.

    (b) Consider the set $B = \{x \in X : u(x) \leq \bar{\lambda}\}$. Similarly, the condition $u(x) \leq \bar{\lambda}$ is equivalent to $x \preceq \bar{\lambda}e$. Thus, $B = \{x \in X : x \preceq \bar{\lambda}e\}$. This is the \textbf{lower contour set} of the bundle $\bar{\lambda}e$. Since $\succeq$ is continuous, all its lower contour sets are also closed. Therefore, $B$ is a closed set.

    Since both the upper and lower contour sets of the function $u(x)$ are closed for any value in its range, the function $u(x)$ is \textbf{continuous}.
\end{proof}

\begin{proposition}
    Given preference $\succeq $ with a utility representation $u$, we have that: \begin{align*}
        \succeq \text{continuous}          & \iff \text{there exists $u'$ continuous representing $\succeq $} \\
        \succeq \text{locally nonsatiated} & \iff \text{$u$ has no local maxima}                              \\
        \succeq \text{strongly monotone}   & \iff \text{$u$ is strictly increasing}                           \\
        \succeq \text{(strict) convex}     & \iff \text{$u$ is (strict) quasiconcave}                         \\
    \end{align*}
\end{proposition}


\section{Walrasian Demand and Indirect Utility}

\begin{definition}
    For program [P] \begin{align*}
         & \max_{x \in X} u(x)      \\
         & s.t. \, x \in \beta(p,w)
    \end{align*}

    \textbf{Indirect Utility}: $v(p,w) \equiv \max_{x \in \beta(p,w)} u(x)$.

    \textbf{Walrasian Demand}: $x(p,w) \equiv \arg\max_{x \in \beta(p,w)} u(x)$.

\end{definition}

\begin{remark*}
    $v(p,w)$ is also called value function, while $x(p,w)$ is a correspondence.

    Difference between "function" and "correspondence": \begin{itemize}
        \item Function: one input only maps to one output.
        \item Correspondence: one input maps to a set of outputs.
    \end{itemize}
\end{remark*}

\begin{proposition}
    If $u(\cdot )$ is continuous, then for any $p \gg 0$ and $w \geq 0$ the program [P] \textbf{has a solution}.

    If $u(\cdot )$ is also strictly quasiconcave, then the solution is \textbf{unique}. Moreover, $x(p,w)$ is a \textbf{continuous} function.
\end{proposition}

\begin{proof} Existence and Uniqueness.

    Existence: (Weierstrass Theorem): $u(\cdot )$ is continuous and $\beta(p,w)$ is compact $\Longrightarrow$ exists a maxima.

    Uniqueness: By contradiction: $\exists x^1, x^2 \in X: x^1, x^2 \in \beta(p,w) $ and $u(x) \leq u(x^1), u(x)\leq u(x^2)\, \forall x \in \beta(p,w)$. Then we have $u(\lambda x^1 + (1-\lambda)x^2) > \min\{u(x^1), u(x^2)\} \Longrightarrow u(x') > u(x^1)$. It contradicts to $u(x) \leq u(x^1)\, \forall x \in \beta(p,w)$.

    Continuity: \underline{MGW 3. Appendix A}
\end{proof}

\begin{proposition}[Walras' Law]
    IF $u(\cdot )$ is continuous and locally nonsatiated, then for any $p \gg 0, w \gg 0$ and any $x \in x(p,w)$, we have $px = w$. \underline{If $x(\cdot )$ is a function} then \[
        px(p,w) = w
    \]
\end{proposition}

\begin{proposition}
    The Walrasian demand is homogenous of degree $0$ in $(p,w)$. \[
        x(p,w) = x(\lambda p, \lambda w), \forall \lambda > 0
    \]
\end{proposition}


\begin{definition}[Indirect Utility Function]
    \[
        v(p,w) \equiv \max_{x \in \beta(p,w)} u(x)
    \]
\end{definition}


\begin{proposition}
    If $u(\cdot )$ is continuous and locally non satiated, then $v(p,w)$ is: \begin{enumerate}
        \item continuous in $(p,w)$.
        \item homogenous of degree $0$ in $(p,w)$.
        \item non increasing in $p$ and strictly increasing in $w$.
        \item quasiconvex in $(p,w)$. \begin{itemize}
                  \item It means we prefer extreme not intermediate conditions.
              \end{itemize}
    \end{enumerate}
\end{proposition}

\begin{proof}
    (a) \underline{MGW 3. Appendix A}

    (b) Because $x(p,w)$ is homogenous of degree $0$.

    (c) Non-increasing in $p$: consider the corner solution. Strictly increasing in $w$, interior point and non-satiation.

    (d) At the intersection $x^*$ of two budget constrains, we have $p^1x^* = p^2 x^*$. Since \[
        \lambda p^1 x^* + (1-\lambda) p^2 x^* = p^1x^*
    \], we know the interval ($\lambda$) budget constrain also cross the intersection $x^*$. And the $\lambda$ budget constraint's slope is between those two original budget constrains. So any bundles in this now budget constrains is affordable in origin cases. So we have \[
        v(\lambda p^1 + (1-\lambda)p^2,\lambda w^1 + (1-\lambda) w^2) \leq \max\{v(p^1,w^1), v(p^2,w^2)\}
    \]
\end{proof}

\subsection{Comparative Statics}

\begin{enumerate}
    \item  Wealth: $\frac{\partial x_l}{\partial w} \geq 0 \Longrightarrow$ normal Good. $\frac{\partial x_l}{\partial w} \leq 0 \Longrightarrow$ inferior Good. \begin{itemize}
              \item Inferior good is a local definition. A good can become an inferior good from a normal good with wealth increasing.
          \end{itemize}
    \item Its own price: $\frac{\partial x_l}{\partial p_l} \leq 0 \Longrightarrow$ normal Good. $\frac{\partial x_l}{\partial p_l} \Longrightarrow $ Giffen Good.
    \item Another good's price: $\frac{\partial x_l}{\partial p_k} \geq 0 \Longleftarrow $ gross substitution. $\frac{\partial x_l}{\partial p_k} \Longrightarrow $ gross complements.
\end{enumerate}

\begin{example}[Giffen Good]
    An excellent and historically cited, though debated, example of a Giffen good is the case of potatoes during the Irish Potato Famine in the 19th century. More recent and empirically supported evidence points to rice in certain impoverished regions of China.

    In these specific contexts, when the price of the staple food (potatoes or rice) increased, impoverished households paradoxically consumed more of it. The reasoning is that as the price of the staple rose, the household's already limited budget was squeezed, forcing them to cut back on more expensive "luxury" foods like meat. To maintain their necessary caloric intake, they had to purchase more of the now higher-priced staple, as it was still the most affordable source of calories.
\end{example}

\section{Hicksian Demand and Expenditure Function}

\begin{definition}
    For program [D]\begin{align*}
         & \min_{x \in X} px      \\
         & s.t. u(x) \geq \bar{u}
    \end{align*}

    \textbf{Expenditure Function}: $e(p,\bar{u}) \equiv \min_{u(x)\geq \bar{u}}$.

    \textbf{Hicksian/Compensated Demand}: $h(p,\bar{u}) \equiv \arg \min_{u(x) \geq \bar{u}} px$.
\end{definition}

\begin{remark*}
    Hicksian/Compensated Demand is a correspondence.
\end{remark*}


\begin{proposition}
    If $u(\cdot )$ is continuous and $p \gg 0$, then [D] has a solution.

    If $u(\cdot )$ is also strictly quasiconcave, then the solution is unique. Moreover, in this case $h(p,u)$ is a continuous function.
\end{proposition}

\begin{proof} Existence and Uniqueness.

    Existence: $p \gg 0 \Longrightarrow x$ is closed but not bounded. We can construct a new program \[
        \left\{\begin{array}{l}
            \min_{x \in X} px \\
            s.t. \left\{\begin{array}{l}
                            u(x) \geq \bar{u}                                        \\
                            px \leq p x_0, \, x_0 \in \{x \in X: u(x) \geq \bar{u}\} \\
                        \end{array}\right.
        \end{array}\right.
    \]
    Now, in our new program, $x$ is in a compact set. With Weierstrass Theorem, we can know there is a solution.

    Uniqueness: By contradiction. $\exists x^1, x^2 \in X, x^1 \neq x^2, u(x^1) \geq \bar{u}, u(x_2) \geq \bar{u}$. And \[
        p x^1 = p^2 x \leq px, \forall x \in \{x \in X: u(x) \geq \bar{u}\}.
    \]

    $\Longrightarrow^{SQC} \lambda \in (0,1), u(x^\lambda) > u(x^1), u(x^\lambda) > u(x^2)$.

    $\Longrightarrow^{Continuity} \exists \delta \in (0,1), u(\delta x^\lambda) > \bar{u}$. However, $p \delta x^\lambda < p x^1 = px^2$ !!
\end{proof}



\begin{proposition}
    If $u(\cdot )$ is continuous then $\forall x \in h(p,u)$ we have that $u(x) = u$. If $h(\cdot )$ is a function, then \[
        u(h(p,u)) = u
    \]
\end{proposition}

\begin{proposition}
    The Hicksian demand is homogenous of degree zero in $p$. \[
        h(p,u) = h(\lambda p, u), \, \forall \lambda > 0
    \]
\end{proposition}

\begin{proposition}
    [\textbf{Compensated Law of Demand}]
    If $u(\cdot )$ is continuous and the Hicksian demand is a function, then $\forall p',p'' \gg 0$ we have that \[
        (p'-p'')\cdot [h(p',u) - h(p'',u)] \leq 0
    \]
\end{proposition}

\begin{proof}
    [Intuitive Proof]
    Let $p_l' = p_l'', l = 2,3,\dots$ and $p_1' \neq p_1''$. Then we have \[
        (p'-p'')\cdot [h(p',u) - h(p'',u)] \leq 0 \iff (p_1'-p_1'')\cdot [h(p_1',u) - h(p_1'',u)] \leq 0
    \]
\end{proof}

\begin{proof} We have
    \[
        h(p',u) p' \leq h(p'',u)p'
    \] since $e(p',u) \leq p'x, \forall x \in \{x \in X: u(x) \geq u\}$.

    Similarly, \[
        h(p'',u) p'' \leq h(p',u) p'' \iff - h(p',u) p'' \leq - h(p'',u) p''
    \]
    Plus these two: \[
        (p'-p'')h(p',u) \leq (p'-p'')h(p'',u) \iff (p'-p'')\cdot [h(p',u) - h(p'',u)] \leq 0
    \]
\end{proof}

\begin{proposition}
    If $u(\cdot )$ is continuous, satisfies LNS, and $p \gg 0$, then $e(p,u)$ is: \begin{enumerate}
        \item homogenous of degree $1$ in $p$: $e(\lambda p,u) = \lambda e(p,u)$.
        \item strictly increasing in $u$, and non-decreasing $p$.
        \item continuous in $(p,u)$.
        \item concave in $p$.
    \end{enumerate}
\end{proposition}

\begin{proof}
    (1) Because $h(p,u)$ is homogenous of degree $1$ in $p$.

    (2) LNS and Continuity; corner solution.

    (3) \underline{MGW 3. Appendix A}

    (4) \[
        p^\lambda h(p^\lambda,u) = \lambda p^1 h(p^\lambda, u) + (1-\lambda)p^2 h(p^\lambda, u)
    \]
    Since $p^1 h(p^\lambda, u) \geq p^1 h(p^1,u)$ and $p^2 h(p^\lambda,u) \geq p^2 h(p^2,u)$, then we have \[
        p^\lambda h(p^\lambda,u) \geq  \lambda p^1 h(p^1, u) + (1-\lambda)p^2 h(p^2, u)
    \]

    Intuition of (4): If $p$ increases, the expenditure will not increase at a same degree, because by rearrange the composition of the consumption bundles, we can alleviate this effect..
\end{proof}

\section{Duality}

\begin{proposition}
    [Duality]
    Consider $u(\cdot )$ continuous and satisfying LNS, and $p \gg 0$. Then: \begin{enumerate}
        \item if $x^0$ solves [P] for $w > 0$, then $x^0$ solves [D] for $\bar{u} = u(x^0) = v(p,w)$. Moreover, \begin{align*}
                   & e(p,v(p,w)) = w      \\
                   & h(p,v(p,w)) = x(p,w)
              \end{align*}
        \item if $x^0$ solves [D] for $\bar{u} > u(0)$ then $x^0$ solves [P] for $w = px^0 = e(p,\bar{u})$. Moreover \begin{align*}
                   & v(p,e(p,\bar{u})) = \bar{u}      \\
                   & x(p,e(p,\bar{u})) = h(p,\bar{u})
              \end{align*}
    \end{enumerate}
\end{proposition}

\begin{proof}[Proof of (1)]

    First Step: Prove $e(p,v(p,w)) \leq w$

    Given $(p,w)$, we have optimal consumption bundle $x^0 \in x(p,w)$, and $u(x^0) = v(p,w)$. We have \[
        e(p,v(p,w)) \leq px^0 = w
    \], because $e(p,v(p,w))$ is a minimum.

    Second Step: Prove $e(p,v(p,w)) \geq w$.

    By contradiction, assume $e(p,v(p,w)) < w = px^0$, then $\exists x^1 p < w$ and $u(x^1) \geq u(x^0)$.

    With LNS, $\exists y \in  B(x_1, r), u(y) > u(x^1), py < w$. It contradicts to the setting $u(x^0)$ maximizing the utility under $w$.

    Now we have both $e(p,v(p,w)) \leq w$ and $e(p,v(p,w)) \geq w$. Then $e(p,v(p,w)) = w$

    Since $e(p,v(p,w)) = p h(p,v(p,w))$ and $w = p x(p,w)$, it is trivial that \[
        h(p,v(p,w)) = x(p,w)
    \]
\end{proof}

\begin{proof}[Proof of (2)]
    First Step: Prove $v(p,e(p,\bar{u})) \geq \bar{u}$.

    Given $p,\bar{u}$, we have optimal consumption bundle $x^0 \in h(p,\bar{u})$, and $px^0 = e(p,\bar{u})$. Then \[
        v(p,e(p,\bar{u})) \geq u(x^0)
    \], because $v(p,e(p,\bar{u}))$ is a maxima.

    Second Step: Prove $v(p,e(p,\bar{u})) \leq \bar{u}$.

    By contradiction, assume $v(p,e(p,\bar{u})) > \bar{u}$, then $\exists x^1 \in x(p,e(p,\bar{u})),  u(x^1) > \bar{u}$ and $px^1 \leq px^0$.

    With LNS, $\exists y \in B(x^1,r), u(y) > u(x^1)>\bar{u}$. But $x^1$ should satisfies $u(x^1) \geq u(x), \forall x \in \{x \in X: px \leq px^0\}$, it contradicts to $u(y)>u(x^1)$.

    Now we have both $v(p,e(p,\bar{u})) \geq \bar{u}$ and $v(p,e(p,\bar{u})) \leq \bar{u}$, so $v(p,e(p,\bar{u})) = \bar{u}$.

    And since $u(x(p,e(p,\bar{u}))) = v(p,e(p,\bar{u}))$ and $u(h(p,\bar{u})) = \bar{u}$, it is trivial that \[
        x(p,e(p,\bar{u})) = h(p,\bar{u})
    \]
\end{proof}

\begin{figure}[h!]
    \centering
    \begin{tikzpicture}[
            font=\small, >=Latex, node distance=15mm,
            box/.style={draw,rounded corners=2pt,align=center,minimum width=4cm,minimum height=13mm},
            lab/.style={fill=white,inner sep=1pt}
        ]

        % ---- Nodes (left column) ----
        \node[box] (primal) {Primal Problem\\ $\max\, u(x)\ \text{s.t.}\ px\le w$};
        \node[box, below=20mm of primal] (walras) {Walrasian Demand\\ $x(p,w)$};
        \node[box, below=20mm of walras] (v) {Indirect Utility Function\\ $v(p,w)$};

        % ---- Nodes (right column) ----
        \node[box, right=50mm of primal] (dual) {Dual Problem\\ $\min\, px\ \text{s.t.}\ u(x)\ge u$};
        \node[box, below=20mm of dual] (hicks) {Hicksian Demand\\ $h(p,u)$};
        \node[box, below=20mm of hicks] (e) {Expenditure Function\\ $e(p,u)$};

        % ---- Top duality between problems ----
        \draw[<->] (primal) -- node[lab,above]{dual problems} (dual);

        % ---- Vertical “Solution” arrows ----
        \draw[->] (primal.south) -- node[lab,left,pos=.45]{Solution} (walras.north);
        \draw[->] (dual.south)   -- node[lab,right,pos=.45]{Solution} (hicks.north);

        % ---- Slutsky (right <-> left) ----
        \draw[<->] (hicks.west) -- node[lab,above]{Slutsky Equation} (walras.east);

        % ---- Down to v and e ----
        \draw[->] (walras.south) -- (v.north);
        \draw[->] (hicks.south)  -- (e.north);

        % ---- Roy's Identity (dashed, upward) ----
        \draw[dashed,->] ($(v.north) + (-0.5,0)$) -- node[lab,left,xshift=-2mm]{Roy's Identity} ($(walras.south) + (-0.5,0)$);

        % ---- Shepard's Lemma (dashed, upward) ----
        \draw[dashed,->] ($(e.north) + (0.5,0)$) -- node[lab,right,xshift=2mm]{Shepard's Lemma} ($(hicks.south)+ (0.5,0)$);

        % ---- Inverse function between v and e ----
        \draw[<->] (v) -- node[lab,below]{Inverse function} (e);

        % ---- Cross “duality” relations (two separate U-shaped curves, with white-backed labels) ----
        % x(p,w) = h(p, v(p,w)) 
        \draw[<-]
        ($(walras.east)+(0,-0.5)$)
        .. controls ($ (walras.east) + (0.1,-2.5) $) ..
        node[lab,below,pos=.52,yshift=-2pt] {$x(p,w)=h\!\big(p,\,v(p,w)\big)$}
        ($(hicks.west) + (0,-0.5)$);

        % h(p,u) = x(p, e(p,u))
        \draw[<-]
        ($(hicks.west) + (0,-0.5)$)
        .. controls ($ (hicks.west) + (-0.1,-2.5) $) ..
        node[lab,below,pos=.52,yshift=-2pt] {$h(p,u)=x\!\big(p,\,e(p,u)\big)$}
        ($(walras.east) + (0,-0.5)$);

        % label “duality” near the crossing (with white background)
        \node[lab] at ($(walras)!0.5!(hicks) + (0,-1)$) {\textbf{duality}};

    \end{tikzpicture}

    \caption{Duality}
    \label{fig:duality}
\end{figure}
