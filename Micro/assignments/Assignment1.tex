% --- LaTeX Homework Template - S. Venkatraman ---

% --- Set document class and font size ---

\documentclass[letterpaper, 11pt]{article}

% --- Package imports ---

\usepackage{
  amsmath, amsthm, amssymb, mathtools, dsfont,	  % Math typesetting
  graphicx, wrapfig, subfig, float,                  % Figures and graphics formatting
  listings, color, inconsolata, pythonhighlight,     % Code formatting
  fancyhdr, sectsty, hyperref, enumerate, enumitem } % Headers/footers, section fonts, links, lists

% --- Page layout settings ---

% Set page margins
\usepackage[left=1.35in, right=1.35in, bottom=1in, top=1.1in, headsep=0.2in]{geometry}

% Anchor footnotes to the bottom of the page
\usepackage[bottom]{footmisc}
\usepackage{xcolor}

% Set line spacing
\renewcommand{\baselinestretch}{1}

% Set spacing between paragraphs
\setlength{\parskip}{1.5mm}

% Allow multi-line equations to break onto the next page
\allowdisplaybreaks

% Enumerated lists: make numbers flush left, with parentheses around them
\setlist[enumerate]{wide=0pt, leftmargin=21pt, labelwidth=0pt, align=left}
\setenumerate[1]{label={(\arabic*)}}

% --- Page formatting settings ---

% Set link colors for labeled items (blue) and citations (red)
\hypersetup{colorlinks=true, linkcolor=blue, citecolor=red}

% Make reference section title font smaller
\renewcommand{\refname}{\large\bf{References}}

% --- Settings for printing computer code ---

% Define colors for green text (comments), grey text (line numbers),
% and green frame around code
\definecolor{greenText}{rgb}{0.5, 0.7, 0.5}
\definecolor{greyText}{rgb}{0.5, 0.5, 0.5}
\definecolor{codeFrame}{rgb}{0.5, 0.7, 0.5}

% Define code settings
\lstdefinestyle{code} {
  frame=single, rulecolor=\color{codeFrame},            % Include a green frame around the code
  numbers=left,                                         % Include line numbers
  numbersep=8pt,                                        % Add space between line numbers and frame
  numberstyle=\tiny\color{greyText},                    % Line number font size (tiny) and color (grey)
  commentstyle=\color{greenText},                       % Put comments in green text
  basicstyle=\linespread{1.1}\ttfamily\footnotesize,    % Set code line spacing
  keywordstyle=\ttfamily\footnotesize,                  % No special formatting for keywords
  showstringspaces=false,                               % No marks for spaces
  xleftmargin=1.95em,                                   % Align code frame with main text
  framexleftmargin=1.6em,                               % Extend frame left margin to include line numbers
  breaklines=true,                                      % Wrap long lines of code
  postbreak=\mbox{\textcolor{greenText}{$\hookto$}\space} % Mark wrapped lines with an arrow
}

% Set all code listings to be styled with the above settings
\lstset{style=code}

% --- Math/Statistics commands ---

% Add a reference number to a single line of a multi-line equation
% Usage: "\numberthis\label{labelNameHere}" in an align or gather environment
\newcommand\numberthis{\addtocounter{equation}{1}\tag{\theequation}}

% Shortcut for bold text in math mode, e.g. $\b{X}$
\let\b\mathbf

% Shortcut for bold Greek letters, e.g. $\bg{\beta}$
\let\bg\boldsymbol

% Shortcut for calligraphic script, e.g. %\mc{M}$
\let\mc\mathcal

% \mathscr{(letter here)} is sometimes used to denote vector spaces
\usepackage[mathscr]{euscript}

% Convergence: right arrow with optional text on top
% E.g. $\converge[w]$ for weak convergence
\newcommand{\converge}[1][]{\xto{#1}}

% Normal distribution: arguments are the mean and variance
% E.g. $\normal{\mu}{\sigma}$
\newcommand{\normal}[2]{\mathcal{N}\left(#1,#2\right)}

% Uniform distribution: arguments are the left and right endpoints
% E.g. $\unif{0}{1}$
\newcommand{\unif}[2]{\text{Uniform}(#1,#2)}

% Independent and identically distributed random variables
% E.g. $ X_1,...,X_n \iid \normal{0}{1}$
\newcommand{\iid}{\stackrel{\smash{\text{iid}}}{\sim}}

% Equality: equals sign with optional text on top
% E.g. $X \equals[d] Y$ for equality in distribution
\newcommand{\equals}[1][]{\stackrel{\smash{#1}}{=}}

% Math mode symbols for common sets and spaces. Example usage: $\R$
\newcommand{\R}{\mathbb{R}}   % Real numbers
\newcommand{\C}{\mathbb{C}}   % Complex numbers
\newcommand{\Q}{\mathbb{Q}}   % Rational numbers
\newcommand{\Z}{\mathbb{Z}}   % Integers
\newcommand{\N}{\mathbb{N}}   % Natural numbers
\newcommand{\F}{\mathcal{F}}  % Calligraphic F for a sigma algebra
\newcommand{\El}{\mathcal{L}} % Calligraphic L, e.g. for L^p spaces

% Math mode symbols for probability
\newcommand{\pr}{\mathbb{P}}    % Probability measure
\newcommand{\E}{\mathbb{E}}     % Expectation, e.g. $\E(X)$
\newcommand{\var}{\text{Var}}   % Variance, e.g. $\var(X)$
\newcommand{\cov}{\text{Cov}}   % Covariance, e.g. $\cov(X,Y)$
\newcommand{\corr}{\text{Corr}} % Correlation, e.g. $\corr(X,Y)$
\newcommand{\B}{\mathcal{B}}    % Borel sigma-algebra

% Other miscellaneous symbols
\newcommand{\tth}{\text{th}}	% Non-italicized 'th', e.g. $n^\tth$
\newcommand{\Oh}{\mathcal{O}}	% Big-O notation, e.g. $\O(n)$
\newcommand{\1}{\mathds{1}}	% Indicator function, e.g. $\1_A$

% Additional commands for math mode
\DeclareMathOperator*{\argmax}{argmax}    % Argmax, e.g. $\argmax_{x\in[0,1]} f(x)$
\DeclareMathOperator*{\argmin}{argmin}    % Argmin, e.g. $\argmin_{x\in[0,1]} f(x)$
\DeclareMathOperator*{\spann}{Span}       % Span, e.g. $\spann\{X_1,...,X_n\}$
\DeclareMathOperator*{\bias}{Bias}        % Bias, e.g. $\bias(\hat\theta)$
\DeclareMathOperator*{\ran}{ran}          % Range of an operator, e.g. $\ran(T) 
\DeclareMathOperator*{\dv}{d\!}           % Non-italicized 'with respect to', e.g. $\int f(x) \dv x$
\DeclareMathOperator*{\diag}{diag}        % Diagonal of a matrix, e.g. $\diag(M)$
\DeclareMathOperator*{\trace}{trace}      % Trace of a matrix, e.g. $\trace(M)$

% Numbered theorem, lemma, etc. settings - e.g., a definition, lemma, and theorem appearing in that 
% order in Section 2 will be numbered Definition 2.1, Lemma 2.2, Theorem 2.3. 
% Example usage: \begin{theorem}[Name of theorem] Theorem statement \end{theorem}
\theoremstyle{definition}
\newtheorem{theorem}{Theorem}[section]
\newtheorem{proposition}[theorem]{Proposition}
\newtheorem{lemma}[theorem]{Lemma}
\newtheorem{corollary}[theorem]{Corollary}
\newtheorem{definition}[theorem]{Definition}
\newtheorem{example}[theorem]{Example}
\newtheorem{remark}[theorem]{Remark}

% Un-numbered theorem, lemma, etc. settings
% Example usage: \begin{lemma*}[Name of lemma] Lemma statement \end{lemma*}
\newtheorem*{theorem*}{Theorem}
\newtheorem*{proposition*}{Proposition}
\newtheorem*{lemma*}{Lemma}
\newtheorem*{corollary*}{Corollary}
\newtheorem*{definition*}{Definition}
\newtheorem*{example*}{Example}
\newtheorem*{remark*}{Remark}
\newtheorem*{claim}{Claim}

% --- Left/right header text (to appear on every page) ---

% Include a line underneath the header, no footer line
\pagestyle{fancy}
\renewcommand{\footrulewidth}{0pt}
\renewcommand{\headrulewidth}{0.4pt}

% Left header text: course name/assignment number
\lhead{Micro I - Problem Set 1}

% Right header text: your name
\rhead{Zian Gong}

% --- Document starts here ---

\begin{document}

\textbf{Problem 1}

(a)

\begin{proof}
  (rational relationship $\Rightarrow$ pre-order)

  A rational relationship satisfies transitivity, we just need to show it also satisfies reflexivity.

  According to completeness of rational relationship, we have \[
    \forall x^1, x^2 \in X \Longrightarrow x^1 \succeq x^2 \vee x^2 \succeq x^1.
  \]
  Let $x^1,x^2$ both equal to an arbitrary $x \in X$, then we have \[
    x \succeq x \vee x \succeq x \Longrightarrow x \succeq x.
  \]
  Then the reflexivity is proved.
\end{proof}

\begin{proof}
  (rational relationship $\nLeftarrow$ pre-order)

  Pre-order satisfies transitivity and reflexivity, but completeness cannot be derived from these two properties. Because transitivity relies on a given relationship between different elements, reflexivity is a relationship of an element and itself. Thus, they cannot assure there is a certain relationship between two arbitrary elements.
\end{proof}
\textcolor{red}{
  Counterexample: 
  $X = \{a, b\}$ and $(X, \succeq) = \{(a,a), (b,b)\}$
}

(b)


\begin{proof}
  We have $x^1 \succ x^2$ and $x^2 \succ x^3$, which means
  \[
    x^1 \succ x^2 \iff x^1 \succeq x^2 \wedge \neg (x^2 \succeq x^1)
  \]
  \[
    x^2 \succ x^3 \iff x^2 \succeq x^3 \wedge \neg (x^2 \succeq x^3)
  \]

  To prove $x^1 \succeq x^3 \wedge \neg (x^3 \succeq x^1)$:

  From transitivity, $x^1 \succeq x^2 \wedge x^2 \succeq x^3 \Longrightarrow  x^1 \succeq x^3$. Next step, we need to prove $\neg (x^3 \succeq x^1)$ is true.

  By contradiction, we can assume $x^3 \succeq x^1$ is true.

  We already have $x^1 \succeq x^2$, by transitivity, we get $x^3 \succeq x^2$, which does not satisfy $\neg (x^3 \succeq x^2)$. Then we have $\neg (x^3 \succeq x^1)$ is true.
\end{proof}

\textbf{Problem 2}

If $X$ is a finite set, there always exists an $x^0 \in X$ such that $x^0 \succeq x, \forall x \in X$.

\begin{proof}
  By randomly choose two elements of $X$: $x^i, x^j$, with completeness we have $x^i \succeq x^j \vee x^j \succeq x^i$, without loss of generality, we can assume $x^i \succeq x^j$. Then choose another element $x^k$, we with completeness and transitivity we could have $x^i \succeq x^j \succeq x^k$ or $x^i \succeq x^k \succeq x^j$ or $x^k \succeq x^i \succeq x^j$. Without loss of generality, we can assume $x^i \succeq x^j \succeq x^k$. By repeating $n-1$ times, where $n$ is the size of $X$, we can have an ordered sequence of $X = (x^1, x^2, \dots, x^n)$, where $x^1 \succeq x^2 \succeq \dots \succeq x^n$.

  Let $x^0 = x^1$, we have $x^0 \succeq x, \forall x \in X$.
\end{proof}

If $X$ is an infinite set, the assertion is not always true. For example, let $X = \mathbb{N}$, define $\succeq$ as $\ge$. $\forall x \in \mathbb{N}, \exists x' = x+1 \in \mathbb{N} \Longrightarrow x' \succeq x$.

So there does not exist the $x^0 \in X$.


\textbf{Problem 3}

To prove it is a rational preference, we only need to prove it satisfies completeness and transitivity.

\begin{proof}
  (completeness) $\forall x,y \in X$.

  If $x \neq 13 \wedge y \neq 13$, if $x \succeq y$ then $x \geq y$; if $y \succeq x$ then $y \geq x$.

  If $x \neq 13 \wedge y = 13$, we have $x \succeq y$.

  If $x = 13 \wedge y \neq 13$, we have $y \succeq x$.

  If $x = 13 \wedge y = 13$, we have $x \succeq  y$.

  In conclusion, $\forall x, y \in X$, we have $x \succeq y \vee y \succeq x$.
\end{proof}

\begin{proof}
  (transitivity) $\forall x,y,z \in X$, if $x \succeq y$ and $y \succeq z$.

  If $z = 13$, $x \succeq z$ is true.

  If $z \neq 13$, from $y \succeq z$, we know $y \neq 13 \wedge y \geq z$. Similarly, $x \neq 13 \wedge x \geq y$. We have $x \geq y \geq z$. Since $x \neq 13 \wedge z \neq 13$, $x \succeq z$ is true.

  In conclusion, $\forall x,y,z \in X$, if $x \succeq y$ and $y \succeq z$, we can get $x \succeq z$.
\end{proof}


% --- Document ends here ---

\end{document}

