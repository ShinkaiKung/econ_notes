
We have 3 equivalent approaches to computing GDP: \begin{itemize}
    \item Expenditure approach: \[
              Y_t = C_t + I_t + G_t + \underbrace{(X_t - M_t)}_{\substack{\text{net exports}}}, \qquad\text{with } K_{t+1} = (1-\delta)K_t + I_t.
          \]
    \item Income approach: \[
              Y_t = R_tK_t+W_tN_t+\Pi_t
          \]
          where: \begin{itemize}
              \item $R_tK_t$ are payments to capital owners (rental income).
              \item $W_tN_t$ are payments to workers (labor income).
              \item $\Pi_t$ are payments to firm owners (firm profits).
          \end{itemize}
    \item Production approach: add up sales minus intermediate expenses in all stages of production.
\end{itemize}

\section{Economic Growth}

``The consequences for human welfare involved in questions like these [of economic growth] are simply staggering: Once one starts to think about them, \textbf{it is hard to think about anything else}. '' --- Robert E. Lucas, “On the Mechanics of Economic Development” [JME 1988]

\textbf{Central Question}: Why some countries grow more than others?

\section{The Stylized Facts of Growth} --- by Kaldor (1961)

Economies at the technology frontier are characterized by these properties \begin{enumerate}
    \item Per capita output grows at a constant rate ($\dot{Y}_t/Y_t$).
    \item Physical capital per worker grows at a constant rate ($\dot{K}_t/K_t$).
    \item The rate of return of capital is nearly constant ($R_t$).
    \item The ratio of capital to output is nearly constant ($K_t/Y_t$).
    \item The shares of labor and capital in national income are constant ($W_tN_t/Y_t$).
    \item The growth rate of output per worker differs substantially across countries.
\end{enumerate}

An economy displaying the \textbf{first 5 facts} is said to be in a \textbf{Balanced Growth Path (BGP)}.

A good theory of growth should be consistent with these facts.

The \textbf{neo-classical growth model} is designed to account for Kaldor's stylized facts. However, by design, this model cannot explain \textbf{fact (6)}. Therefore, it is not a good model to explain why some similar economies differ in long term.

\begin{remark*} However:
    these facts are not so solid over time and across countries.
\end{remark*}
