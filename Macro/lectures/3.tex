

There are at least 3 sources of dissatisfaction with the Solow model. \begin{enumerate}
    \item Differences in long run growth depend on technical progress. \begin{itemize}
              \item We need a theory of technical progress. (Endogenous Growth Theory)
          \end{itemize}
    \item The saving rate is a key determinant of capital accumulation and per capita output. \begin{itemize}
              \item We need a theory of household consumption and saving. (Ramsey Model)
          \end{itemize}
    \item Population growth is a key determinant of per capita output in the long run. \begin{itemize}
              \item We need a theory of fertility to endogenize it.
          \end{itemize}
\end{enumerate}

\section{Settings}

\textbf{Environment}: \begin{itemize}
    \item Single homogeneous good that can be either invested or consumed.
    \item Closed economy.
    \item \textbf{Households choose how much to consume and save every period}.
    \item Population grows at an exogenous and constant rate, $n$.
    \item Technology level grows at an exogenous and constant rate, $x$.
\end{itemize}


\textbf{Decentralized solution}: \begin{itemize}
    \item There is a representative household and a representative firm.
    \item There are three (competitive) markets: capital, labor, final good.
\end{itemize}

\textbf{Households}: \begin{itemize}
    \item The live forever ($\infty $ periods) and discount future utilities at rate $\rho$.
    \item We have to think of them as families (or dynasties) where each member cares about herself as much as about her offspring.
    \item The size of the representative family at a given period $t$ is $N_t = N_0 e^{nt}$.\\(where we can normalize $N_0 = 1$)
    \item They own the \underline{production factors} and rent them to firms.
    \item They buy the final good from firms.
\end{itemize}

\textbf{Firms}: \begin{itemize}
    \item Representative firm characterized by a neoclassical production function.
    \item They rent capital and labor from households to produce the homogeneous final good.
    \item They sell the final good to households.
\end{itemize}

\section{Household Problem}

\subsection{Budget constraint}

Every period, households supply one unit of labor \textbf{inelastically} in exchange of a wage $w_t$, and choose how much to consume $C_t$ and how much to save.

When saving, household accumulate wealth (financial assets) $A_t$. Assets generate a return $r_t$. $A_t > 0$ are savings, $A_t < 0$ implies that the household is in debt.

Then, the flow budget constraint every period is given by: \[
    \dot{A}_t = N_tw_t+r_tA_t-C_t \Longrightarrow c_t + \dot{a}_t + na_t = w_t + r_ta_t
\]
\begin{remark*}[Interpretation]
    The resources obtained from working ($w_t$) and the interest payments on asset holdings ($r_ta_t$) have to be used for consumption ($c_t$), increasing the asset holding ($\dot{a}_t$) and providing assets for the new family members ($na_t$).
\end{remark*}

\subsection{Preferences and Maximization} The period utility $u(c_t)$ has the usual properties: $u_c > 0$ and $u_{cc} < 0$.

A representative \textbf{household} chooses the path of consumption $\{c_t\}_{t=0}^{\infty }$ that maximizes the discounted sum of utilities for \textbf{all its members}: \[
    \int_{0}^{\infty } u(c_t)N_te^{-\rho t} \, dt
\]

In doing so, the household has to: \begin{itemize}
    \item satisfy the sequence of period budget constants;
    \item take the sequence of prices $\{r_t,w_t\}_{t=0}^{\infty }$ as given (competitive equilibrium); \\ (In a perfectly competitive market, individuals cannot decide the prices)
    \item take initial wealth $a_0$ as given.
\end{itemize}

\subsection{No Ponzi Game Condition}

To prevent the household achieve infinite utility by accumulating more and more debt, we should require the \textbf{No Ponzi Game (NPG) Condition}: \[
    \lim_{t \to \infty} e^{-(\bar{r}-n)t}a_t \geq 0
\] where $\bar{r}_t$ is the \underline{\textit{average interest}} rate between dates $0$ and $t$, defined as: $\bar{r}_t \equiv \frac{1}{t}\int_{0}^{t} r_s \, ds$.

\begin{remark*}[Understanding the NPG Condition]
    Consider the flow budget constant: \[
        \dot{a}_t = (r_t - n)a_t + w_t - c_t
    \]

    Solving the differential equation for some terminal period $t=T$ yields: \[
        \underbrace{\int_{0}^{T} e^{-(\bar{r}_t-n)t}c_t \, dt}_{\text{PDV of consumption up to T}} + \underbrace{e^{-(\bar{r}_t-n)T}a_T}_{\text{PDV wealth left at T}} = \underbrace{a_0 + \int_{0}^{T} e^{-(\bar{r}_t-n)t}w_t \, dt}_{\text{PDV of resources generated up to T}}
    \]
    (\textit{The detailed derivation process are in the later The Consumption Function section.})

    As $T$ goes to infinity, the NGP condition requires: $\lim_{T \to \infty} e^{-(\bar{r}_T - n)T}a_T \geq 0$.

    Hence, \[
        \underbrace{\int_{0}^{\infty } e^{-(\bar{r}_t-n)t}c_t \, dt}_{\text{PDV of lifetime consumption}} \leq \underbrace{a_0 + \int_{0}^{\infty } e^{-(\bar{r}_t-n)t}w_t \, dt}_{\text{PDV of lifetime resources}}
    \]
    If households ended life with a debt position, they would be able to consume above the level of their resources (this is what people want to achieve when running Ponzi schemes)
\end{remark*}

\begin{remark*}[$e^{-\bar{r}_t t} A_t =  e^{-(\bar{r}_t - n)t} a_t\geq 0$ or $e^{-\bar{r}_tt}a_t \geq 0$? \textbf{Why we cannot consider debt in per capita scale directly?}]
    The short answer is: We \emph{can} and \emph{do} consider debt on a per capita scale, but we cannot do it ``directly'' or naively. We must use a crucial adjustment because a direct, unadjusted analysis would ignore the fundamental economic effects of population growth, leading to incorrect conclusions.

    The correct approach requires adjusting the interest rate $r$ to an ``effective rate'' of $(r-n)$. Here is a complete explanation using two perspectives.

    \textbf{Angle 1: From a Macro View, Total Debt Grows Even When Per Capita Debt Is Stable}

    The fundamental ``No-Ponzi Game'' rule applies to the \textbf{total debt} of an economy, not an individual's average debt. Population growth causes total debt to grow even if every individual's financial situation remains unchanged.

    \textbf{Consider our island example:}
    \begin{itemize}
        \item \textbf{Initial state:} There are 100 people, and the average debt per person is stable at \textbf{100 euros}. The island's \textbf{total debt} is $100 \text{ people} \times 100 \text{ euros/person} = \textbf{10,000 euros}$.
        \item \textbf{After one year:} The population grows by 2\% ($n=0.02$) to 102 people. To maintain the same economic structure, the average debt per person remains \textbf{100 euros}.
        \item However, the new \textbf{total debt} is now $102 \text{ people} \times 100 \text{ euros/person} = \textbf{10,200 euros}$.
    \end{itemize}
    This demonstrates the core problem of a ``direct'' per capita analysis: If you look only at the stable per capita debt, you might falsely conclude that the debt situation is static. But in reality, the total debt---the quantity that the No-Ponzi rule actually governs---has grown by 2\%. This 2\% growth is an ``automatic'' inflation of total debt caused purely by population growth $n$.

    \textbf{Angle 2: From an Individual's View, the Effective Rate of Return Is Lowered by ``Dilution''}

    When we build economic models, the decisions (to save, consume, or borrow) are made from the perspective of an individual. For an individual, the real return on their assets is ``diluted'' by new people entering the economy.

    \textbf{Consider the savings perspective of the same example:}
    \begin{itemize}
        \item \textbf{Initial state:} Each of the 100 people has \textbf{100 euros} in savings, and the market interest rate $r$ is \textbf{5\%}.
        \item \textbf{After one year:} The total savings pool grows to $10,000 \text{ euros} \times (1 + 5\%) = \textbf{10,500 euros}$.
        \item \textbf{The ``dilution effect'':} This larger pool must now be shared among \textbf{102 people}. The new per capita wealth is $10,500 \text{ euros} / 102 \text{ people} \approx \textbf{102.94 euros}$.
        \item From an individual's point of view, their wealth only grew from 100 euros to 102.94 euros. Their \textbf{personal rate of return} was only $\approx \textbf{2.94\%}$, which is approximately $r - n$ (5\% - 2\%).
    \end{itemize}
    This shows why a ``direct'' approach is flawed from a behavioral standpoint: A rational person makes decisions based on the actual return they expect to receive, which is the diluted rate of $(r-n)$, not the headline market rate $r$.

    \textbf{Conclusion}

    To answer the question directly: We cannot consider debt in per capita scale ``directly'' because doing so would mean ignoring two critical facts:
    \begin{enumerate}
        \item \textbf{Macro Fact:} Total debt, the subject of the No-Ponzi rule, automatically grows with population.
        \item \textbf{Micro Fact:} Individuals experience a ``diluted'' rate of return, $(r-n)$, which guides their financial decisions.
    \end{enumerate}
    The correct method, which adjusts the interest rate to $(r-n)$ for per capita analysis, elegantly solves both issues. It correctly translates the macro-level aggregate debt rule into a micro-level behavioral framework that is consistent with an individual's experience in a growing economy.
\end{remark*}

\subsection{Household Problem}

The optimization problem in \textbf{per capita} terms:
\begin{align*}
    \max_{\{c_t,a_t\}_{t=0}^{\infty}} & \left\{\int_{0}^{\infty } u(c_t)e^{-(\rho-n)t} \, dt \right\}  \\
    s.t. \qquad                       & c_t \in \mathbb{R}_+, a_t \in \mathbb{R}                       \\
                                      & c_t + \dot{a}_t + na_t \leq w_t + r_ta_t                       \\
                                      & a_0 \text{ is given}                                           \\
                                      & \lim_{t \to \infty} e^{-\int_{0}^{t} (r_s-n) \, ds} a_t \geq 0
\end{align*}
This is a \textbf{deterministic constrained optimization problem in continuous time}, with one state and one control variables. We can use optimal control theory to solve it.

\subsection{FOC}

Using optimal control theory, we write the \underline{present value} Hamiltonian \[
    \mathcal{H}(c_t, a_t,v_t,t) = u(c_t)e^{-(\rho-n)t}+v_t[w_t+(r_t-n)a_t - c_t]
\], where $v_t$ is the shadow value, in terms of utility at time $0$, of a unit of income at time $t$.

The necessary conditions for an optimum are: \begin{align*}
     & \mathcal{H}_c = 0          &  & \Longrightarrow &  & v_t = u_c(c_t)e^{-(\rho-n)t}     \\
     & \mathcal{H}_a = -\dot{v}_t &  & \Longrightarrow &  & \dot{v}_t = -(r_t - n)v_t        \\
     & \mathcal{H}_v = \dot{a}_t  &  & \Longrightarrow &  & \dot{a}_t=w_t + (r_t-n)a_t - c_t
\end{align*}

Plus the transversality condition: \[
    \lim_{t \to \infty} (v_ta_t) = 0
\]

\subsection{The Euler Equation} Combining $\dot{v}_t = -(r_t - n)v_t$ and $v_t = u_c(c_t)e^{-(\rho-n)t} \iff \ln v_t = \ln u_c(c_t) - (\rho - n)t \Longrightarrow \frac{\dot{v}_t}{v_t} = c_t \frac{u_{cc}(c_t)}{u_c(c_t)}\frac{\dot{c}_t}{c_t} - (\rho - n)$, we get the Euler equation: \[
    \frac{\dot{c}_t}{c_t} = \frac{1}{\theta}(r_t - \rho)
\]
where $\theta \equiv -\frac{u_{cc}(c_t)c_t}{u_c(c_t)} > 0$ is the Arrow-Pratt coefficient of relative risk aversion (elasticity of the marginal utility).

\noindent\textbf{Implications}:

\underline{\textit{Sign}}: Optimal household consumption grows (falls) over time if the return on savings $r_t$ exceeds (fall short of) the subjective discount rate $\rho$. \[
    \gamma_{r} > 0 \iff r_t > \rho
\]

\underline{\textit{Size}}: The quantitative importance of the difference $(r_t-\rho)$ is inversely proportional to the curvature of the utility function, as measured by $\theta_t$. \[
    \frac{\partial \dot{c}_t/c_t}{\partial (r_t - \rho)} = \frac{1}{\theta_t}
\]\begin{itemize}
    \item The higher the curvature $\theta_t$, the lower the effect of $(r_t-\rho)$ is inversely proportional.
    \item $\theta_t$ captures the desire for consumption smoothing.
    \item $\frac{1}{\theta_t}$ is the \textbf{Intertemporal Elasticity of Substitution (IES)}.
\end{itemize}


\begin{remark*}[\textbf{\textit{Elasticity: Why the elasticity of the \underline{marginal} utility is more important than the elasticity of the utility?}}]
    \[
        \theta \equiv -\frac{u_{cc}(c_t)c_t}{u_c(c_t)} = -\frac{d\ln u_c(c_t)}{d\ln c_t}
    \]


    The elasticity of marginal utility is more important because it is a \textbf{fundamental and robust} parameter describing consumer behavior, whereas the elasticity of utility is an \textbf{arbitrary and fragile} measure that changes with behaviorally-irrelevant transformations of the utility function. This fragility is revealed by the economic principle of \emph{ordinal utility}.

    \noindent\textbf{The Core Argument: The Ordinal Nature of Utility}

    Modern economic theory is built on the concept of \emph{ordinal utility}. This principle states that a utility function is simply a tool to rank preferences; what matters is that $u(\text{option A}) > u(\text{option B})$, not the specific numerical values of the utility. A key implication is that any positive monotonic transformation of a utility function, such as adding a constant, will not change the consumer's behavior in any way.

    Let's consider a consumer with a utility function $u(c)$. Now, let's create a new utility function $\hat{u}(c) = u(c) + k$, where $k$ is any arbitrary constant (e.g., one million). A consumer with utility $u(c)$ and one with $\hat{u}(c)$ are behaviorally identical. They will make the exact same consumption-saving decisions. Any parameter that is truly important for describing their behavior must therefore be the same for both functions.

    \noindent\textbf{Testing Both Elasticities against the Ordinality Principle}

    \noindent\textit{1. The Robustness of the Elasticity of Marginal Utility ($\theta$)}

    First, we calculate the first and second derivatives for the transformed utility function, $\hat{u}(c)$:
    $$ \hat{u}'(c) = \frac{d}{dc}(u(c) + k) = u'(c) $$
    $$ \hat{u}''(c) = \frac{d}{dc}(u'(c)) = u''(c) $$
    Since the first and second derivatives are unchanged, the elasticity of marginal utility for the new function, $\hat{\theta}$, is identical to the original:
    $$ \hat{\theta} = - \frac{\hat{u}''(c) c}{\hat{u}'(c)} = - \frac{u''(c) c}{u'(c)} = \theta $$
    \textbf{This parameter is invariant.} It is a fundamental property of the preference structure that is not affected by behaviorally-irrelevant modifications to the utility function.

    \noindent\textit{2. The Fragility of the Elasticity of Utility ($E_u$)}

    Now, we calculate the elasticity of total utility for the transformed function, $\hat{E}_u$:
    $$ \hat{E}_u = \frac{\hat{u}'(c) c}{\hat{u}(c)} = \frac{u'(c) c}{u(c) + k} $$
    \textbf{This parameter is not invariant.} Its value depends directly on the arbitrary constant $k$. If $k$ is a very large positive number, $\hat{E}_u$ will approach zero, regardless of the consumer's true underlying preferences. A parameter that can be manipulated to be any value without a corresponding change in behavior cannot be fundamental.

    \noindent\textbf{Conclusion}

    A parameter cannot be considered fundamental to behavior if its value can be arbitrarily changed by a transformation that has no effect on that behavior. The elasticity of total utility fails this test; its value is not unique to a given preference structure and is therefore a meaningless metric for choice problems.

    The elasticity of marginal utility ($\theta$), however, passes the test. It is a robust measure of the curvature of preferences, which governs the core economic trade-off of intertemporal substitution. It is for this reason that it is the more important—and indeed, the only relevant—elasticity in the context of the Euler equation.

\end{remark*}


Alternatively, write \[
    r_t = \theta_t \frac{\dot{c}_t}{c_t} + \rho
\]

In a growing economy ($\frac{\dot{c}_t}{c_t} > 0$), consumption tomorrow is less valuable for two reasons: \begin{enumerate}
    \item Discount of the future: $\rho > 0$
    \item Decreasing marginal utility: If $\frac{\dot{c}_t}{c_t} >0$, then the consumption level of tomorrow is higher but with lower marginal utility $u_c$. How much lower depends on the curvature $\theta_t$.
\end{enumerate}

Households face the trade-off between consumption and saving, and the Euler equation says that the interest rate must be just enough to compensate for these two effects: \[
    \underbrace{r_t}_{\text{return of saving}} = \underbrace{\theta_t \frac{\dot{c}_t}{c_t}}_{\text{compensate the decreasing marginal utility}} + \underbrace{\rho}_{\text{compensate the future discount}}
\]
In equilibrium, households are indifferent at the margin between consuming and saving.

\begin{example*}
    [Constant $\theta_t$]
    The class of functions with this property is known as \textbf{the constant relative risk aversion (CRRA) function}, and takes the form: \[
        u(c) = \frac{c^{1-\theta} - 1}{1- \theta}
    \] (where $u_{cc} < 0 \iff \theta > 0$)

    In this case, the Euler Equation becomes: \[
        \frac{\dot{c}_t}{c_t} = \frac{1}{\theta}(r_t-\rho)
    \]
\end{example*}



\subsection{The Transversality Condition}

We have the TVC $\lim_{t \to \infty} (v_ta_t) = 0$.

\noindent What does the TVC impose?

From SOC: $\frac{\dot{v}_t}{v_t} = -(r_t - n) \Longrightarrow v_t = v_0e^{(-\bar{r}_t - n)t}$, substituting into the TVC:
\[
    \lim_{t \to \infty} v_0e^{-(\bar{r}_t - n)t}a_t = 0
\]

Since $v_0 = u_c(c_0) > 0$, we have \[
    \lim_{t \to \infty} e^{-(\bar{r}_t - n)t}a_t = \lim_{t \to \infty} \Big[a_t e^{-\int_{0}^{t} (t_{\tau}-n) \, d\tau}\Big] = 0
\]

\textbf{The NPG condition  must hold with \textit{\underline{\uppercase{equality}}}.}

\subsection{The Consumption Function (CRRA case)}

The Euler equation characterizes the \textbf{derivative} of optimal consumption, not its \textbf{level}.
\[
    \frac{\dot{c}_t}{c_t} = \frac{1}{\theta}(r_t-\rho) \Longrightarrow c_t = c_0 e^{\frac{1}{\theta}(\bar{r}_t - \rho)t}
\] where the constant $c_0$ determining the level is not pinned down by the Euler equation.

To obtain the level of the consumption function we need to enrol the budget constraint.
\[
    {\int_{0}^{\infty } e^{-(\bar{r}_t-n)t}c_t \, dt} = \underbrace{a_0 + \int_{0}^{\infty } e^{-(\bar{r}_t-n)t}w_t \, dt}_{\equiv \omega_0}
\]

\begin{remark*}[How to get the ${\int_{0}^{\infty } e^{-(\bar{r}_t-n)t}c_t \, dt} = {a_0 + \int_{0}^{\infty } e^{-(\bar{r}_t-n)t}w_t \, dt}$?]
    Same steps in NPG section we have \begin{align*}
        \dot{a}_t - (r_t -n) a_t                                                               & = w_t - c_t                                                                                                                            \\
        e^{-\int_{0}^{t} (r_s - n) \, ds} (\dot{a}_t - (r_t -n) a_t)                           & = w_t  e^{-\int_{0}^{t} (r_s - n) \, ds} - c_t e^{-\int_{0}^{t} (r_s - n) \, ds}                                                       \\
        \int_{0}^{\infty } e^{-\int_{0}^{t} (r_s - n) \, ds} (\dot{a}_t - (r_t -n) a_t)  \, dt & = \int_{0}^{\infty }  w_t  e^{-\int_{0}^{t} (r_s - n) \, ds}  \, dt - \int_{0}^{\infty } c_t e^{-\int_{0}^{t} (r_s - n) \, ds}  \, dt  \\
        e^{-\int_{0}^{t} (r_s - n) \, ds}a_t \Big|_0^\infty                                    & =  \int_{0}^{\infty }  w_t  e^{-\int_{0}^{t} (r_s - n) \, ds}  \, dt - \int_{0}^{\infty } c_t e^{-\int_{0}^{t} (r_s - n) \, ds}  \, dt \\
        \lim_{t \to \infty} e^{-\int_{0}^{t} (r_s - n) \, ds}a_t -  a_0                        & = \int_{0}^{\infty }  w_t  e^{-\int_{0}^{t} (r_s - n) \, ds}  \, dt - \int_{0}^{\infty } c_t e^{-\int_{0}^{t} (r_s - n) \, ds}  \, dt
    \end{align*}
    and we know from TVC condition \[
        \lim_{t \to \infty} e^{-(\bar{r}_t - n)t}a_t = \lim_{t \to \infty} \Big[a_t e^{-\int_{0}^{t} (t_{\tau}-n) \, d\tau}\Big] = 0
    \]
    Then we have \[
        \underbrace{\int_{0}^{\infty } e^{-(\bar{r}_t-n)t}c_t \, dt}_{\text{PDV of lifetime consumption}} = \underbrace{a_0 + \int_{0}^{\infty } e^{-(\bar{r}_t-n)t}w_t \, dt}_{\text{PDV of lifetime resources}}
    \]
\end{remark*}

Then substitute \underline{the optimal consumption path $c_t = c_0 e^{\frac{1}{\theta}(\bar{r}_t - \rho)t}$} into the intertemporal budget constraint: \[
    c_0 = \beta \omega_0
\] where $\beta = 1/\int_{0}^{\infty } e^{-(\bar{r}_t-n)t} e^{\frac{1}{\theta}(\bar{r}_t - \rho)t}\, dt$.

The initial consumption level $c_0$ is a $\beta$ fraction of the PDV of total time-$0$ resources. And it grows over time according to the rate $\frac{1}{\theta}(r_t-\rho)$ (following the Euler equation).

For the particular case in which $r_t = \rho, \forall t$ (like in the BGP), we have $\beta = \rho - n$.
\begin{itemize}
    \item Consumption per capita remains flat over time (Euler equation).
    \item Consumption per capita equals (per capita) permanent income: $c_0 = \underbrace{(\rho-n)\omega_0}_{\text{permanent income}}$
    \item Current income only matters through its effect on permanent income \[
              \frac{\partial c_0}{\partial w_t} = \frac{\partial c_0}{\partial \omega_0}\frac{\partial \omega_0}{\partial w_t} = (\rho - n)e^{-(\rho-n)t}
          \] Lottery example:  a rational household does not spend all the money in a short time, they will evaluate its effect on the lifetime wealth, and smooth the consumption.
\end{itemize}

This  simple result is known as the \textbf{\textit{Permanent Income Hypothesis}} \textit{(Friedman, 1957)}. \begin{itemize}
    \item It requires the absence of binding borrowing constraints ($a_t \in \mathbb{R}$).
    \item It leads to the \textbf{Ricardian equivalence}. \begin{itemize}
              \item Ricardian equivalence is the idea that under ideal conditions (lump-sum taxes, perfect foresight/credit access, unchanged spending) government debt is just deferred taxation, so a debt-financed tax cut leaves households’ lifetime resources—and thus consumption—unchanged.
          \end{itemize}
    \item Very much used in consumption theory.
\end{itemize}

In consumption theory, \textbf{permanent income} is defined as the constant income flow whose discounted present valuable is equal to the discounted present value of the actual income flow of an individual / household (including initial assets).

Hence, permanent income at time $0$, $\tilde{w}_0$ is implicitly defined as
\[
    \underbrace{\int_{0}^{\infty } e^{-(\bar{r}_t - n)t}\tilde{w}_0 \, dt}_{\text{PDV of a constant income flow}} = \underbrace{a_0 + \int_{0}^{\infty } e^{-(\bar{r}_t - n)t}w_t \, dt \equiv \omega_0}_{\text{PDV of the actual income flow}} \Longrightarrow \tilde{w}_0 \int_{0}^{\infty } e^{-(\bar{r}_t-n)t} \, dt = \omega_0
\]

And when $r_t = r \Rightarrow \tilde{w}_0 = (r-n)\omega_0$.

