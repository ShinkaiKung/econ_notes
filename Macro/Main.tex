\documentclass[11pt]{amsart}

% --- Packages ---
\usepackage[a4paper,margin=1in]{geometry} % Better for screen reading
\usepackage{amsmath, amssymb, amsthm}    % AMS math packages
\usepackage{hyperref}                    % Clickable references
\usepackage{lipsum}                      % For dummy text (remove later)
\usepackage{tikz}
\usepackage{caption}
\usetikzlibrary{calc,patterns}
\usepackage{setspace}
\usepackage[utf8]{inputenc}
\usepackage[T1]{fontenc}
\usepackage{lmodern}
\usepackage{tcolorbox}
\usepackage{multirow}
\usepackage{booktabs}       % For professional table rules (\toprule, \midrule, \bottomrule)
\usepackage{array}          % For advanced table column specifications


\tcbuselibrary{theorems, breakable, skins, listings}


% --- Theorem Environments ---
\newtheorem{theorem}{Theorem}[section]
\newtheorem{proposition}[theorem]{Proposition}
\newtheorem{lemma}[theorem]{Lemma}
\newtheorem{corollary}[theorem]{Corollary}
\theoremstyle{definition}
\newtheorem{definition}[theorem]{Definition}
\newtheorem{example}[theorem]{Example}

\theoremstyle{remark}
\newtheorem{remark}[theorem]{Remark}

% Unnumbered versions
\newtheorem*{theorem*}{Theorem}
\newtheorem*{proposition*}{Proposition}
\newtheorem*{lemma*}{Lemma}
\newtheorem*{corollary*}{Corollary}
\newtheorem*{definition*}{Definition}
\newtheorem*{example*}{Example}
\newtheorem*{remark*}{Remark}
\newtheorem*{claim}{Claim}


% --- Lecture counter ---
\newcounter{lecture}
\renewcommand{\thelecture}{\arabic{lecture}}
\makeatletter
\renewcommand{\theHsection}{\thelecture.\arabic{section}}
\renewcommand{\theHsubsection}{\thelecture.\arabic{section}.\arabic{subsection}}
\renewcommand{\theHsubsubsection}{\thelecture.\arabic{section}.\arabic{subsection}.\arabic{subsubsection}}
\makeatother
% Make sections depend on lecture
\renewcommand{\thesection}{\thelecture.\arabic{section}}
\newcommand{\lecture}[1]{%
  \refstepcounter{lecture}%
  \bigskip\hrule\bigskip
  {\Large\bfseries Lecture \thelecture: #1 \par}\medskip
  \addcontentsline{toc}{chapter}{Lecture \thelecture: #1}%
  \setcounter{section}{0} % reset sections for each lecture
}

\theoremstyle{remark}
\newtheorem{remarkinner}{Remark}      % 创建带编号的内部 remark
\newtheorem*{remarkinner*}{Remark}     % 创建不带编号的内部 remark*

% 3. 定义一个新的 tcolorbox 样式用于包装
% 这个盒子是可跨页的 (breakable)
\newtcolorbox{remarkbox}{
    enhanced,            % 启用增强功能
    breakable,           % 核心功能: 允许盒子跨页
    colback=white,       % 内部背景颜色 (白色)
    colframe=black,      % 边框颜色 (黑色)
    boxrule=0.5pt,       % 边框线宽
    arc=2mm,             % 圆角半径
    left=4mm,            % 内部左边距
    right=4mm,           % 内部右边距
    top=3mm,             % 内部上边距
    bottom=3mm,          % 内部下边距
    pad at break=2mm,    % 跨页断开处的间距
    % 在盒子断开处上方添加一条线
    % underlay unbroken and first={%
    %   \draw[black,line width=0.5pt] (frame.north west) -- (frame.north east);
    % },
    % % 在盒子断开处下方添加一条线
    % underlay unbroken and last={%
    %   \draw[black,line width=0.5pt] (frame.south west) -- (frame.south east);
    % }
}

% 4. 使用新定义的盒子重新定义 remark 和 remark* 环境
% 将原始的 "remarkinner" 环境包裹在 "remarkbox" 中
\renewenvironment{remark}
  {\begin{remarkbox}\begin{remarkinner}}
  {\end{remarkinner}\end{remarkbox}}

\renewenvironment{remark*}
  {\begin{remarkbox}\begin{remarkinner*}}
  {\end{remarkinner*}\end{remarkbox}}



\theoremstyle{example}
\newtheorem{exampleinner}{Example}      % 创建带编号的内部 example
\newtheorem*{exampleinner*}{Example}     % 创建不带编号的内部 example*

% 3. 定义一个新的 tcolorbox 样式用于包装
% 这个盒子是可跨页的 (breakable)
\newtcolorbox{examplebox}{
    enhanced,            % 启用增强功能
    breakable,           % 核心功能: 允许盒子跨页
    colback=white,       % 内部背景颜色 (白色)
    colframe=black,      % 边框颜色 (黑色)
    boxrule=0.5pt,       % 边框线宽
    arc=2mm,             % 圆角半径
    left=4mm,            % 内部左边距
    right=4mm,           % 内部右边距
    top=3mm,             % 内部上边距
    bottom=3mm,          % 内部下边距
    pad at break=2mm,    % 跨页断开处的间距
    % 在盒子断开处上方添加一条线
    % underlay unbroken and first={%
    %   \draw[black,line width=0.5pt] (frame.north west) -- (frame.north east);
    % },
    % % 在盒子断开处下方添加一条线
    % underlay unbroken and last={%
    %   \draw[black,line width=0.5pt] (frame.south west) -- (frame.south east);
    % }
}

% 4. 使用新定义的盒子重新定义 example 和 example* 环境
% 将原始的 "exampleinner" 环境包裹在 "examplebox" 中
\renewenvironment{example}
  {\begin{examplebox}\begin{exampleinner}}
  {\end{exampleinner}\end{examplebox}}

\renewenvironment{example*}
  {\begin{examplebox}\begin{exampleinner*}}
  {\end{exampleinner*}\end{examplebox}}


% --- Metadata ---
\title{Macroeconomics}
\author{Zian Gong}
\date{\today}

\begin{document}

\maketitle
\tableofcontents

\setlength{\parskip}{0.8em}   % 段落之间距离
\setlength{\parindent}{2em}   % 段首缩进
% % 1.5 倍行距
\onehalfspacing

% % 双倍行距
% \doublespacing

% 或者手动指定,比如 1.2 倍
% \setstretch{1.2}



\newpage
\lecture{Dynamic Optimization}
\section{Optimal Control Theory}

\subsection{The Canonical Problem}

\begin{align}
    \max_{x,y} W(x,y) & \equiv \int_{0}^{T} f(t,x_t,y_t) \, dt                       \\
    s.t. \, \dot{x}_t & = g(t,x_t,y_t)                                               \\
    y_t               & \in \mathcal{Y}, x_t \in \mathcal{X}, \, \forall t \in [0,T] \\
    x_0               & = \underbar{x}
\end{align}
Where: \begin{itemize}
    \item $x_t \in \mathcal{X}$ is the \textbf{state variable} (variable pre-determined at time $t$).
    \item $y_t \in \mathcal{Y}$ is the \textbf{control variable} (variable chosen at time $t$).
    \item $f: \mathbb{R}_+ \times \mathcal{X} \times \mathcal{Y} \to \mathbb{R}$ is the \textbf{instantaneous payoff function}.
    \item $g: \mathbb{R}_+ \times \times \mathcal{X} \times \mathcal{Y} \to \mathcal{X}$ is the \textbf{law of motion} of the state variable.
    \item $\underbar{x} \in \mathcal{X}$ is the \textbf{initial condition} for the state (which we take as given).
    \item $x: [0,T] \to \mathcal{X}$ and $y: [0,T] \to \mathcal{Y}$ are the \textbf{time paths} of the state and control variables.
    \item $W$ is a functional describing the \textbf{objective function}.
\end{itemize}

\textbf{Definition}: A pair $(x,y)$ is called an \textbf{admissible pair} if constraints (1)-(4) hold.

\textbf{Assumption}: $W(x,y) < \infty $ for all admissible pairs.


A note on \textbf{terminal conditions}: \begin{itemize}
    \item In problem (1)-(4) terminal time $T$ is given but the terminal state $x_T$ is free to choose.
    \item In some applications, both a terminal time $T$ and a terminal state $x_T = \bar{x} \in \mathcal{X}$ are given.
    \item In others, $x_T$ is given but $T$ is not, so the planning horizon $T$ is chosen optimally.
    \item Finally, it may be that both $T$ and $x_T$ are free (but related by a constraint).
\end{itemize}

\subsubsection{The Hamiltonian Method}

\[
    \mathcal{H}(t,x_t,y_t,\lambda_t) \equiv f(t,x_t,y_t) + \lambda_t g(t,x_t,y_t)
\]

FOC (Necessary Conditions): \begin{align}
     & \frac{\partial }{\partial y_t}\mathcal{H}(t,x_t^*, y_t^*, \lambda_t) = 0                \\
     & \frac{\partial }{\partial x_t}\mathcal{H}(t,x_t^*, y_t^*, \lambda_t) = -\dot{\lambda}_t \\
     & \frac{\partial }{\partial \lambda_t}\mathcal{H}(t,x_t^*, y_t^*,\lambda_t) = \dot{x}_t   \\
     & \lim_{t \to \infty} \lambda_tx_t^* = 0
\end{align}

\subsection{The Canonical Problem with Exponential Discounting}

\begin{align}
    \max_{x,y} W(x,y) & \equiv \int_{0}^{\infty } e^{-\rho t} f(x_t,y_t) \, dt              \\
    s.t. \, \dot{x}_t & = g(t,x_t,y_t)                                                      \\
    y_t               & \in \mathcal{Y}, x_t \in \mathcal{X}, \, \forall t \in \mathbb{R}_+ \\
    x_0               & = \underbar{x} \text{ and } \lim_{t \to \infty} b_tx_t \geq \bar{x}
\end{align}

\subsubsection{The Hamiltonian Method}

\begin{align*}
    \mathcal{H}(t,x_t,y_t,\lambda_t) & = \overbrace{e^{-\rho t}f(x_t,y_t) + \lambda_t g(t,x_t,y_t)}^{present-value Hamiltonian}                                                           \\
                                     & = e^{-\rho t}\Big[\underbrace{f(x_t,y_t) + \mu_t g(t,x_t,y_t)}_{current-value Hamiltonian}\Big] \, \text{where } \mu_t \equiv e^{\rho t} \lambda_t
\end{align*}
Current-value Hamiltonian:
\[
    \hat{\mathcal{H}}(t,x_t^*, y_t^*, \mu_t) \equiv f(x_t,y_t) + \mu_t g(t,x_t,y_t)
\]

FOC (Necessary Conditions): \begin{align}
     & \frac{\partial }{\partial y_t}\hat{\mathcal{H}}(t,x_t^*, y_t^*, \mu_t) = 0                        \\
     & \frac{\partial }{\partial x_t}\hat{\mathcal{H}}(t,x_t^*, y_t^*, \mu_t) = \rho \mu_t - \dot{\mu}_t \\
     & \frac{\partial }{\partial \mu_t}\hat{\mathcal{H}}(t,x_t^*, y_t^*,\mu_t) = \dot{x}_t               \\
     & \lim_{t \to \infty} e^{-\rho t}\mu_t x_t^* = 0
\end{align}

\section{Dynamic Programming}

\subsection{The Canonical Problem}
\begin{align}
    V_0(x_0)          & \equiv \max_{x,y} \int_{0}^{\infty} f(t,x_t,y_t) \, dt              \\
    s.t. \, \dot{x}_t & = g(t,x_t,y_t),                                                     \\
    y_t               & \in \mathcal{Y}, x_t \in \mathcal{X}, \, \forall t \in \mathbb{R}_+ \\
    x_0               & \text{ given and } \lim_{t \to \infty} b_tx_t \geq \bar{x}
\end{align}
Where $V_0(x_0)$ is the \textbf{value function}. It states the \textbf{optimal value} of the problem starting at time $t = 0$ with initial state $x_0$. More generally we can define $V_\tau(x_\tau)$ as the \textbf{optimal value} at time $0$ of the problem starting at time $\tau$ with some state $x_\tau$.

\begin{theorem}
    [\textbf{Bellman's Principle of Optimality}]
    Suppose we have an interior solution such that $(x_t^*, y_t^*)$ reaches the maximum value $V_0(x_0)$. Then, we have \[
        V_0(x_0) = \int_{0}^{t} f(s,x_s^*,y_s^*)  \, ds + V_t(x_t^*), \, \forall t \geq 0
    \] where $V_t(x_t) \equiv \int_{t}^{\infty } f(s,x_s^*, y_s^*) \, ds$.
\end{theorem}

\begin{definition}[\textbf{Hamilton-Jacobi-Bellmam (HJB) Equation}] By differentiating the equation in ``Bellman's Principle of Optimality'', we have the Hamilton-Jacobi-Bellmam (HJB) Equation characterizing the optimal pair $(x^*,y^*)$
    \[
        f(t,x_t^*,y_t^*) + \frac{\partial V_t(x_t^*)}{\partial t} + \frac{\partial V_t(x_t^*)}{\partial x_t}g(t,x_t^*,y_t^*) = 0, \, \forall t \geq 0
    \]
\end{definition}


\subsection{The Stationary Problem}

\begin{itemize}
    \item Exponential discounting, i.e. $f(t,x_t,y_t) = e^{-\rho t}f(x_t,y_t)$.
    \item Time-autonomous constraint, i.e. $\dot{x}_t = g(x_t,y_t)$ instead of $\dot{x}_t = g(t,x_t,y_t)$.
\end{itemize}
The solution of this problem is \textbf{time-consistent}.
\begin{align}
    V_0(x_0)          & \equiv \max_{x,y} \int_{0}^{\infty} e^{-\rho t} f(x_t,y_t) \, dt    \\
    s.t. \, \dot{x}_t & = g(x_t,y_t),                                                       \\
    y_t               & \in \mathcal{Y}, x_t \in \mathcal{X}, \, \forall t \in \mathbb{R}_+ \\
    x_0               & \text{ given and } \lim_{t \to \infty} b_tx_t \geq \bar{x}
\end{align}


\begin{definition}
    [\textbf{Time Consistency}]
    A solution $(x^*,y^*)$ of a dynamic optimization problem starting at time $0$ is time-consistent if $\forall t > 0$ the sub-sequence $(x_s^*,y_s^*: s \geq t) \subset (x^*,y^*)$ is the solution at time $t$ of the sub-problem starting at time $t$ with state $x_t^*$.
\end{definition}

In time consistent settings, the planner does not gain anything from revisiting the plan at any future date. It allows us to solve the problem \textbf{just once}, at time $t=0$.

\begin{remark*}[Bellmam to HJB]
    \begin{align*}
        V_0(x_0)                                 & = \int_{0}^{t} e^{-\rho s} f(x_s^*,y_s^*) \, ds + \int_{t}^{\infty } e^{-\rho s} f(x_s^*,y_s^*) \, ds                                \\
        \Longrightarrow^{\tau = s - t}  V_0(x_0) & = \int_{0}^{t} e^{-\rho s} f(x_s^*,y_s^*) \, ds + e^{-\rho t}\int_{0}^{\infty } e^{-\rho \tau} f(x_{\tau+t}^*,y_{\tau+t}^*) \, d\tau \\
        \Longrightarrow  V_0(x_0)                & = \int_{0}^{t} e^{-\rho s} f(x_s^*,y_s^*) \, ds + e^{-\rho t}V_0(x_t^*)
    \end{align*}
    Let $v(x) \equiv V_0(x)$, the HJB equation becomes \[
        f(x_t^*,y_t^*) = \rho v(x_t^*) - v_x(x_t^*)\dot{x}_t^*
    \]
\end{remark*}

\begin{remark*}[HJB and Hamiltonian]
    Rewrite HJB as:
    \[
        \rho v(x_t) = \max_{y_t} \{\underbrace{f(x_t,y_t) + v_x(x_t)g(x_t,y_t)}_{\hat{\mathcal{H}}(x_t,y_t,v_x(x_t))}\}
    \]

    The FOC \[
        f_y(x_t,y_t) + v_x(x_t)g_t(x_t,y_t) = 0
    \] corresponds to $\frac{\partial }{\partial y_t}\hat{\mathcal{H}}(x_t,y_t,v_x(x_t)) = 0$.

    And the \textbf{Envelope Condition} characterizes $v_x(x_t)$ \[
        \rho v_x(x_t) = f_x(x_t,y_t) + v_{xx}(x_t)g(x_t,y_t) + v_x(x_t)g_x(x_t,y_t)
    \] corresponds to $\rho \mu_t - \dot{\mu}_t = \frac{\partial }{\partial x_t}\hat{\mathcal{H}}(x_t,y_t,\mu_t)$, when $\mu_t(v_x(x_t))$.
\end{remark*}

\subsection{Stochastic Dynamic Optimization: Poisson point process}

\subsubsection{Poisson Point Process} A Poisson process is a specific type if \textbf{counting process}.
\begin{definition}
    A \textbf{counting process} is a stochastic process $(N_t:t \in \mathbb{R}_+)$ that records the number of events (or "arrivals") that occurred within some interval $(0,t]$, and that satisfies the following properties: \begin{itemize}
        \item $N_t \in \mathbb{Z}_+ \equiv \mathbb{N}\cup\{0\}$, i.e. $N_t$ must be a non-negative integer number.
        \item $\forall t \geq s, N_t \geq N_s$, i.e. events do not ``disappear''.
        \item $\forall t > s, N_t-N_s$ denotes the \textbf{number of arrivals} within the interval $(s,t]$.
    \end{itemize}
    \begin{example*} $N_t$ may record the number of \begin{itemize}
            \item Job offers received by an unemployed worker.
            \item Innovations by a firm doing R\&D.
            \item Buses that pass through a given bus stop.
            \item Number of births in a hospital.
        \end{itemize}
    \end{example*}
\end{definition}


\begin{definition}
    A counting process $N_t: t \in \mathbb{R}_+$ is \textbf{Poisson process} with ``arrival rate'' $\alpha > 0$ if: \begin{itemize}
        \item $N_0 = 0$ (i.e. at the beginning of time, no events have yet been recorded).
        \item $N_t$ has independent and stationary disjoint increments, that is: \begin{itemize}
                  \item Independent increments:\\The random variables $(N_{t_1} - N_{t_0}), (N_{t_2} - N_{t_1}), \dots, (N_{t_n} - N_{t_{n-1}})$ are independent.
                  \item Stationary increments:\\The distribution of $(N_{t+s}-N_t)$ is a function of $s$ but not of $t$.
              \end{itemize}
        \item The number of events $m \in \mathbb{Z}_+$ within a time interval $\Delta$ follows a \textbf{Poisson distribution}: \[
                  Pr[N_{t+\Delta} - N_t = m] = \frac{(\alpha\Delta)^m e^{-\alpha\Delta}}{m!}, \forall \Delta \geq 0
              \] which has mean $E[N_{t+\Delta} - N_t] = \alpha \Delta$.\\
              (Hence, the average number of events increases with the length $\Delta$ of the time interval and with the parameter $\alpha$ defining the rate of arrival of events)
    \end{itemize}
\end{definition}

Probabilities and Rates: \begin{itemize}
    \item Probabilities. Let $\Delta$ be some arbitrary small time interval. Then \begin{align*}
               & Pr[N_\Delta = 0] = e^{-\alpha \Delta} = 1-\alpha\Delta + o(\Delta) \leftarrow \text{ Prob. of \textbf{zero} events within } t \in [0,\Delta]          \\
               & Pr[N_\Delta = 1] =\alpha\Delta e^{-\alpha \Delta} = \alpha\Delta + o(\Delta) \leftarrow \text{ Prob. of \textbf{one} events within } t \in [0,\Delta] \\
               & Pr[N_\Delta = 2] =\cdots \cdots = o(\Delta) \leftarrow \text{ Prob. of \textbf{two} events within } t \in [0,\Delta]
          \end{align*}
    \item Rates. We can also express these probabilities in rates per unit of time \begin{align*}
               & Pr[N_\Delta = 0]/\Delta = o(\Delta)/\Delta + 1/\Delta - \alpha \\
               & Pr[N_\Delta = 1]/\Delta = o(\Delta)/\Delta + \alpha            \\
               & Pr[N_\Delta = 2]/\Delta = o(\Delta)/\Delta
          \end{align*}
    \item Hence, taking the \textbf{continuous time} limit ($\Delta \to 0$) we have \begin{itemize}
              \item At any instant of time we can have either $0$ or $1$ event, but no more.
              \item $\alpha$ is the ``Poisson arrival rate'' at which events occur ($\lim_{\Delta \to 0} \frac{Pr[N_\Delta = 1]}{\Delta} = \alpha$).
              \item $\alpha$ is NOT a probability, indeed it can be that $\alpha > 1$.
          \end{itemize}
\end{itemize}

Some other facts about Poisson process: \begin{itemize}
    \item The process is memoryless: \begin{itemize}
              \item The occurrence of one event does not affect the probability that a second event can occur. \\ (Earthquakes are not Poisson, as one big earthquake typically generates several replicas afterward)
              \item The length of time without events does not predict the occurrence of the next one \\ (So buses arriving at a bus stop are probably not Poisson)
          \end{itemize}
    \item The distribution for the length of time periods between arrivals is exponential: \begin{itemize}
              \item Let $\{t_n\}_{n=1}^\infty$ be the sequence of interarrival times.\\(times elapsed between consecutive arrivals, i.e., points in time at which a worker gets a job offer)
              \item Then, $\{t_n\}_{n=1}^\infty$ are  identically distributed exponential random variables with mean $1/\alpha$: \[
                        Pr[t_n \leq t \Big| \{t_j\}_{j=1}^{n-1}] = 1- e^{-\alpha t}, \, \forall n \in \mathbb{Z}_+, \, \forall t > t_{n-1}
                    \]
          \end{itemize}
\end{itemize}

\subsubsection{Dynamic Programming with a Poisson Point Process}

Consider a standard non-stochastic consumption/savings problem: \[
    \max_{(c_s,\alpha_s)} \int_{0}^{\infty } e^{-\rho s} u(c_s) \, ds, \qquad s.t. \, \dot{a}_s = w + ra_s -c_s
\] with $w$ and $a_0$ are given, plus some transversality condition on $a_s$.

Using the maximum principle we have \[
    v(a_0) \equiv \max_{c_s: \sin [0,t)} \{\int_{0}^{t} e^{-\rho s}u(c_s) \, ds + e^{-\rho t} v(a_t)\}
\]

Now we add a stochastic shock (Poisson): \begin{itemize}
    \item Let's assume that the consumer may go bankrupt at Poisson arrival rate $\delta > 0$.
    \item When this happens, the consumer’s wealth drops to zero.
\end{itemize}
The HJB equation of this stochastic problem \begin{align*}
    v(a_0) = \max_{c_s: s \in [0,t)} \{\int_{0}^{t} e^{-\rho s}u(c_s) \, ds
    + e^{-\rho t} \Big[
    \underbrace{e^{-\delta t}}_{\substack{\text{Prob. of not}              \\ \text{going bankrupt}}}v(a_t)
    + \underbrace{\delta t e^{-\delta t}}_{\substack{\text{Prob. of going} \\ \text{bankrupt only once}}}v(0) \\
    + \underbrace{Pr[N_t \geq 2]}_{\substack{\text{Prob. of going}         \\ \text{bankrupt twice or more}}}v(0)\Big]\}
\end{align*}

Next, compute a \underline{Taylor expansion} with respect to $t$ around $t = 0$, $\Delta$ is a very small interval near $t=0$ \begin{itemize}
    \item The first term becomes: $\int_{0}^{t} e^{-\rho s} u(c_s) \, ds = u(c_0)\Delta + o(\Delta)$ \begin{itemize}
              \item $c_s \simeq c_0$ and $e^{-\rho s} = 1 - \rho s + o(s^2)$
          \end{itemize}
    \item The second: $e^{-(\rho + \delta)t}v(a_t) = v(a_0) + v_a(a_0)\dot{a}_0\Delta - (\rho+\delta)v(a_0)\Delta + o(\Delta)$ \begin{itemize}
              \item $e^{-(\rho + \delta)t} = 1 - (\rho + \delta)t + o(t)$ and $a_t-a_0 = \dot{a}_0t + o(t)$, and $v(a_t) = v(a_0) - v_a(a_0)(a_t-a_0) + o(a_t - a_0) = v(a_0) + v_a(a_0)\dot{a}_0t+o(t)$.
          \end{itemize}
    \item The third: $e^{-(\rho+\delta)t}\delta t v(0) = \delta v(0)\Delta$ \begin{itemize}
              \item $e^{-(\rho + \delta)t} = 1 - (\rho + \delta)t + o(t)$
          \end{itemize}
    \item And the forth: $Pr[N_t \geq 2]v(0) = o(\Delta)$
\end{itemize}

So we can write \[
    v(a_0) = \max_{c_0} \{u(c_0)\Delta + v(a_0) + v_a(a_0)\dot{a}_0\Delta - (\rho + \delta)v(a_0)\Delta + \delta v(0)\Delta + o(\Delta)\}
\]

Divide both sides by $\Delta$, take the limit as $\Delta \to 0$, and write it more generally at time $t$: \[
    \rho v(a_t) = \max_{c_t > 0}\{u(c_t) + \frac{\partial v(a_t)}{\partial a_t}\dot{a}_t + \delta\underbrace{(v(0)-v(a_t))}_{\substack{\text{change in value due} \\ \text{to bankruptcy shock}}}\}
\]

A particular form of this \textbf{HJB equation} is when the problem is such that $v(0) = 0$: \[
    (\rho + \delta)v(a_t) = \max_{c_t > 0}\{u(c_t) + \frac{\partial v(a_t)}{\partial a_t} \dot{a}_t\}
\](the effective discount is given by the time discount ρplus the probability disocunt δ, as at the stochastic rate δthe value
function goes to zero)
\newpage
\lecture{Introduction of Neo-Classical Growth Model}

\begin{definition}[Commodity Bundles and Constraints]
    Budget Set = consumption set + economic constrains \[
        \beta(p,w) = \{x \in X: px \leq w\}
    \] where:
    \begin{itemize}
        \item $p = (p_1, p_2,\dots,p_L) \in \mathbb{R}^{L}$ is a price vector,
        \item $w \in \mathbb{R}^{+}$ is wealth,
        \item Consumption Plans: $x = (x_1, x_2, \dots, x_L) \in \mathbb{R}^{L}$,  $\left\{\begin{array}{l}
                      x_l > 0: \text{receive}   \\
                      x_l < 0: \text{give away} \\
                  \end{array}\right.$
        \item Consumption Set: $X \in \mathbb{R}^{L}$, in general we assume $X = \mathbb{R}^{L}_{+}\equiv\{x\in \mathbb{R}^{L}: x \geq 0\}$
        \item  Commodity Space $\mathbb{R}^{L}$: finite number of goods, perfect divisibility.
    \end{itemize}

\end{definition}
\begin{remark*}
    Implicit Assumptions: \begin{itemize}
        \item Price-taking consumers
        \item Perfect Information
        \item Complete Markets
        \item Exogenous Wealth
        \item Linear Budget Constraint
    \end{itemize}

    Normally, budget set is \textbf{compact} and \textbf{convex}, and $p \gg 0$.
    A counterexample: quantity discount.
\end{remark*}

\section{Preference}
\begin{definition}[Preference]
    \textbf{Preference} is represented by $\succeq$, a binary relationship on $X$: \[
        (X, \succeq) \subset X \times X: x^{1} \succeq x^{2} \iff(x^{1}, x^{2}) \in (X, \succeq)
    \] which means $x^{1}$ is at least as good as $x^{2}$.
    \begin{remark*}
        $X$ represents the consumption set, $X \times X$ is the \underline{Cartesian Product}, which represents all the possible combo pairings of commodities. $(x^{1}, x^{2})$ represents an \underline{Ordered Pair}.
    \end{remark*}

    \textbf{Strict Preference} $\succ$: \[
        x^{1} \succ x^{2} \iff x^{1} \succeq x^{2} \wedge \neg(x^{2} \succeq x^{1})
    \]

    \textbf{Indifference} $\sim$: \[
        x^{1} \sim x^{2} \iff x^{1} \succeq x^{2} \wedge x^{2} \succeq x^{1}
    \]
\end{definition}

Properties of \textbf{rational preferences}: \begin{enumerate}
    \item Transitivity: $\forall x^{1}, x^{2}, x^{3} \in X$: $ x^{1} \succeq x^{2} \wedge  x^{2} \succeq x^{3} \Longrightarrow  x^{1} \succeq x^{3}$.
    \item Completeness: $\forall x^{1}, x^{2} \in X \Longrightarrow x^{1} \succeq x^{2} \vee x^{2} \succeq x^{1}$.
\end{enumerate}
Other preference properties:
\begin{enumerate}
    \setcounter{enumi}{2}
    \item Continuity: $\forall x^{0} \in X$, the sets $\{x \in X: x \succeq x^{0}\}$ and $\{x \in X: x^{0} \succeq x\}$ are closed. \begin{itemize}
              \item Upper Contour Set and Lower Contour Set are closed.
              \item $\succeq$ is continuous $\iff \forall x^{0} \in X, $ an arbitrary convergent sequence $(x^{n})^{\infty }_{n=1}$, \[
                        \Big[\forall n, x^{n} \succeq x^{0}\Big] \Longrightarrow \Big[\lim_{n \to \infty} x^{n} = x \succeq x^{0}\Big]
                    \] \[
                        \Big[\forall n, x^{0} \succeq x^{n}\Big] \Longrightarrow \Big[\lim_{n \to \infty} x^{n} = x,\, x^{0} \succeq x\Big]
                    \]
                    A counterexample is the \underline{Lexicographic Order} in $\mathbb{R}^{2}_{+}$: \[
                        x \succeq x^{0} \iff (x_1 > x_1^0) \wedge (x_1=x_1^0 \wedge x_2 \geq x_2^0)
                    \], e.g., $x^{0}=(5,2), x^{n} = (5+1/n,0)$ for any $n$, we have $5+1/n > 5 \Longrightarrow x^{n} \succeq x^{0}$, however, $\lim_{n \to \infty} x^{n} = (5,0) \nsucceq x^{0}$.
              \item Closed Set: A set $A$ is closed iff all convergent sequence in $A$ has their limits in $A$.
          \end{itemize}
    \item Local nonsatiation: $\forall x^{0} \in X, \forall \varepsilon > 0, \exists x^{1} \in X \Longrightarrow ||x^{0}-x^{1}||<\varepsilon \wedge x^{1} \succ x^{0}$ \begin{itemize}
              \item It means we can always find a direction to improve.
          \end{itemize}
    \item[(4')] Monotonicity: $\forall x^{0}, x^{1}: x^{1} \gg x^{0} \Longrightarrow x^{1} \succ x^{0}$. (All goods more than initial set)
    \item[(4'')] Strong Monotonicity: $\forall x^{0}, x^{1}, x^{1} \neq x^{0}: x^{1} \geq x^{0} \Longrightarrow x^{1} \succ x^{0}$. (Some goods more than initial set) \begin{itemize}
              \item We can prove (4'') $\Longrightarrow$ (4') $\Longrightarrow$ (4).
          \end{itemize}
    \item Convexity: $\forall x \in X, \{x \in X: x \succeq x^0\}$ is convex. \begin{itemize}
              \item The upper contour set is convex.
              \item Convexity also gives the information about how the MRS changes while increasing one good.
              \item $(3)+(4)+(5) \Longrightarrow$ a nice indifference curve: \begin{enumerate}
                        \item Continuity gives us a solid, unbroken line.
                        \item Monotonicity makes that line slope downwards from left to right.
                        \item Convexity bends that downward-sloping line so it's bowed in toward the origin.
                    \end{enumerate}
          \end{itemize}
    \item[(5')] Strict Convexity: $\forall \lambda \in (0,1)\, \forall x^0,x^1,x^2 \in X: x^1 \neq x^2, x^1 \succeq x^0 \Longrightarrow \lambda x^1 + (1-\lambda)x^2 \succ x^0$. \begin{itemize}
              \item It's just a strict version of $(5)$.
          \end{itemize}
\end{enumerate}

\section{Utility Function}

\begin{definition}
    Given a rational preference $\succeq $ on $X$. A \textbf{utility function} $u: X \longrightarrow \mathbb{R}$ that represents $\succeq $ is such that \[
        \forall x^0,x^1 \in X, x^0 \succeq x^1 \iff u(x^0) \geq u(x^0).
    \]
\end{definition}

\begin{proposition}
    \[
        \left\{\begin{array}{l}
            x^0 \succ x^1 \iff u(x^0) > u(x^1) \\
            x^0 \sim x^1 \iff u(x^0) = u(x^1)  \\
        \end{array}\right.
    \]
\end{proposition}

\begin{proposition}[ordinality]
    Given $u: X \longrightarrow \mathbb{R}$ representing $\succeq $ and $f: \mathbb{R} \longrightarrow \mathbb{R}$ any strict increasing function, then $v = f \circ u$ also represents $\succeq $.
\end{proposition}

\begin{proof}
    $\forall x^0,x^1 \in X, x^0 \succeq x^1$.

    $u$ represents $succeq$: $u(x^0) \geq u(x^1)$

    $f$ is strictly increasing: $f(u(x^0)) \geq f(u(x^1)) \iff v(x^1) \geq v(x^0)$
\end{proof}

\begin{remark*}
    The "ordinality" implicitly means the scale of the utility function means nothing.
\end{remark*}

\begin{theorem}[Debreu's]
    Given a \textbf{rational preference} $\succeq$ on $X \subset \mathbb{R}^{L}_{+}$. If $\succeq $ is \textbf{continuous}, then there \textbf{exists} a \underline{continuous utility function} $u: X \longrightarrow \mathbb{R}$ that represents $\succeq $.
\end{theorem}

\begin{proof}
    To give a simple and intuitive proof, we assume $X = \mathbb{R}^{L}_{+}$, and that the preference $\succeq$ is also \textbf{monotone}.

    The proof proceeds by construction. We define a function $u(x)$ by mapping each bundle $x$ to a specific amount of a reference bundle, $e=(1,\dots,1)$. We then show this function is well-defined, represents the preferences, and is continuous.

    \textbf{Step 1:} The utility function $u(x)$ is well-defined

    For any bundle $x \in \mathbb{R}^L_+$, we must show there exists a \textbf{unique} scalar $\lambda_x \ge 0$ such that $\lambda_x e \sim x$.

    To do this, we define the set $F_x = \{\lambda e : \lambda \geq 0, \lambda e \preceq x \}$. This set contains all bundles on the main diagonal that are weakly inferior to $x$. We will show this set is compact (closed and bounded) and non-empty.

    \begin{itemize}
        \item[\textit{(1)}] \textit{\underline{Proof of non-empty:}} The zero bundle $\mathbf{0} = 0 \cdot e$. By monotonicity, $x \succeq \mathbf{0}$ for any $x \in \mathbb{R}^L_+$. Thus, $0 \cdot e \in F_x$, and $F_x$ is not empty.

        \item[\textit{(2)}] \textit{\underline{Proof of bounded:}} By monotonicity, for any $x \in \mathbb{R}^L_+$, we can find a bundle $z$ such that $z \gg x$ (e.g., $z_i = x_i + 1$ for all $i$), which implies $z \succ x$. Now, for a sufficiently large scalar $\bar{\lambda}$, we will have $\bar{\lambda} e \gg z$. Monotonicity then implies $\bar{\lambda} e \succ z$. By transitivity, $\bar{\lambda} e \succ x$. This means that any vector $\lambda e \in F_x$ (where $\lambda e \preceq x$) must have its components bounded by those of $\bar{\lambda}e$. Thus, the set of vectors $F_x$ is bounded.

        \item[\textit{(3)}] \textit{\underline{Proof of closed:}} The set $F_x$ is the intersection of the ray $R = \{\lambda e : \lambda \ge 0\}$ and the lower contour set $LCS_x = \{y \in \mathbb{R}^L_+ : y \preceq x\}$. The ray $R$ is a closed set. Because the preference relation $\succeq$ is \textbf{continuous}, the set $LCS_x$ is also closed. The intersection of two closed sets is closed, therefore $F_x$ is closed.
    \end{itemize}

    Since $F_x$ is a non-empty, closed, and bounded subset of $\mathbb{R}^L_+$, it is \textbf{compact}.

    Now, consider the function $f: F_x \to \mathbb{R}$ defined by $f(\lambda e) = \lambda$. This function is continuous. By the \textbf{Weierstrass Extreme Value Theorem}, a continuous function on a compact set attains its maximum. Let this maximum value be $\lambda^*$, achieved at the point $\lambda^* e \in F_x$.

    By construction, since $\lambda^* e \in F_x$, we know $\lambda^* e \preceq x$.

    We must also show $\lambda^* e \succeq x$. We prove this by contradiction. Assume $\lambda^* e \prec x$. This means $\lambda^* e$ is in the \textbf{strict lower contour set} of $x$, $SLCS_x = \{y \in X : y \prec x\}$. Since $\succeq$ is continuous, the set $SLCS_x$ is \textbf{open}.

    Because $\lambda^* e \in SLCS_x$ and $SLCS_x$ is open, there exists an $\varepsilon > 0$ such that the bundle $(\lambda^* + \varepsilon)e$ is also in $SLCS_x$. This means $(\lambda^* + \varepsilon)e \prec x$, which in turn implies $(\lambda^* + \varepsilon)e \preceq x$. By definition, this means $(\lambda^* + \varepsilon)e \in F_x$.

    However, this implies that the function $f(\lambda e) = \lambda$ attains the value $\lambda^* + \varepsilon$ on the set $F_x$. This contradicts $\lambda^*$ being the maximum value of $f$ on $F_x$.

    Therefore, the assumption $\lambda^* e \prec x$ must be false. Since we have both $\lambda^* e \preceq x$ and $\neg(\lambda^* e \prec x)$, we conclude that $\lambda^* e \sim x$. Uniqueness of $\lambda^*$ follows from strict monotonicity.

    We can now define the utility of $x$ as this unique scalar: $u(x) \doteqdot \lambda^*$.

    \textbf{Step 2:} $u(x)$ represents $\succeq$

    We must prove that for any $x, y \in X$, we have $x \succeq y \iff u(x) \geq u(y)$.

    ($\Longrightarrow$) Assume $x \succeq y$. By definition of our function, we have $x \sim u(x)e$ and $y \sim u(y)e$.
    By transitivity, the relations $x \succeq y$ and $x \sim u(x)e$ imply $u(x)e \succeq y$.
    Again by transitivity, $u(x)e \succeq y$ and $y \sim u(y)e$ imply $u(x)e \succeq u(y)e$.
    By monotonicity, for bundles on the main diagonal, $u(x)e \succeq u(y)e$ holds if and only if $u(x) \geq u(y)$.

    ($\Longleftarrow$) Assume $u(x) \geq u(y)$.
    By monotonicity, this implies $u(x)e \succeq u(y)e$.
    By definition, we have $x \sim u(x)e$ and $y \sim u(y)e$.
    Using transitivity on the entire relation: $x \sim u(x)e \succeq u(y)e \sim y$. This implies $x \succeq y$.

    \textbf{Step 3:} $u(x)$ is continuous

    \textbf{A real-valued function is continuous if and only if the inverse images of all closed intervals are closed sets.} ("A real-valued function is continuous if and only if the inverse images of all open intervals are open sets." is also correct.) This is equivalent to showing that for any scalar $\bar{\lambda}$, the sets $\{x \in X : u(x) \geq \bar{\lambda}\}$ and $\{x \in X : u(x) \leq \bar{\lambda}\}$ are both closed.

    (a) Consider the set $A = \{x \in X : u(x) \geq \bar{\lambda}\}$. From Step 2, the condition $u(x) \geq \bar{\lambda}$ is equivalent to $x \succeq \bar{\lambda}e$. Thus, $A = \{x \in X : x \succeq \bar{\lambda}e\}$. This is, by definition, the \textbf{upper contour set} of the bundle $\bar{\lambda}e$. By the initial assumption that the preference relation $\succeq$ is continuous, all its upper contour sets are closed. Therefore, $A$ is a closed set.

    (b) Consider the set $B = \{x \in X : u(x) \leq \bar{\lambda}\}$. Similarly, the condition $u(x) \leq \bar{\lambda}$ is equivalent to $x \preceq \bar{\lambda}e$. Thus, $B = \{x \in X : x \preceq \bar{\lambda}e\}$. This is the \textbf{lower contour set} of the bundle $\bar{\lambda}e$. Since $\succeq$ is continuous, all its lower contour sets are also closed. Therefore, $B$ is a closed set.

    Since both the upper and lower contour sets of the function $u(x)$ are closed for any value in its range, the function $u(x)$ is \textbf{continuous}.
\end{proof}

\begin{proposition}
    Given preference $\succeq $ with a utility representation $u$, we have that: \begin{align*}
        \succeq \text{continuous}          & \iff \text{there exists $u'$ continuous representing $\succeq $} \\
        \succeq \text{locally nonsatiated} & \iff \text{$u$ has no local maxima}                              \\
        \succeq \text{strongly monotone}   & \iff \text{$u$ is strictly increasing}                           \\
        \succeq \text{(strict) convex}     & \iff \text{$u$ is strict quasiconcave}                           \\
    \end{align*}
\end{proposition}


\section{Walrasian Demand and Indirect Utility}

\begin{definition}
    For program [P] \begin{align*}
         & \max_{x \in X} u(x)      \\
         & s.t. \, x \in \beta(p,w)
    \end{align*}

    \textbf{Indirect Utility}: $v(p,w) \equiv \max_{x \in \beta(p,w)} u(x)$.

    \textbf{Walrasian Demand}: $x(p,w) \equiv \arg\max_{x \in \beta(p,w)} u(x)$.

\end{definition}

\begin{remark*}
    $v(p,w)$ is also called value function, while $x(p,w)$ is a correspondence.

    Difference between "function" and "correspondence": \begin{itemize}
        \item Function: one input only maps to one output.
        \item Correspondence: one input maps to a set of outputs.
    \end{itemize}
\end{remark*}

\begin{proposition}
    If $u(\cdot )$ is continuous, then for any $p \gg 0$ and $w \geq 0$ the program [P] \textbf{has a solution}.

    If $u(\cdot )$ is also strictly quasiconcave, then the solution is \textbf{unique}. Moreover, $x(p,w)$ is a continuous function.
\end{proposition}

\begin{proof} Existence and Uniqueness.

    Existence: (Weierstrass Theorem): $u(\cdot )$ is continuous and $\beta(p,w)$ is compact $\Longrightarrow$ exists a maxima.

    Uniqueness: By contradiction: $\exists x^1, x^2 \in X: x^1, x^2 \in \beta(p,w) $ and $u(x) \leq u(x^1), u(x)\leq u(x^2)\, \forall x \in \beta(p,w)$. Then we have $u(\lambda x^1 + (1-\lambda)x^2) > \min\{u(x^1), u(x^2)\} \Longrightarrow u(x') > u(x^1)$. It contradicts to $u(x) \leq u(x^1)\, \forall x \in \beta(p,w)$.
\end{proof}
\newpage
\lecture{The Solow Model}
% TeX root = ../Main.tex
% First argument to \section is the title that will go in the table of contents. Second argument is the title that will be printed on the page.

\section{Basics of Linear Algebra}

\begin{definition}[Matrix]
    $A$ is an $m \times n$ \textbf{\textit{matrix}}
    \begin{equation*}
        A = (a_{ij}) = \begin{pmatrix}
            a_{11} & a_{12} & \dots  & a_{1n} \\
            a_{11} & a_{12} & \dots  & a_{1n} \\
            \vdots & \vdots & \ddots & \vdots \\
            a_{m1} & a_{m2} & \dots  & a_{mn} \\
        \end{pmatrix}
    \end{equation*}
    where the number $a_{ij}$ is called the $ij ^{th}$ \textbf{\textit{component}} or \textbf{\textit{entry}}.
\end{definition}

\subsection{Operations of Matrix}

Let $A = (a_{ij}), B=(b_{ij})$,

\textbf{Sum}: $A + B = (a_{ij} + b_{ij})$.

\textbf{Scalar Multiplication}: $\alpha A = (\alpha a_{ij})$.

\textbf{Dot Product/Inner Product} of two \textbf{vectors}: $a = (a_{1}, a_{2}, \dots, a_{n}), b = (b_{1}, b_{2}, \dots, b_{n})$ is defined as: \begin{equation*}
    a \cdot b = \sum_{i=1}^{n} a_{i}b_{i}
\end{equation*}

\textbf{Matrix Multiplication}: $(AB)_{ij} = \sum_{k=1}^{n} a_{ik}b_{kj}$.

Properties of Matrix Multiplication:
\begin{itemize}
    \item (AB)C = A(BC)
    \item A(B+C) = AB + AC
    \item (A+B)C = AC+BC
\end{itemize}

\subsection{Some Special Matrix}
\begin{itemize}
    \item \textbf{Square Matrix}
    \item \textbf{Zero Matrix}
    \item \textbf{Diagonal Matrix}
    \item \textbf{Upper Triangle Matrix}
    \item \textbf{Lower Triangle Matrix}
    \item \textbf{Identity Matrix}
\end{itemize}

\subsection{Transpose}

\begin{definition}[Transpose]
    The \textbf{transpose} of a matrix $A$ is denoted by $A^T$ or $A'$, is obtained by reversing its rows and columns.
\end{definition}
Properties of Transpose: $(A')' = A, (A+B)' = A' + B', (\alpha A)'=\alpha A', (AB)'=B'A'$

\subsubsection{Some Special Matrix Related to Transpose}
\begin{itemize}
    \item \textbf{Symmetric Matrix}: $A=A'$
    \item \textbf{Orthogonal Matrix}: $A'A=I=AA'$
    \item \textbf{Idempotent Matrix}: $AA=A$
\end{itemize}

\subsection{Determinants}
\begin{remark*}
    Only for \textbf{\textit{Square Matrix}}.
\end{remark*}
\begin{definition}
    Determinant is given by
    \begin{equation*}
        |A|=\det(A)=\sum_{j=1}^{n} a_{ij}A_{ij}
    \end{equation*}
    where $A_{ij}$ is the $ij ^{th}$ \textit{\textbf{cofactor}} of $A$, and $A_{ij} = (-1)^{i+j}M_{ij}$, $M_{ij}$ is the $ij ^{th}$ minor.
\end{definition}


\begin{remark*}
    The determinants represent the area/volume/... change after the Matrix Transform.
\end{remark*}

Properties of Determinants:
\begin{itemize}
    \item $|A'| = |A|$,
    \item $|AB| = |A||B|$,
    \item $|\alpha A| = \alpha ^{n}|A|$,
    \item $|A| = \prod_{i=1}^{n} a_{ii}$ if $A$ is triangular.
\end{itemize}

\subsection{Inverse}
\begin{remark*}
    Only for \textbf{\textit{Square Matrix}}.
\end{remark*}

\begin{definition}
    The \textbf{Inverse} $A ^{-1}$ of an $n \times n$ Square Matrix $A$ is the matrix $B$ that satisfies:
    \begin{equation*}
        AB=I_{n}, BA=I_{n}
    \end{equation*}
\end{definition}

How to calculate?
\begin{equation*}
    A^{-1} = \frac{1}{\det(A)}adj(A) \quad \text{where} \quad adj(A) = \begin{pmatrix}
        A_{11} & \dots  & A_{n1} \\
        \vdots & \vdots & \vdots \\
        A_{1n} & \dots  & A_{nn}
    \end{pmatrix}
\end{equation*}
where $A_{ij}$ is the $ij ^{th}$ \textit{\textbf{cofactor}} of $A$.

Properties of Inverse:
\begin{itemize}
    \item $AA^{-1} = A^{-1} A = I$
    \item $(A^{-1})^{-1} = A$
    \item $(AB)^{-1} = B^{-1}A^{-1}$
    \item $(A')^{-1} = (A^{-1})'$
    \item $|A^{-1}| = |A|^{-1}$
\end{itemize}

\subsection{Solve Linear Equations}

For the system of linear equations, we can denote as $Ax=b$, where $A$ is a matrix (not necessary to be Square Matrix) and $b$ is a column vector.

\subsubsection{Using Inverse Matrix of A}
If $A$ is invertible, then \begin{equation*}
    x = A^{-1}b \to x = \frac{1}{\det(A)}adj(A)b
\end{equation*}

\subsubsection{Cramer's Rule}

\begin{equation*}
    x_j = \frac{|A_{j}|}{|A|}, j=1,2,\dots,n
\end{equation*}
where $A_{j}$ is the matrix formed by replacing the $j^{th}$ column of $A$ with the vector $b$.

\begin{remark*}
    This method is important to solve a certain unknown variable of the system.
\end{remark*}

\subsubsection{Gaussian Elimination}

Using \textbf{\textit{Elementary Row Operations}} on $(A|b)$ to reduce the system to \textbf{\textit{row echelon form}}.

\subsection{Linear Independence}

\begin{definition}[Linearly Dependent]
    For vectors $a_{1}, a_{2}, \dots, a_{n}$, there exists a non-zero vector $c=(c_{1}, c_{2}, \dots, c_{n})$ such that $a_{1}c_{1} + a_{2}c_{2} + \dots + a_{n}c_{n} = 0$.
\end{definition}

\begin{definition}[Linearly Independent]
    For vectors $a_{1}, a_{2}, \dots, a_{n}$, only when vector $c=(c_{1}, c_{2}, \dots, c_{n}) = 0$ such that $a_{1}c_{1} + a_{2}c_{2} + \dots + a_{n}c_{n} = 0$.
\end{definition}

\subsection{Rank}

\begin{definition}
    The \textbf{\textit{column rank}} of $A$ is the number of the number of Linearly Independent column vectors of $A$.    The \textbf{\textit{row rank}} of $A$ is the number of the number of Linearly Independent row vectors of $A$.

    Always: $\text{column rank} = \text{row rank}$

    Denoted as $r(A)$.
\end{definition}

Properties of Rank:
\begin{itemize}
    \item $r(A) = r(A') = r(AA') = r(A'A)$,
    \item $r(AB) \leq min[r(A),r(B)]$,
    \item $r(AB)= r(A)$ if $B$ is a Square Matrix of full rank and $r(A+B) \leq r(A) + r(B)$.
\end{itemize}

\subsubsection{Rank and Solutions}

\begin{itemize}
    \item If $r(A|b)=r(A) = n$, there is one solution.
    \item If $r(A|b)=r(A) < n$, there are infinite solutions.
    \item If $r(A|b) \neq  r(A)$, there is no solution.
\end{itemize}

\section{Eigenvalue and Eigenvectors}

\begin{definition}
    Let $x$ be a non-zero vector and $\lambda$ is a scalar, \begin{equation*}
        Ax=\lambda x
    \end{equation*}
    , we call $\lambda$ is an \textbf{eigenvalue}(\textbf{characteristic value}) of $A$, and $x$ is an \textbf{eigenvector} of $A$.
\end{definition}

How to calculate?
\begin{enumerate}
    \item Calculate $\det (A-\lambda I) = 0$, then we have eigenvalues.
    \item Then solve the system of linear equations $(A-\lambda_i I)x_i = 0$.
\end{enumerate}

\section{Trace}

\begin{definition}
    The trace of a Square Matrix $A$ is given by,
    \begin{equation*}
        tr(A) = \sum_{i=1}^{n} a_{ii}
    \end{equation*}, the sum of diagonal numbers.
\end{definition}

Properties of trace:
\begin{itemize}
    \item $tr(cA) = c[tr(A)]$
    \item $tr(A') = tr(A)$
    \item $tr(A+B) = tr(A) + tr(B)$
    \item $tr(I_{n}) = n$
    \item \textbf{$x'x=tr(x'x)=tr(xx')$, if $x$ is a column vector}
    \item $tr(AB) = tr(BA)$, moreover, $tr(ABC) = tr(CAB) = tr(BCA)$
\end{itemize}

If the Square Matrix A has eigenvalues $\lambda_1, \lambda_2, \dots, \lambda_n$, then
\begin{itemize}
    \item $\det(A) = \prod_{i=1}^{n} \lambda_i$
    \item $tr(A) = \sum_{i=1}^{n} \lambda_i$
\end{itemize}

\section{Diagonalization}

\begin{definition}
    A matrix is \textbf{\textit{diagonalizable}} if it can be written as \begin{equation*}
        A = PDP ^{-1}
    \end{equation*}, where $D$ is a diagonal matrix, $P$ is, of course, a invertible matrix.
\end{definition}

\begin{remark*}
    If we rewrite the definition like this: \begin{equation*}
        AP = PD
    \end{equation*}, comparing to the eigen equation $Ax = \lambda x$, it's easy to know the columns of matrix $P$ represent eigenvectors of matrix $A$ and the diagonal numbers in matrix $D$ represent eigenvalues of matrix $A$.
    \begin{equation*}
        P ^{-1} A P = D = diag(\lambda_{1}, \lambda_{2}, \dots, \lambda_{n})
    \end{equation*}
\end{remark*}

\begin{remark*}
    A matrix is diagonalizable $\iff$ it has a set of linearly independent eigenvectors.
\end{remark*}

\section{Orthonormal}

\begin{definition}
    A matrix is \textbf{\textit{Orthonormal}} if \begin{equation*}
        P ^{-1} = P'
    \end{equation*}, and the column vectors are unit vectors, orthogonal to each others.
\end{definition}

\begin{remark*}
    For \textbf{vectors} $x, y$, $x$ and $y$ are orthogonal $\iff x'y = 0$
\end{remark*}

\begin{remark*}
    If matrix $A$ is \textbf{\textit{symmetric}}, then:
    \begin{itemize}
        \item All of its eigenvectors are real.
        \item Eigenvectors corresponding to distinct eigenvalues are orthogonal.
        \item If $A$ is diagonalizable, the eigenvectors' matrix can be written in an orthogonal form.
    \end{itemize}
\end{remark*}

\section{Quadratic Forms}

\begin{definition}
    A general quadratic form in $n$ variables is \begin{equation*}
        Q(x_{1},\dots,x_{n}) = \sum_{i=1}^{n}\sum_{j=1}^{n} a_{ij}x_ix_j = x'Ax
    \end{equation*}
    , the matrix $A$ can be written in a symmetric form.
\end{definition}

\subsection{Definiteness}

\begin{definition}[Positive Definite]
    $\forall x \neq 0, \,x'Ax > 0$
\end{definition}

\begin{definition}[Negative Definite]
    $\forall x \neq 0, \,x'Ax < 0$
\end{definition}

\begin{definition}[Positive Semi-definite]
    $\forall x \neq 0, \,x'Ax \geq 0$
\end{definition}

\begin{definition}[Positive Semi-definite]
    $\forall x \neq 0, \,x'Ax \leq 0$
\end{definition}

\subsubsection{Methods to Determine Definiteness}
\begin{itemize}
    \item Minors - \emph{See the Lecture 04}
    \item Eigenvalues: Let $A$ be symmetric, and $\lambda_{1}, \lambda_{2}, \dots, \lambda_{n}$ are eigenvalues,
          \begin{itemize}
              \item positive definite $\iff \lambda_{1} > 0, \lambda_{2}>0, \dots, \lambda_{n}>0$
              \item negative definite $\iff \lambda_{1} < 0, \lambda_{2}<0, \dots, \lambda_{n}<0$
              \item positive semi-definite $\iff \lambda_{1} \geq 0, \lambda_{2} \geq 0, \dots, \lambda_{n}\geq 0$
              \item negative semi-definite $\iff \lambda_{1} \leq 0, \lambda_{2}\leq 0, \dots, \lambda_{n}\leq 0$
          \end{itemize}
\end{itemize}

\begin{proposition}[Cholesky Decomposition]
    If $A$ is positive definite, then it can be decomposed as \begin{equation*}
        A = LL'
    \end{equation*}, where $L$ is a lower triangular matrix with strictly positive diagonal entries. $L$ is unique.
\end{proposition}

\section{Vector Space and Subspace}

\begin{definition}[Vector Space]
    $V$ is a \textbf{\textit{vector space}} if for all $u,v,w$ in $V$ and all scalars $r, s$ in $\mathbb{R}$ we have:
    \begin{enumerate}
        \item $(u+v) \in V$, (closure under addition)
        \item $u+v = v+u$, (commutative law for addition)
        \item $u+(v+w) = (u+v) + w$, (associative law for addition)
        \item $\exists \textbf{0} \in V: \forall v \in V, v + \textbf{0} = v$, (additive identity)
        \item $\forall v \in V, \exists w \in V: v+w=\textbf{0}$, (additive inverse)
        \item $rv \in V$, (closure under scalar multiplication)
        \item $r(u+v) = ru+rv$, (distributive under scalar multiplication)
        \item $(r+s)u = ru+su$
        \item $(rs)u = r(su)$
        \item $\textbf{1}u = u$
    \end{enumerate}
\end{definition}

\begin{definition}[Subspace in $\mathbb{R}^{n}$]
    Any \textbf{\textit{subset}} $V$ of $\mathbb{R}^{n}$ which satisfies Properties (1)-(10) is called a \textbf{\textit{subspace in $\mathbb{R}^{n}$}}.
\end{definition}

\begin{theorem}[How to check if a subset is a subspace of $\mathbb{R}^{n}$? - Two Conditions]
    Let $V$ be a \textbf{\textit{subset}} of $\mathbb{R}^{n}$. Assume that $\forall u,v \in V$ we have $(u+v) \in V$ and $\forall v \in V, \forall r \in \mathbb{R}$ we have $rv \in V$. Then $V$ is a subspace.
\end{theorem}

\begin{remark*}
    The definition of a vector space and a subspace in a vector space is not limited to $\mathbb{R}^{n}$; it also works for other sets. To determine a subset is a subspace of a known vector space (not necessarily to be $\mathbb{R}^{n}$), we just need closure under addition and closure under scalar multiplication.
\end{remark*}

\section{Span and Basis}

\begin{definition}
    Let $\{a_{1}, a_{2}, \dots, a_{k}\}$ be a collection of vectors in $\mathbb{R}^{n}$.

    The set $V = \{c_{1}a_{1}+c_{2}a_{2}+\dots+c_{k}a_{k}: x_{1}, \dots,x_{k} \in \mathbb{R}\}$ is called the "\textbf{span of $a_{1}, a_{2}, \dots, a_{k}$}".

    Denoted as: \begin{equation*}
        V=sp[a_{1}, a_{2}, \dots, a_{k}] \quad \text{or} \quad V = \mathcal{L}[a_{1}, a_{2}, \dots, a_{k}]
    \end{equation*}
\end{definition}

\begin{definition}
    Let $V$ be a subspace of $\mathbb{R}^{n}$. The set of vectors ${b_{1}, b_{2}, \dots, b_{k}}$ form a \textbf{basis} of $V$ if
    \begin{enumerate}
        \item $b_{1}, b_{2}, b_{k}$ are linearly independent, and
        \item $\forall v \in V, v = \sum_{i=1}^{k}c_{i}b_{i}$
    \end{enumerate}
\end{definition}

\begin{definition}
    The number of vectors in any basis of $V$ is called the \textbf{dimension} of $V$. Denoted as $\dim(A)$.
\end{definition}

\section{Row Space, Column Space and Rank} % TODO: Review it!

\begin{definition}
    Let $A$ be an $m \times n$ matrix. The \textbf{column space} and the \textbf{row space} are the span of the columns and the rows of $A$, respectively:
    \begin{equation*}
        Row(A)=\mathcal{L}[a'_{1\cdot},a'_{2\cdot},\dots,a'_{n\cdot}] \quad \text{and} \quad Col(A) = \mathcal{L}[a_{\cdot 1},a_{\cdot 2},\dots,a_{\cdot n}]
    \end{equation*}
\end{definition}

\begin{theorem}
    \begin{equation*}
        \dim[Col(A)] = \dim[Row(A)] = r(A)
    \end{equation*}
\end{theorem}


\section{Null-Space} % TODO: Review it!

\begin{theorem}
    Let $A$ be an $m \times n$ matrix. Then, set $V$ of \textbf{solutions} to the homogenous system $Ax=0$ is a subspace of $\mathbb{R}^{n}$.
\end{theorem}

\begin{remark*}
    It is called the \textbf{null-space} of $A$ or the \textbf{kernel} of $A$, and written as $Null(A)$ or $\ker(A)$. (\textbf{Explanation}: It is the set of all vectors that are mapped to \textbf{0} by the matrix $A$.)
\end{remark*}

\section{Solutions of a Linear System (Vector Space Perspective)}

\begin{theorem}
    Let $A$ be an $m \times n$ matrix of coefficients:
    \begin{itemize}
        \item $Ax=b$ has a solution for a given $b \in \mathbb{R}^{m} \iff b \in Col(A)$,
        \item  $Ax=b$ has a solution for every $b \in \mathbb{R}^{m} \iff r(A) = m$,
        \item If $Ax=b$ has a solution $\forall b \Rightarrow r(A) \leq \# cols(A)=n$.
    \end{itemize}
\end{theorem}


\begin{theorem}[Fundamental Theorem of Linear Algebra] Let $A$ be an $m \times n$ matrix of coefficients, then
    \begin{equation*}
        \dim[Null(A)] + r(A) = n
    \end{equation*}.
\end{theorem}
\newpage
\lecture{The Ramsey Model}
\section{Random Vectors}

Joint cdf - def. Consider \(k\) random variables.
\(X_{1},X_{2},\ldots ,X_{k}\)
\[
    F (x _ {1}, x _ {2}, \ldots , x _ {k}) \equiv p (X _ {1} \leq x _ {1}, X _ {2} \leq x _ {2}, \ldots , X _ {k} \leq x _ {k})
\]

\textbf{Discrete variables}: Joint pmf \(p(X_{1} = x_{1},\ldots ,X_{k} = x_{k})\)

\textbf{Continuous variables}: There exists a joint pdf
\(f(x_{1},\ldots ,x_{k})\) such that
\[
    F (x _ {1}, \ldots , x _ {k}) = \int_ {- \infty} ^ {x _ {1}} \dots \int_ {- \infty} ^ {x _ {k}} f (z _ {1}, \ldots , z _ {k}) d z _ {1} \dots d z _ {k}
\] \begin{enumerate}
    \item   \(f(x_{1},\ldots ,x_{k})\geq 0\)
    \item   \(\int_{-\infty}^{\infty}\dots \int_{-\infty}^{\infty}f\big(z_1,\ldots ,z_k\big)dz_1\dots dz_k = 1\)
    \item    \begin{enumerate}
              \item     \(p\big(a_1 < X_1 < b_1, \ldots, a_k < X_k < b_k\big) =\int_ {a _ {1}} ^ {b _ {1}} \dots \int_ {a _ {k}} ^ {b _ {k}} f (z _ {1}, \ldots , z _ {k}) d z _ {1} \dots d z _ {k}\)
              \item   \(p(X_{1} = a_{1},X_{2} = a_{2},\ldots ,X_{k} = a_{k}) = 0\)
              \item \(p(X_{1} = a_{1},a_{2} < X_{2} < b_{2},\ldots ,a_{k} < X_{k} < b_{k}) = 0\)
          \end{enumerate}
    \item     \(\frac{\partial^k}{\partial x_1 \cdots \partial x_k} F(\cdot ) = f(\cdot )\).
\end{enumerate}

\textbf{Marginal distributions:}

The marginal cdf of the r.v. \(X_{i}\) is
\[
    \begin{array}{l} F _ {i} (x) \equiv p (X _ {i} \leq x) = p (X _ {1} \leq \infty , \ldots , X _ {i} \leq x, \ldots , X _ {k} \leq \infty) = \\ = F (\infty , \dots , x, \dots , \infty) \\ \end{array}
\]

The marginal pdf of \(X_{i}\) is
\[
    \begin{array}{l} f _ {i} (x) = \int \dots \int f \left(z _ {1}, \dots , z _ {i - 1}, x, z _ {i + 1}, \dots , z _ {k}\right). \\ \cdot d z _ {1} \cdot \cdot \cdot d z _ {i - 1} d z _ {i + 1} \cdot \cdot \cdot d z _ {k} \\ \end{array}
\]


\section{Conditional Probability \& Independence}

\begin{definition}[Conditional probability]
    Consider two events \(A\) and \(B\) in a probability space with
    \(p(B)\neq 0\) ; then
    \[
        p \left(A \mid B\right) = \frac {p (A \cap B)}{p (B)}
    \]
\end{definition}

\textbf{Independent events}: Events \(A\) and \(B\) are independent if conditional and unconditional probabilities are the same:
\begin{enumerate}
    \item \(p(A \mid B) = p(A)\)
    \item \(p(B \mid A) = p(B)\)
    \item \(p(A \cap B) = p(A)p(B)\)
\end{enumerate}





% You can add more sections here
% \section{Advanced Topic}
% % TeX root = ../Main.tex
% First argument to \section is the title that will go in the table of contents. Second argument is the title that will be printed on the page.

\section{Basics of Linear Algebra}

\begin{definition}[Matrix]
    $A$ is an $m \times n$ \textbf{\textit{matrix}}
    \begin{equation*}
        A = (a_{ij}) = \begin{pmatrix}
            a_{11} & a_{12} & \dots  & a_{1n} \\
            a_{11} & a_{12} & \dots  & a_{1n} \\
            \vdots & \vdots & \ddots & \vdots \\
            a_{m1} & a_{m2} & \dots  & a_{mn} \\
        \end{pmatrix}
    \end{equation*}
    where the number $a_{ij}$ is called the $ij ^{th}$ \textbf{\textit{component}} or \textbf{\textit{entry}}.
\end{definition}

\subsection{Operations of Matrix}

Let $A = (a_{ij}), B=(b_{ij})$,

\textbf{Sum}: $A + B = (a_{ij} + b_{ij})$.

\textbf{Scalar Multiplication}: $\alpha A = (\alpha a_{ij})$.

\textbf{Dot Product/Inner Product} of two \textbf{vectors}: $a = (a_{1}, a_{2}, \dots, a_{n}), b = (b_{1}, b_{2}, \dots, b_{n})$ is defined as: \begin{equation*}
    a \cdot b = \sum_{i=1}^{n} a_{i}b_{i}
\end{equation*}

\textbf{Matrix Multiplication}: $(AB)_{ij} = \sum_{k=1}^{n} a_{ik}b_{kj}$.

Properties of Matrix Multiplication:
\begin{itemize}
    \item (AB)C = A(BC)
    \item A(B+C) = AB + AC
    \item (A+B)C = AC+BC
\end{itemize}

\subsection{Some Special Matrix}
\begin{itemize}
    \item \textbf{Square Matrix}
    \item \textbf{Zero Matrix}
    \item \textbf{Diagonal Matrix}
    \item \textbf{Upper Triangle Matrix}
    \item \textbf{Lower Triangle Matrix}
    \item \textbf{Identity Matrix}
\end{itemize}

\subsection{Transpose}

\begin{definition}[Transpose]
    The \textbf{transpose} of a matrix $A$ is denoted by $A^T$ or $A'$, is obtained by reversing its rows and columns.
\end{definition}
Properties of Transpose: $(A')' = A, (A+B)' = A' + B', (\alpha A)'=\alpha A', (AB)'=B'A'$

\subsubsection{Some Special Matrix Related to Transpose}
\begin{itemize}
    \item \textbf{Symmetric Matrix}: $A=A'$
    \item \textbf{Orthogonal Matrix}: $A'A=I=AA'$
    \item \textbf{Idempotent Matrix}: $AA=A$
\end{itemize}

\subsection{Determinants}
\begin{remark*}
    Only for \textbf{\textit{Square Matrix}}.
\end{remark*}
\begin{definition}
    Determinant is given by
    \begin{equation*}
        |A|=\det(A)=\sum_{j=1}^{n} a_{ij}A_{ij}
    \end{equation*}
    where $A_{ij}$ is the $ij ^{th}$ \textit{\textbf{cofactor}} of $A$, and $A_{ij} = (-1)^{i+j}M_{ij}$, $M_{ij}$ is the $ij ^{th}$ minor.
\end{definition}


\begin{remark*}
    The determinants represent the area/volume/... change after the Matrix Transform.
\end{remark*}

Properties of Determinants:
\begin{itemize}
    \item $|A'| = |A|$,
    \item $|AB| = |A||B|$,
    \item $|\alpha A| = \alpha ^{n}|A|$,
    \item $|A| = \prod_{i=1}^{n} a_{ii}$ if $A$ is triangular.
\end{itemize}

\subsection{Inverse}
\begin{remark*}
    Only for \textbf{\textit{Square Matrix}}.
\end{remark*}

\begin{definition}
    The \textbf{Inverse} $A ^{-1}$ of an $n \times n$ Square Matrix $A$ is the matrix $B$ that satisfies:
    \begin{equation*}
        AB=I_{n}, BA=I_{n}
    \end{equation*}
\end{definition}

How to calculate?
\begin{equation*}
    A^{-1} = \frac{1}{\det(A)}adj(A) \quad \text{where} \quad adj(A) = \begin{pmatrix}
        A_{11} & \dots  & A_{n1} \\
        \vdots & \vdots & \vdots \\
        A_{1n} & \dots  & A_{nn}
    \end{pmatrix}
\end{equation*}
where $A_{ij}$ is the $ij ^{th}$ \textit{\textbf{cofactor}} of $A$.

Properties of Inverse:
\begin{itemize}
    \item $AA^{-1} = A^{-1} A = I$
    \item $(A^{-1})^{-1} = A$
    \item $(AB)^{-1} = B^{-1}A^{-1}$
    \item $(A')^{-1} = (A^{-1})'$
    \item $|A^{-1}| = |A|^{-1}$
\end{itemize}

\subsection{Solve Linear Equations}

For the system of linear equations, we can denote as $Ax=b$, where $A$ is a matrix (not necessary to be Square Matrix) and $b$ is a column vector.

\subsubsection{Using Inverse Matrix of A}
If $A$ is invertible, then \begin{equation*}
    x = A^{-1}b \to x = \frac{1}{\det(A)}adj(A)b
\end{equation*}

\subsubsection{Cramer's Rule}

\begin{equation*}
    x_j = \frac{|A_{j}|}{|A|}, j=1,2,\dots,n
\end{equation*}
where $A_{j}$ is the matrix formed by replacing the $j^{th}$ column of $A$ with the vector $b$.

\begin{remark*}
    This method is important to solve a certain unknown variable of the system.
\end{remark*}

\subsubsection{Gaussian Elimination}

Using \textbf{\textit{Elementary Row Operations}} on $(A|b)$ to reduce the system to \textbf{\textit{row echelon form}}.

\subsection{Linear Independence}

\begin{definition}[Linearly Dependent]
    For vectors $a_{1}, a_{2}, \dots, a_{n}$, there exists a non-zero vector $c=(c_{1}, c_{2}, \dots, c_{n})$ such that $a_{1}c_{1} + a_{2}c_{2} + \dots + a_{n}c_{n} = 0$.
\end{definition}

\begin{definition}[Linearly Independent]
    For vectors $a_{1}, a_{2}, \dots, a_{n}$, only when vector $c=(c_{1}, c_{2}, \dots, c_{n}) = 0$ such that $a_{1}c_{1} + a_{2}c_{2} + \dots + a_{n}c_{n} = 0$.
\end{definition}

\subsection{Rank}

\begin{definition}
    The \textbf{\textit{column rank}} of $A$ is the number of the number of Linearly Independent column vectors of $A$.    The \textbf{\textit{row rank}} of $A$ is the number of the number of Linearly Independent row vectors of $A$.

    Always: $\text{column rank} = \text{row rank}$

    Denoted as $r(A)$.
\end{definition}

Properties of Rank:
\begin{itemize}
    \item $r(A) = r(A') = r(AA') = r(A'A)$,
    \item $r(AB) \leq min[r(A),r(B)]$,
    \item $r(AB)= r(A)$ if $B$ is a Square Matrix of full rank and $r(A+B) \leq r(A) + r(B)$.
\end{itemize}

\subsubsection{Rank and Solutions}

\begin{itemize}
    \item If $r(A|b)=r(A) = n$, there is one solution.
    \item If $r(A|b)=r(A) < n$, there are infinite solutions.
    \item If $r(A|b) \neq  r(A)$, there is no solution.
\end{itemize}

\section{Eigenvalue and Eigenvectors}

\begin{definition}
    Let $x$ be a non-zero vector and $\lambda$ is a scalar, \begin{equation*}
        Ax=\lambda x
    \end{equation*}
    , we call $\lambda$ is an \textbf{eigenvalue}(\textbf{characteristic value}) of $A$, and $x$ is an \textbf{eigenvector} of $A$.
\end{definition}

How to calculate?
\begin{enumerate}
    \item Calculate $\det (A-\lambda I) = 0$, then we have eigenvalues.
    \item Then solve the system of linear equations $(A-\lambda_i I)x_i = 0$.
\end{enumerate}

\section{Trace}

\begin{definition}
    The trace of a Square Matrix $A$ is given by,
    \begin{equation*}
        tr(A) = \sum_{i=1}^{n} a_{ii}
    \end{equation*}, the sum of diagonal numbers.
\end{definition}

Properties of trace:
\begin{itemize}
    \item $tr(cA) = c[tr(A)]$
    \item $tr(A') = tr(A)$
    \item $tr(A+B) = tr(A) + tr(B)$
    \item $tr(I_{n}) = n$
    \item \textbf{$x'x=tr(x'x)=tr(xx')$, if $x$ is a column vector}
    \item $tr(AB) = tr(BA)$, moreover, $tr(ABC) = tr(CAB) = tr(BCA)$
\end{itemize}

If the Square Matrix A has eigenvalues $\lambda_1, \lambda_2, \dots, \lambda_n$, then
\begin{itemize}
    \item $\det(A) = \prod_{i=1}^{n} \lambda_i$
    \item $tr(A) = \sum_{i=1}^{n} \lambda_i$
\end{itemize}

\section{Diagonalization}

\begin{definition}
    A matrix is \textbf{\textit{diagonalizable}} if it can be written as \begin{equation*}
        A = PDP ^{-1}
    \end{equation*}, where $D$ is a diagonal matrix, $P$ is, of course, a invertible matrix.
\end{definition}

\begin{remark*}
    If we rewrite the definition like this: \begin{equation*}
        AP = PD
    \end{equation*}, comparing to the eigen equation $Ax = \lambda x$, it's easy to know the columns of matrix $P$ represent eigenvectors of matrix $A$ and the diagonal numbers in matrix $D$ represent eigenvalues of matrix $A$.
    \begin{equation*}
        P ^{-1} A P = D = diag(\lambda_{1}, \lambda_{2}, \dots, \lambda_{n})
    \end{equation*}
\end{remark*}

\begin{remark*}
    A matrix is diagonalizable $\iff$ it has a set of linearly independent eigenvectors.
\end{remark*}

\section{Orthonormal}

\begin{definition}
    A matrix is \textbf{\textit{Orthonormal}} if \begin{equation*}
        P ^{-1} = P'
    \end{equation*}, and the column vectors are unit vectors, orthogonal to each others.
\end{definition}

\begin{remark*}
    For \textbf{vectors} $x, y$, $x$ and $y$ are orthogonal $\iff x'y = 0$
\end{remark*}

\begin{remark*}
    If matrix $A$ is \textbf{\textit{symmetric}}, then:
    \begin{itemize}
        \item All of its eigenvectors are real.
        \item Eigenvectors corresponding to distinct eigenvalues are orthogonal.
        \item If $A$ is diagonalizable, the eigenvectors' matrix can be written in an orthogonal form.
    \end{itemize}
\end{remark*}

\section{Quadratic Forms}

\begin{definition}
    A general quadratic form in $n$ variables is \begin{equation*}
        Q(x_{1},\dots,x_{n}) = \sum_{i=1}^{n}\sum_{j=1}^{n} a_{ij}x_ix_j = x'Ax
    \end{equation*}
    , the matrix $A$ can be written in a symmetric form.
\end{definition}

\subsection{Definiteness}

\begin{definition}[Positive Definite]
    $\forall x \neq 0, \,x'Ax > 0$
\end{definition}

\begin{definition}[Negative Definite]
    $\forall x \neq 0, \,x'Ax < 0$
\end{definition}

\begin{definition}[Positive Semi-definite]
    $\forall x \neq 0, \,x'Ax \geq 0$
\end{definition}

\begin{definition}[Positive Semi-definite]
    $\forall x \neq 0, \,x'Ax \leq 0$
\end{definition}

\subsubsection{Methods to Determine Definiteness}
\begin{itemize}
    \item Minors - \emph{See the Lecture 04}
    \item Eigenvalues: Let $A$ be symmetric, and $\lambda_{1}, \lambda_{2}, \dots, \lambda_{n}$ are eigenvalues,
          \begin{itemize}
              \item positive definite $\iff \lambda_{1} > 0, \lambda_{2}>0, \dots, \lambda_{n}>0$
              \item negative definite $\iff \lambda_{1} < 0, \lambda_{2}<0, \dots, \lambda_{n}<0$
              \item positive semi-definite $\iff \lambda_{1} \geq 0, \lambda_{2} \geq 0, \dots, \lambda_{n}\geq 0$
              \item negative semi-definite $\iff \lambda_{1} \leq 0, \lambda_{2}\leq 0, \dots, \lambda_{n}\leq 0$
          \end{itemize}
\end{itemize}

\begin{proposition}[Cholesky Decomposition]
    If $A$ is positive definite, then it can be decomposed as \begin{equation*}
        A = LL'
    \end{equation*}, where $L$ is a lower triangular matrix with strictly positive diagonal entries. $L$ is unique.
\end{proposition}

\section{Vector Space and Subspace}

\begin{definition}[Vector Space]
    $V$ is a \textbf{\textit{vector space}} if for all $u,v,w$ in $V$ and all scalars $r, s$ in $\mathbb{R}$ we have:
    \begin{enumerate}
        \item $(u+v) \in V$, (closure under addition)
        \item $u+v = v+u$, (commutative law for addition)
        \item $u+(v+w) = (u+v) + w$, (associative law for addition)
        \item $\exists \textbf{0} \in V: \forall v \in V, v + \textbf{0} = v$, (additive identity)
        \item $\forall v \in V, \exists w \in V: v+w=\textbf{0}$, (additive inverse)
        \item $rv \in V$, (closure under scalar multiplication)
        \item $r(u+v) = ru+rv$, (distributive under scalar multiplication)
        \item $(r+s)u = ru+su$
        \item $(rs)u = r(su)$
        \item $\textbf{1}u = u$
    \end{enumerate}
\end{definition}

\begin{definition}[Subspace in $\mathbb{R}^{n}$]
    Any \textbf{\textit{subset}} $V$ of $\mathbb{R}^{n}$ which satisfies Properties (1)-(10) is called a \textbf{\textit{subspace in $\mathbb{R}^{n}$}}.
\end{definition}

\begin{theorem}[How to check if a subset is a subspace of $\mathbb{R}^{n}$? - Two Conditions]
    Let $V$ be a \textbf{\textit{subset}} of $\mathbb{R}^{n}$. Assume that $\forall u,v \in V$ we have $(u+v) \in V$ and $\forall v \in V, \forall r \in \mathbb{R}$ we have $rv \in V$. Then $V$ is a subspace.
\end{theorem}

\begin{remark*}
    The definition of a vector space and a subspace in a vector space is not limited to $\mathbb{R}^{n}$; it also works for other sets. To determine a subset is a subspace of a known vector space (not necessarily to be $\mathbb{R}^{n}$), we just need closure under addition and closure under scalar multiplication.
\end{remark*}

\section{Span and Basis}

\begin{definition}
    Let $\{a_{1}, a_{2}, \dots, a_{k}\}$ be a collection of vectors in $\mathbb{R}^{n}$.

    The set $V = \{c_{1}a_{1}+c_{2}a_{2}+\dots+c_{k}a_{k}: x_{1}, \dots,x_{k} \in \mathbb{R}\}$ is called the "\textbf{span of $a_{1}, a_{2}, \dots, a_{k}$}".

    Denoted as: \begin{equation*}
        V=sp[a_{1}, a_{2}, \dots, a_{k}] \quad \text{or} \quad V = \mathcal{L}[a_{1}, a_{2}, \dots, a_{k}]
    \end{equation*}
\end{definition}

\begin{definition}
    Let $V$ be a subspace of $\mathbb{R}^{n}$. The set of vectors ${b_{1}, b_{2}, \dots, b_{k}}$ form a \textbf{basis} of $V$ if
    \begin{enumerate}
        \item $b_{1}, b_{2}, b_{k}$ are linearly independent, and
        \item $\forall v \in V, v = \sum_{i=1}^{k}c_{i}b_{i}$
    \end{enumerate}
\end{definition}

\begin{definition}
    The number of vectors in any basis of $V$ is called the \textbf{dimension} of $V$. Denoted as $\dim(A)$.
\end{definition}

\section{Row Space, Column Space and Rank} % TODO: Review it!

\begin{definition}
    Let $A$ be an $m \times n$ matrix. The \textbf{column space} and the \textbf{row space} are the span of the columns and the rows of $A$, respectively:
    \begin{equation*}
        Row(A)=\mathcal{L}[a'_{1\cdot},a'_{2\cdot},\dots,a'_{n\cdot}] \quad \text{and} \quad Col(A) = \mathcal{L}[a_{\cdot 1},a_{\cdot 2},\dots,a_{\cdot n}]
    \end{equation*}
\end{definition}

\begin{theorem}
    \begin{equation*}
        \dim[Col(A)] = \dim[Row(A)] = r(A)
    \end{equation*}
\end{theorem}


\section{Null-Space} % TODO: Review it!

\begin{theorem}
    Let $A$ be an $m \times n$ matrix. Then, set $V$ of \textbf{solutions} to the homogenous system $Ax=0$ is a subspace of $\mathbb{R}^{n}$.
\end{theorem}

\begin{remark*}
    It is called the \textbf{null-space} of $A$ or the \textbf{kernel} of $A$, and written as $Null(A)$ or $\ker(A)$. (\textbf{Explanation}: It is the set of all vectors that are mapped to \textbf{0} by the matrix $A$.)
\end{remark*}

\section{Solutions of a Linear System (Vector Space Perspective)}

\begin{theorem}
    Let $A$ be an $m \times n$ matrix of coefficients:
    \begin{itemize}
        \item $Ax=b$ has a solution for a given $b \in \mathbb{R}^{m} \iff b \in Col(A)$,
        \item  $Ax=b$ has a solution for every $b \in \mathbb{R}^{m} \iff r(A) = m$,
        \item If $Ax=b$ has a solution $\forall b \Rightarrow r(A) \leq \# cols(A)=n$.
    \end{itemize}
\end{theorem}


\begin{theorem}[Fundamental Theorem of Linear Algebra] Let $A$ be an $m \times n$ matrix of coefficients, then
    \begin{equation*}
        \dim[Null(A)] + r(A) = n
    \end{equation*}.
\end{theorem}

\end{document}
