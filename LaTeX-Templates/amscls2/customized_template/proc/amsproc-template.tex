%% filename: amsproc-template.tex
%% version: 1.1
%% date: 2014/07/24
%%
%% American Mathematical Society
%% Technical Support
%% Publications Technical Group
%% 201 Charles Street
%% Providence, RI 02904
%% USA
%% tel: (401) 455-4080
%%      (800) 321-4267 (USA and Canada only)
%% fax: (401) 331-3842
%% email: tech-support@ams.org
%% 
%% Copyright 2008-2010, 2014 American Mathematical Society.
%% 
%% This work may be distributed and/or modified under the
%% conditions of the LaTeX Project Public License, either version 1.3c
%% of this license or (at your option) any later version.
%% The latest version of this license is in
%%   http://www.latex-project.org/lppl.txt
%% and version 1.3c or later is part of all distributions of LaTeX
%% version 2005/12/01 or later.
%% 
%% This work has the LPPL maintenance status `maintained'.
%% 
%% The Current Maintainer of this work is the American Mathematical
%% Society.
%%
%% ====================================================================

%     AMS-LaTeX v.2 template for use with amsproc
%
%     Remove any commented or uncommented macros you do not use.

\documentclass{amsproc}

% \newtheorem{theorem}{Theorem}[section]
% \newtheorem{lemma}[theorem]{Lemma}

% \theoremstyle{definition}
% \newtheorem{definition}[theorem]{Definition}
% \newtheorem{example}[theorem]{Example}
% \newtheorem{xca}[theorem]{Exercise}

% \theoremstyle{remark}
% \newtheorem{remark}[theorem]{Remark}
% --- Theorem Environments ---
\newtheorem{theorem}{Theorem}[section]
\newtheorem{proposition}[theorem]{Proposition}
\newtheorem{lemma}[theorem]{Lemma}
\newtheorem{corollary}[theorem]{Corollary}

\theoremstyle{definition}
\newtheorem{definition}[theorem]{Definition}
\newtheorem{example}[theorem]{Example}

\theoremstyle{remark}
\newtheorem{remark}[theorem]{Remark}

% Unnumbered versions
\newtheorem*{theorem*}{Theorem}
\newtheorem*{proposition*}{Proposition}
\newtheorem*{lemma*}{Lemma}
\newtheorem*{corollary*}{Corollary}
\newtheorem*{definition*}{Definition}
\newtheorem*{example*}{Example}
\newtheorem*{remark*}{Remark}
\newtheorem*{claim}{Claim}


\numberwithin{equation}{section}

\begin{document}

\title{}

%    Remove any unused author tags.

%    author one information
\author{}
\address{}
\curraddr{}
\email{}
\thanks{}

%    author two information
\author{}
\address{}
\curraddr{}
\email{}
\thanks{}

\subjclass[2010]{Primary }

\keywords{}

\date{}

\dedicatory{}

\begin{abstract}
\end{abstract}

\maketitle

% TeX root = ../Main.tex
% First argument to \section is the title that will go in the table of contents. Second argument is the title that will be printed on the page.
\section[Univariate Analysis]{Univariate Analysis}

\subsection{Limits and Continuity}

\subsubsection{Limits}

\begin{definition}
    $\forall \epsilon > 0, \exists \eta >0$ such that if $|x - a| < \eta$, then $|f(x) - A| < \epsilon$.
    \begin{equation*}
        \lim_{x \to a} f(x) = A
    \end{equation*}
\end{definition}

\begin{remark*}
    Need to prove with definition.
\end{remark*}


\subsubsection{LHS\&RHS}

\begin{proposition}[$LHS=RHS$]
    One-sided limits:
    \begin{itemize}
        \item LHS limit: $\lim_{x \to a^-} f(x)=B_1$ if $f(x) \to B_1$ as $x \to a^-$.
        \item RHS limit: $\lim_{x \to a^+} f(x)=B_2$ if $f(x) \to B_2$ as $x \to a^+$.
    \end{itemize}

    A limit at $a$ exists $\iff \text{LHS limit} = \text{RHS limit} \iff B_1=B_2$.
\end{proposition}

Limits Rules: if $\lim_{x \to a}f(x)=A $ and $\lim_{x \to a}g(x)=B $, then:
\begin{itemize}
    \item $\lim_{x \to a}[f(x) \pm g(x)] = A \pm B $,
    \item $\lim_{x \to a} f(x)g(x) = AB $,
    \item $\lim_{x \to a}\frac{f(x)}{g(x)} = \frac{A}{B} (\text{if } B \neq 0) $,
    \item $\lim_{x \to a} [f(x)]^r = A^r(\text{if $A^r$ is defined and $r$ is a real number})$.
\end{itemize}

\subsubsection{Limits at Infinity}

\begin{definition}[$\lim_{x \to \pm\infty} f(x)$]Limits at Infinity:

    $\forall \epsilon > 0, \exists N > 0$ such that if $x > N$, then $|f(x) - L|< \epsilon$.
    \begin{equation*}
        \lim_{x \to \infty} f(x) = L
    \end{equation*}

    $\forall \epsilon > 0, \exists N > 0$ such that if $x < -N$, then $|f(x) - L|< \epsilon$.
    \begin{equation*}
        \lim_{x \to -\infty} f(x) = L
    \end{equation*}
\end{definition}

\subsubsection{Continuity}

\begin{definition}
    A function $f$ is continuous at a given point $x=a$ if $\lim_{x \to a} f(x) = f(a)$. $\iff \forall \epsilon > 0, \exists \delta > 0$ such that if $|x-a|<\delta$ then $|f(x) - f(a)| < \epsilon$.
\end{definition}

Properties: if $f$ and $g$ are continuous on $a$, then
\begin{itemize}
    \item $f \pm g$ is continuous on $a$,
    \item $fg$ is continuous on $a$,
    \item $f / g$ is continuous on $a$, if $g(a) \neq 0$,
    \item $f(x)^{r}$ is continuous on $a$ if $f(x)^{r}$ is defined and $r$ is a real number.
\end{itemize}


\begin{theorem}[Intermediate Value Theorem]
    If $f(x)$ is continuous on $[a,b]$, $\forall y \in [f(a), f(b)]$(assume $f(a) < f(b)$), then $\exists c \in [a,b]$ such that $y=f(c)$.
\end{theorem}

\begin{theorem}[Bolzano's Theorem]
    If $f(x)$ is continuous on $[a,b]$ and assume $f(a)f(b)<0$, then $\exists c \in [a,b]$ such that $f(c) = 0$.
    \begin{remark*}
        It's a special case of "Intermediate Value Theorem".
    \end{remark*}
\end{theorem}

\subsection{Differentiation}

\subsubsection{Differentiation}

\begin{definition}
    We say a function $f$ is \textit{\textbf{differentiable}} at $x$ if
    \begin{equation*}
        \frac{d f(x)}{dx} = \lim_{h \to 0}\frac{f(x+h)-f(x)}{h}
    \end{equation*}
    exists.
\end{definition}

\begin{definition}
    If function $f$ is differentiable for all $x$ in its domain, then we say $f$ is a \textbf{\textit{differentiable function}}.
\end{definition}

\begin{remark*}
    Function $f$ is differentiable at point A $\iff $
    \begin{itemize}
        \item It's continuous at point A.
        \item Its $LHS = RHS$.
    \end{itemize}
\end{remark*}

\subsubsection{Higher-order Derivatives}

Notation: The $k$th order differentiation is denoted $\frac{d ^{k}f(x)}{dx ^{k}}$ or $f ^{(k)}(x)$. The class of all $k-$times continuously differentiable functions is called $C^k$.

\subsubsection{Implicit Differentiation}

\begin{proposition}
    For $f(x, y(x)) = 0$, we have \begin{equation*}
        \frac{df}{dx} + \frac{df}{dy}\frac{dy}{dx} = 0
    \end{equation*}
\end{proposition}

\begin{example*}
    %TODO:
\end{example*}


\subsubsection{Inverse Function Theorem}

\begin{theorem}[Inverse Function Theorem]
    If $f$ is differentiable and strictly increasing or decreasing in an interval $I$, then $f$ has an inverse function $g$. If $x_0$ is an intetior point of $I$ and $f'(x_0) \neq 0$, then $g$ is differentiable at $y_0 = f(x_0)$ and: \begin{equation*}
        g'(x_0) = \frac{1}{f'(x_0)} = \frac{1}{f'[g(y_0)]}.
    \end{equation*}
\end{theorem}

\begin{example*}
    %TODO:
\end{example*}

\subsubsection{Taylor Expansions}

\textbf{Taylor expansions}:
\begin{align*}
    f(x) & \approx f(c) + f'(c)(x-c) + \frac{1}{2!}f''(c)(x-c)^{2} + \dots + \frac{1}{k!}f ^{(k)}(c)(x-c)^{k} \\
         & \approx f(c) + \sum_{i=1}^{k}\frac{f ^{(i)}(c)}{i!}(x-c)^{i}
\end{align*}

\begin{theorem}[Rolle's Theorem]
    Let $f: [a, b] \to \mathbb{R}$. If $f$ is continuous on $[a, b]$ and differentiable on (a, b). If $f(a)=f(b)$, then $\exists c \in (a,b)$ such that $f'(c) = 0$.
\end{theorem}

\begin{theorem}[Mean Value Theorem]
    Let $f: [a, b] \to \mathbb{R}$. If $f$ is continuous on $[a, b]$ and differentiable on (a, b), then $\exists z \in (a,b)$ such that:
    \begin{equation*}
        f'(z) = \frac{f(b)- f(a)}{b-a}.
    \end{equation*}
\end{theorem}

\begin{proposition}
    \begin{equation*}
        f(x) = f(c) + \sum_{i=1}^{k}\frac{f ^{(i)}(c)}{i!}(x-c)^{i} + \frac{f ^{(k+1)}(z)}{(k+1)!}(x-x_0)^{k+1}
    \end{equation*}
\end{proposition}


\begin{proof}[\textbf{Proof of Taylor expansions}]:

    The first-order: \begin{equation*}
        f(x) \approx f(x_0) + f'(x_0)(x-x_0)
    \end{equation*}

    The higher order approximation:
    \begin{align*}
        P(x) & = a_0 + a_{1}(x-x_0) + a_{2}(x-x_0)^{2} + a_{3}(x-x_0)^{3} + a_{n}(x-x_0)^{n} + U(x) = f(x) \\
        P(x) & = Q(x) + U(x) = f(x)
    \end{align*}

    We have:
    \begin{align*}
        a_0     & = f(x_0)       \\
        a_{1}   & = f'(x_0)      \\
        2!a_{2} & = f''(x_0)     \\
        n!a_{n} & = f^{(n)}(x_0)
    \end{align*}
    then $a_{0} = f(x_0), a_{1} = f'(x_0), a_{2}=\frac{f''(x_0)}{2!}, \dots, a_{n} = \frac{f^{(n)}(x_0)}{n!} \Rightarrow Q(x) = \sum_{i=1}^{n} \frac{f ^{(k)}(x_0)}{k!}(x-x_0)^{n}$

    For functions $U(x), R(x) = (x-x_0)^{n+1}$, they are high-order continuous and differentiable on interval $I$. And $U(x_0) = R(x_0) = 0, U'(x_1) = R'(x_1) = 0, U''(x_0) = R''(x_0) = 0, \dots, U^{(n)}(x_0) = R^{(n)}(x_0) = 0$, then using "\textit{\textbf{Cauchy Mean Value Theorem}}": \begin{align*}
         & \text{On domain $[x_0, x]$, there $\exists c_1 \in (x_0, x)$ such that}                                                                                                                                       \\
         & \frac{U(x) - U(x_0)}{R(x)- R(x_0)} = \frac{U'(c_1)}{R'(c_1)} \Rightarrow  \frac{U(x)}{R(x)} = \frac{U'(c_1)}{R'(c_1)}                                                                                         \\
         & \text{On domain $[x_0, c_1]$, there $\exists c_2 \in (x_0, c_1)$ such that}                                                                                                                                   \\
         & \frac{U'(c_1) - U'(x_0)}{R'(c_1)- R'(x_0)} = \frac{U''(c_2)}{R''(c_2)} \Rightarrow  \frac{U'(c_1)}{R'(c_1)} = \frac{U''(c_2)}{R''(c_2)}                                                                       \\
         & \dots                                                                                                                                                                                                         \\
         & \text{On domain $[x_0, c_n]$, there $\exists c_{n+1} \in (x_0, c_n)$ such that}                                                                                                                               \\
         & \frac{U^{(n)}(c_n) - U^{(n)}(x_0)}{R^{(n)}(c_n)- R^{(n)}(x_0)} = \frac{U^{(n+1)}(c_{n+1})}{R^{(n+1)}(c_{n+1})} \Rightarrow  \frac{U^{(n)}(c_n)}{R^{(n)}(c_n)} = \frac{U^{(n+1)}(c_{n+1})}{R^{(n+1)}(c_{n+1})} \\
         & \text{Then}                                                                                                                                                                                                   \\
         & \frac{U(x)}{R(x)} = \frac{U'(c_1)}{R'(c_1)} = \frac{U''(c_2)}{R''(c_2)} = \dots = \frac{U^{(n+1)}(c_{n+1})}{R^{(n+1)}(c_{n+1})}                                                                               \\
         & U(x) = \frac{U^{(n+1)}(c_{n+1})}{(n+1)!}(x-x_0)^{n+1}                                                                                                                                                         \\
         & \text{Since $f^{(n+1)}(x) = Q^{(n+1)}(x) + U^{(n+1)}(x) = U^{(n+1)}(x)$, then}                                                                                                                                \\
         & U(x) = \frac{f^{(n+1)}(c_{n+1})}{(n+1)!}(x-x_0)^{n+1}
    \end{align*}
\end{proof}


\end{document}

%-----------------------------------------------------------------------
% End of amsproc-template.tex
%-----------------------------------------------------------------------
