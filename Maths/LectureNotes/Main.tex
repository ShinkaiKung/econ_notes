% --- LaTeX Lecture Notes Template - S. Venkatraman ---

% --- Set document class and font size ---

\documentclass[letterpaper, 12pt]{article}

% --- Package imports ---

% Extended set of colors
\usepackage[dvipsnames]{xcolor}

\usepackage{
  amsmath, amsthm, amssymb, mathtools, dsfont, units,          % Math typesetting
  graphicx, wrapfig, subfig, float,                            % Figures and graphics formatting
  listings, color, inconsolata, pythonhighlight,               % Code formatting
  fancyhdr, sectsty, hyperref, enumerate, enumitem, framed }   % Headers/footers, section fonts, links, lists

% lipsum is just for generating placeholder text and can be removed
\usepackage{hyperref, lipsum} 

% --- Fonts ---

\usepackage{newpxtext, newpxmath, inconsolata}

% --- Page layout settings ---

% Set page margins
\usepackage[left=1.35in, right=1.35in, top=1.0in, bottom=.9in, headsep=.2in, footskip=0.35in]{geometry}

% Anchor footnotes to the bottom of the page
\usepackage[bottom]{footmisc}

% Set line spacing
\renewcommand{\baselinestretch}{1.2}

% Set spacing between paragraphs
\setlength{\parskip}{1.3mm}

% Allow multi-line equations to break onto the next page
\allowdisplaybreaks

% --- Page formatting settings ---

% Set image captions to be italicized
\usepackage[font={it,footnotesize}]{caption}

% Set link colors for labeled items (blue), citations (red), URLs (orange)
\hypersetup{colorlinks=true, linkcolor=RoyalBlue, citecolor=RedOrange, urlcolor=ForestGreen}

% Set font size for section titles (\large) and subtitles (\normalsize) 
\usepackage{titlesec}
\titleformat{\section}{\large\bfseries}{{\fontsize{19}{19}\selectfont\textreferencemark}\;\; }{0em}{}
\titleformat{\subsection}{\normalsize\bfseries\selectfont}{\thesubsection\;\;\;}{0em}{}

% Enumerated/bulleted lists: make numbers/bullets flush left
%\setlist[enumerate]{wide=2pt, leftmargin=16pt, labelwidth=0pt}
\setlist[itemize]{wide=0pt, leftmargin=16pt, labelwidth=10pt, align=left}

% --- Table of contents settings ---

\usepackage[subfigure]{tocloft}

% Reduce spacing between sections in table of contents
\setlength{\cftbeforesecskip}{.9ex}

% Remove indentation for sections
\cftsetindents{section}{0em}{0em}

% Set font size (\large) for table of contents title
\renewcommand{\cfttoctitlefont}{\large\bfseries}

% Remove numbers/bullets from section titles in table of contents
\makeatletter
\renewcommand{\cftsecpresnum}{\begin{lrbox}{\@tempboxa}}
\renewcommand{\cftsecaftersnum}{\end{lrbox}}
\makeatother

% --- Set path for images ---

\graphicspath{{Images/}{../Images/}}

% --- Math/Statistics commands ---

% Add a reference number to a single line of a multi-line equation
% Usage: "\numberthis\label{labelNameHere}" in an align or gather environment
\newcommand\numberthis{\addtocounter{equation}{1}\tag{\theequation}}

% Shortcut for bold text in math mode, e.g. $\b{X}$
\let\b\mathbf

% Shortcut for bold Greek letters, e.g. $\bg{\beta}$
\let\bg\boldsymbol

% Shortcut for calligraphic script, e.g. %\mc{M}$
\let\mc\mathcal

% \mathscr{(letter here)} is sometimes used to denote vector spaces
\usepackage[mathscr]{euscript}

% Convergence: right arrow with optional text on top
% E.g. $\converge[p]$ for converges in probability
\newcommand{\converge}[1][]{\xrightarrow{#1}}

% Weak convergence: harpoon symbol with optional text on top
% E.g. $\wconverge[n\to\infty]$
\newcommand{\wconverge}[1][]{\stackrel{#1}{\rightharpoonup}}

% Equality: equals sign with optional text on top
% E.g. $X \equals[d] Y$ for equality in distribution
\newcommand{\equals}[1][]{\stackrel{\smash{#1}}{=}}

% Normal distribution: arguments are the mean and variance
% E.g. $\normal{\mu}{\sigma}$
\newcommand{\normal}[2]{\mathcal{N}\left(#1,#2\right)}

% Uniform distribution: arguments are the left and right endpoints
% E.g. $\unif{0}{1}$
\newcommand{\unif}[2]{\text{Uniform}(#1,#2)}

% Independent and identically distributed random variables
% E.g. $ X_1,...,X_n \iid \normal{0}{1}$
\newcommand{\iid}{\stackrel{\smash{\text{iid}}}{\sim}}

% Sequences (this shortcut is mostly to reduce finger strain for small hands)
% E.g. to write $\{A_n\}_{n\geq 1}$, do $\bk{A_n}{n\geq 1}$
\newcommand{\bk}[2]{\{#1\}_{#2}}

% Math mode symbols for common sets and spaces. Example usage: $\R$
\newcommand{\R}{\mathbb{R}}	% Real numbers
\newcommand{\C}{\mathbb{C}}	% Complex numbers
\newcommand{\Q}{\mathbb{Q}}	% Rational numbers
\newcommand{\Z}{\mathbb{Z}}	% Integers
\newcommand{\N}{\mathbb{N}}	% Natural numbers
\newcommand{\F}{\mathcal{F}}	% Calligraphic F for a sigma algebra
\newcommand{\El}{\mathcal{L}}	% Calligraphic L, e.g. for L^p spaces

% Math mode symbols for probability
\newcommand{\pr}{\mathbb{P}}	% Probability measure
\newcommand{\E}{\mathbb{E}}	% Expectation, e.g. $\E(X)$
\newcommand{\var}{\text{Var}}	% Variance, e.g. $\var(X)$
\newcommand{\cov}{\text{Cov}}	% Covariance, e.g. $\cov(X,Y)$
\newcommand{\corr}{\text{Corr}}	% Correlation, e.g. $\corr(X,Y)$
\newcommand{\B}{\mathcal{B}}	% Borel sigma-algebra

% Other miscellaneous symbols
\newcommand{\tth}{\text{th}}	% Non-italicized 'th', e.g. $n^\tth$
\newcommand{\Oh}{\mathcal{O}}	% Big-O notation, e.g. $\O(n)$
\newcommand{\1}{\mathds{1}}	% Indicator function, e.g. $\1_A$

% Additional commands for math mode
\DeclareMathOperator*{\argmax}{argmax}		% Argmax, e.g. $\argmax_{x\in[0,1]} f(x)$
\DeclareMathOperator*{\argmin}{argmin}		% Argmin, e.g. $\argmin_{x\in[0,1]} f(x)$
\DeclareMathOperator*{\spann}{Span}		% Span, e.g. $\spann\{X_1,...,X_n\}$
\DeclareMathOperator*{\bias}{Bias}		% Bias, e.g. $\bias(\hat\theta)$
\DeclareMathOperator*{\ran}{ran}			% Range of an operator, e.g. $\ran(T) 
\DeclareMathOperator*{\dv}{d\!}			% Non-italicized 'with respect to', e.g. $\int f(x) \dv x$
\DeclareMathOperator*{\diag}{diag}		% Diagonal of a matrix, e.g. $\diag(M)$
\DeclareMathOperator*{\trace}{trace}		% Trace of a matrix, e.g. $\trace(M)$
\DeclareMathOperator*{\supp}{supp}		% Support of a function, e.g., $\supp(f)$

% Numbered theorem, lemma, etc. settings - e.g., a definition, lemma, and theorem appearing in that 
% order in Lecture 2 will be numbered Definition 2.1, Lemma 2.2, Theorem 2.3. 
% Example usage: \begin{theorem}[Name of theorem] Theorem statement \end{theorem}
\theoremstyle{definition}
\newtheorem{theorem}{Theorem}[section]
\newtheorem{proposition}[theorem]{Proposition}
\newtheorem{lemma}[theorem]{Lemma}
\newtheorem{corollary}[theorem]{Corollary}
\newtheorem{definition}[theorem]{Definition}
\newtheorem{example}[theorem]{Example}
\newtheorem{remark}[theorem]{Remark}

% Un-numbered theorem, lemma, etc. settings
% Example usage: \begin{lemma*}[Name of lemma] Lemma statement \end{lemma*}
\newtheorem*{theorem*}{Theorem}
\newtheorem*{proposition*}{Proposition}
\newtheorem*{lemma*}{Lemma}
\newtheorem*{corollary*}{Corollary}
\newtheorem*{definition*}{Definition}
\newtheorem*{example*}{Example}
\newtheorem*{remark*}{Remark}
\newtheorem*{claim}{Claim}

% --- Left/right header text (to appear on every page) ---

% Do not include a line under header or above footer
\pagestyle{fancy}
\renewcommand{\footrulewidth}{0pt}
\renewcommand{\headrulewidth}{0pt}

% Right header text: Lecture number and title
\renewcommand{\sectionmark}[1]{\markright{#1} }
\fancyhead[R]{\small\textit{\nouppercase{\rightmark}}}

% Left header text: Short course title, hyperlinked to table of contents
\fancyhead[L]{\hyperref[sec:contents]{\small Maths Camp}}

% --- Document starts here ---

\begin{document}

% --- Main title and subtitle ---

\title{Mathematics \\[1em]
\normalsize Econ Preparation}

% --- Author and date of last update ---

\author{\normalsize Zian Gong}
\date{\normalsize\vspace{-1ex} Last updated: \today}

% --- Add title and table of contents ---

\maketitle
\tableofcontents\label{sec:contents}

% --- Main content: import lectures as subfiles ---

% % TeX root = ../Main.tex
% First argument to \section is the title that will go in the table of contents. Second argument is the title that will be printed on the page.
\section[Pre-requirements]{Lecture 1}

\subsection{Matrices}

\subsubsection{Transpose}

\begin{definition}[Transpose]
	Let $A$ be an $n \times m$ (i.e. a matrix with $n$ rows and $m$ columns). The \textit{\textbf{transpose}} $A'$ of $A$ is the $m \times n$ matrix in which, for $i = 1, ..., m$, the $i$th row is the $i$th column of $A$.
\end{definition}


In particular, if $x$ is a column vector ($n \times 1$ matrix) then $x'$ is a row vector.

\begin{example}
	\begin{equation*}
		\text{If}\, A = \begin{pmatrix}
			a & b \\
			c & d
		\end{pmatrix},\quad \text{then}\, A' = \begin{pmatrix}
			a & c \\
			b & d
		\end{pmatrix}
	\end{equation*}
\end{example}

\subsubsection{Determinant}


\textit{\textbf{Determinant}} is an important characteristic of \textit{\textbf{Square Matrix}}, which has the same number of rows as columns.

\begin{definition}[Determinant]
	The \textit{\textbf{determinant}} of a $1 \times 1$ matrix is the single number in the matrix. For any $n \geq 2$, the \textit{\textbf{determinant}} of the $n \times n$ matrix A is
	\begin{equation*}
		|A| = \sum_{j=1}^{n} (-1)^{1+j}a_{1j}|A_{1j}|
	\end{equation*}
\end{definition}

\begin{proposition}[Calculate determinant]
	\begin{equation*}
		|A| = \sum_{j=1}^{n} (-1)^{i+j}a_{ij}|A_{ij}| = \sum_{i=1}^{n} (-1)^{i+j}a_{ij}|A_{ij}|
	\end{equation*}
\end{proposition}

\begin{example}
	\begin{equation*}
		\begin{vmatrix}
			a & b \\
			c & d
		\end{vmatrix} = ad - bc
	\end{equation*}

	\begin{equation*}
		\begin{vmatrix}
			a & b & c \\
			c & d & e \\
			f & g & h
		\end{vmatrix} = a(dh-eg) - b(ch-ef) + c(cg-df)
	\end{equation*}
\end{example}

If the square matrix $A$ like $\begin{pmatrix}
		a_{00} & 0      & \dots  & 0      \\
		0      & a_{11} & \dots  & 0      \\
		\vdots & \vdots & \ddots & \vdots \\
		0      & 0      & \dots  & a_{nn}
	\end{pmatrix}$, then $|A|$ equals $\prod_{i=0}^{n}a_{ii}$.

\subsubsection{Inverse}

\begin{definition}[Nonsingular]
	The square matrix is \textit{\textbf{nonsingular}} if its determinant is not zero.
\end{definition}

\begin{definition}
	Let $A$ be an $n \times n$ matrix. If there exists an $n \times n$ matrix $B$ such that
	\begin{equation*}
		AB = BA = I
	\end{equation*},
	where $I$ is the $n \times n$ identity matrix, then $A$ is said to be \textit{\textbf{invertible}} (or \textit{\textbf{nonsingular}}), and $B$ is called the \textit{\textbf{inverse}} of $A$, denoted by $A^{-1}$.
\end{definition}

\begin{proposition}
	A square matrix has at most one inverse.
\end{proposition}

\begin{proposition}
	A matrix has an inverse if and only if it is nonsingular.
\end{proposition}

\begin{proposition}
	The inverse of the nonsingular matrix $A$ is the $n \times n$ matrix for the $(i,j)$th component is \begin{equation*}
		(-1)^{i+j}\frac{|A_{ji}|}{|A|}.
	\end{equation*}
\end{proposition}

\begin{example}
	\begin{equation*}
		A = \begin{pmatrix}
			a & b \\
			c & d
		\end{pmatrix},\quad A^{-1} = \frac{1}{ad-bc}\begin{pmatrix}
			d  & -b \\
			-c & a
		\end{pmatrix}
	\end{equation*}
\end{example}
\begin{example}
	\begin{equation*}
		% 3 x 3 matrix
		A = \begin{pmatrix}
			a & b & c \\
			d & e & f \\
			g & h & i
		\end{pmatrix},\quad A^{-1}=\frac{1}{|A|}\begin{pmatrix}
			|A_{11}   & -|A_{21}| & |A_{31}|  \\
			-|A_{12}| & |A_{22}|  & -|A_{32}| \\
			|A_{13}|  & -|A_{23}| & |A_{33}|
		\end{pmatrix}
	\end{equation*}
\end{example}

\subsubsection{Rank}

\begin{definition}
	The rank of a matrix $A$ is the number of rows and columns in the \textbf{largest} square matrix obtained by deleting rows and columns of $A$ that has a determinant different from 0.
\end{definition}

\subsection{Linear Equations}

\subsubsection{Solutions of Linear Equations}

Linear Equations can be written in matrix form, as $Ax=b$.

If $A$ is nonsingular, then the solution $x = A^{-1}b$.

\begin{definition}[Augmented Matrix]
	The \textit{\textbf{augmented matrix}} of the system of linear equations $Ax = b$ is the $n \times (m+1)$ matrix obtained by appending the column vector $b$ to the matrix $A$.
\end{definition}

\begin{proposition}
	Let $A$ be an $n \times n$ matrix.
	\begin{itemize}
		\item If $rank(A) = n$ (nonsingular), then the system $Ax = b$ has a unique solution.
		\item If $rank(A) < n$ and $rank(A|b) = rank(A)$, then the system $Ax = b$ has infinitely many solutions.
		\item If $rank(A) < n$ and $rank(A|b) > rank(A)$, then the system $Ax = b$ has no solution.
	\end{itemize}
\end{proposition}


\subsubsection{Cramer's Rule}


\begin{proposition}
	Let $A$ be an $n \times n$ matrix, let $b$ be an $n \times 1$ column vector, and consider the system of linear equations $Ax=b$ where $x$ is an $n \times 1$ column vector. if $A$ is nonsingular then the (unique) value of $x$ that satisfies the system is given by \begin{equation*}
		x_{i} = \frac{|A^{*(b,i)}|}{|A|} \quad\text{for}\quad i = 1, \dots, n,
	\end{equation*}
	where $A^{*(b,i)}$ is the matrix obtained from $A$ by replacing the $i$th column with $b$.
\end{proposition}

Cramer's Rule is partially useful if you want to calculate the value of only some og the variables in a solution of a system of linear equations.



\newpage
% TeX root = ../Main.tex
% First argument to \section is the title that will go in the table of contents. Second argument is the title that will be printed on the page.
\section[Univariate Analysis]{Univariate Analysis}

\subsection{Limits and Continuity}

\subsubsection{Limits}

\begin{definition}
    $\forall \epsilon > 0, \exists \eta >0$ such that if $|x - a| < \eta$, then $|f(x) - A| < \epsilon$.
    \begin{equation*}
        \lim_{x \to a} f(x) = A
    \end{equation*}
\end{definition}

\begin{remark*}
    Need to prove with definition.
\end{remark*}


\subsubsection{LHS\&RHS}

\begin{proposition}[$LHS=RHS$]
    One-sided limits:
    \begin{itemize}
        \item LHS limit: $\lim_{x \to a^-} f(x)=B_1$ if $f(x) \to B_1$ as $x \to a^-$.
        \item RHS limit: $\lim_{x \to a^+} f(x)=B_2$ if $f(x) \to B_2$ as $x \to a^+$.
    \end{itemize}
    A limit at $a$ exists $\iff \text{LHS limit} = \text{RHS limit} \iff B_1=B_2$.
\end{proposition}

Limits Rules: if $\lim_{x \to a}f(x)=A $ and $\lim_{x \to a}g(x)=B $, then:
\begin{itemize}
    \item $\lim_{x \to a}[f(x) \pm g(x)] = A \pm B $,
    \item $\lim_{x \to a} f(x)g(x) = AB $,
    \item $\lim_{x \to a}\frac{f(x)}{g(x)} = \frac{A}{B} (\text{if } B \neq 0) $,
    \item $\lim_{x \to a} [f(x)]^r = A^r(\text{if $A^r$ is defined and $r$ is a real number})$.
\end{itemize}

\subsubsection{Limits at Infinity}

\begin{definition}[$\lim_{x \to \pm\infty} f(x)$]Limits at Infinity:

    $\forall \epsilon > 0, \exists N > 0$ such that if $x > N$, then $|f(x) - L|< \epsilon$.
    \begin{equation*}
        \lim_{x \to \infty} f(x) = L
    \end{equation*}

    $\forall \epsilon > 0, \exists N > 0$ such that if $x < -N$, then $|f(x) - L|< \epsilon$.
    \begin{equation*}
        \lim_{x \to -\infty} f(x) = L
    \end{equation*}
\end{definition}

\subsubsection{Continuity}

\begin{definition}
    A function $f$ is continuous at a given point $x=a$ if $\lim_{x \to a} f(x) = f(a)$. $\iff \forall \epsilon > 0, \exists \delta > 0$ such that if $|x-a|<\delta$ then $|f(x) - f(a)| < \epsilon$.
\end{definition}

Properties: if $f$ and $g$ are continuous on $a$, then
\begin{itemize}
    \item $f \pm g$ is continuous on $a$,
    \item $fg$ is continuous on $a$,
    \item $f / g$ is continuous on $a$, if $g(a) \neq 0$,
    \item $f(x)^{r}$ is continuous on $a$ if $f(x)^{r}$ is defined and $r$ is a real number.
\end{itemize}


\begin{theorem}[Intermediate Value Theorem]
    If $f(x)$ is continuous on $[a,b]$, $\forall y \in [f(a), f(b)]$(assume $f(a) < f(b)$), then $\exists c \in [a,b]$ such that $y=f(c)$.
\end{theorem}

\begin{theorem}[Bolzano's Theorem]
    If $f(x)$ is continuous on $[a,b]$ and assume $f(a)f(b)<0$, then $\exists c \in [a,b]$ such that $f(c) = 0$.
    \begin{remark*}
        It's a special case of "Intermediate Value Theorem".
    \end{remark*}
\end{theorem}

\subsection{Differentiation}

\subsubsection{Differentiation}

\begin{definition}
    We say a function $f$ is \textit{\textbf{differentiable}} at $x$ if
    \begin{equation*}
        \frac{d f(x)}{dx} = \lim_{h \to 0}\frac{f(x+h)-f(x)}{h}
    \end{equation*}
    exists.
\end{definition}

\begin{definition}
    If function $f$ is differentiable for all $x$ in its domain, then we say $f$ is a \textbf{\textit{differentiable function}}.
\end{definition}

\begin{remark*}
    Function $f$ is differentiable at point A $\iff $
    \begin{itemize}
        \item It's continuous at point A.
        \item Its $LHS = RHS$.
    \end{itemize}
\end{remark*}

\subsubsection{Higher-order Derivatives}

Notation: The $k$th order differentiation is denoted $\frac{d ^{k}f(x)}{dx ^{k}}$ or $f ^{(k)}(x)$. The class of all $k-$times continuously differentiable functions is called $C^k$.

\subsubsection{Implicit Differentiation}

\begin{proposition}
    For $f(x, y(x)) = 0$, we have \begin{equation*}
        \frac{df}{dx} + \frac{df}{dy}\frac{dy}{dx} = 0
    \end{equation*}
\end{proposition}

\begin{example*}
    %TODO:
\end{example*}


\subsubsection{Inverse Function Theorem}

\begin{theorem}[Inverse Function Theorem]
    If $f$ is differentiable and strictly increasing or decreasing in an interval $I$, then $f$ has an inverse function $g$. If $x_0$ is an intetior point of $I$ and $f'(x_0) \neq 0$, then $g$ is differentiable at $y_0 = f(x_0)$ and: \begin{equation*}
        g'(x_0) = \frac{1}{f'(x_0)} = \frac{1}{f'[g(y_0)]}.
    \end{equation*}
\end{theorem}

\begin{example*}
    %TODO:
\end{example*}

\subsubsection{Taylor Expansions}

\textbf{Taylor expansions}:
\begin{align*}
    f(x) & \approx f(c) + f'(c)(x-c) + \frac{1}{2!}f''(c)(x-c)^{2} + \dots + \frac{1}{k!}f ^{(k)}(c)(x-c)^{k} \\
         & \approx f(c) + \sum_{i=1}^{k}\frac{f ^{(i)}(c)}{i!}(x-c)^{i}
\end{align*}

\begin{theorem}[Rolle's Theorem]
    Let $f: [a, b] \to \mathbb{R}$. If $f$ is continuous on $[a, b]$ and differentiable on (a, b). If $f(a)=f(b)$, then $\exists c \in (a,b)$ such that $f'(c) = 0$.
\end{theorem}

\begin{theorem}[Mean Value Theorem]
    Let $f: [a, b] \to \mathbb{R}$. If $f$ is continuous on $[a, b]$ and differentiable on (a, b), then $\exists z \in (a,b)$ such that:
    \begin{equation*}
        f'(z) = \frac{f(b)- f(a)}{b-a}.
    \end{equation*}
\end{theorem}

\begin{proposition}
    \begin{equation*}
        f(x) = f(c) + \sum_{i=1}^{k}\frac{f ^{(i)}(c)}{i!}(x-c)^{i} + \frac{f ^{(k+1)}(z)}{(k+1)!}(x-x_0)^{k+1}
    \end{equation*}
\end{proposition}


\begin{proof}[\textbf{Proof of Taylor expansions}]:

    The first-order: \begin{equation*}
        f(x) \approx f(x_0) + f'(x_0)(x-x_0)
    \end{equation*}

    The higher order approximation:
    \begin{align*}
        P(x) & = a_0 + a_{1}(x-x_0) + a_{2}(x-x_0)^{2} + a_{3}(x-x_0)^{3} + a_{n}(x-x_0)^{n} + U(x) = f(x) \\
        P(x) & = Q(x) + U(x) = f(x)
    \end{align*}

    We have:
    \begin{align*}
        a_0     & = f(x_0)       \\
        a_{1}   & = f'(x_0)      \\
        2!a_{2} & = f''(x_0)     \\
        n!a_{n} & = f^{(n)}(x_0)
    \end{align*}
    then $a_{0} = f(x_0), a_{1} = f'(x_0), a_{2}=\frac{f''(x_0)}{2!}, \dots, a_{n} = \frac{f^{(n)}(x_0)}{n!} \Rightarrow Q(x) = \sum_{i=1}^{n} \frac{f ^{(k)}(x_0)}{k!}(x-x_0)^{n}$

    For functions $U(x), R(x) = (x-x_0)^{n+1}$, they are high-order continuous and differentiable on interval $I$. And $U(x_0) = R(x_0) = 0, U'(x_1) = R'(x_1) = 0, U''(x_0) = R''(x_0) = 0, \dots, U^{(n)}(x_0) = R^{(n)}(x_0) = 0$, then using "\textit{\textbf{Cauchy Mean Value Theorem}}": \begin{align*}
         & \text{On domain $[x_0, x]$, there $\exists c_1 \in (x_0, x)$ such that}                                                                                                                                       \\
         & \frac{U(x) - U(x_0)}{R(x)- R(x_0)} = \frac{U'(c_1)}{R'(c_1)} \Rightarrow  \frac{U(x)}{R(x)} = \frac{U'(c_1)}{R'(c_1)}                                                                                         \\
         & \text{On domain $[x_0, c_1]$, there $\exists c_2 \in (x_0, c_1)$ such that}                                                                                                                                   \\
         & \frac{U'(c_1) - U'(x_0)}{R'(c_1)- R'(x_0)} = \frac{U''(c_2)}{R''(c_2)} \Rightarrow  \frac{U'(c_1)}{R'(c_1)} = \frac{U''(c_2)}{R''(c_2)}                                                                       \\
         & \dots                                                                                                                                                                                                         \\
         & \text{On domain $[x_0, c_n]$, there $\exists c_{n+1} \in (x_0, c_n)$ such that}                                                                                                                               \\
         & \frac{U^{(n)}(c_n) - U^{(n)}(x_0)}{R^{(n)}(c_n)- R^{(n)}(x_0)} = \frac{U^{(n+1)}(c_{n+1})}{R^{(n+1)}(c_{n+1})} \Rightarrow  \frac{U^{(n)}(c_n)}{R^{(n)}(c_n)} = \frac{U^{(n+1)}(c_{n+1})}{R^{(n+1)}(c_{n+1})} \\
         & \text{Then}                                                                                                                                                                                                   \\
         & \frac{U(x)}{R(x)} = \frac{U'(c_1)}{R'(c_1)} = \frac{U''(c_2)}{R''(c_2)} = \dots = \frac{U^{(n+1)}(c_{n+1})}{R^{(n+1)}(c_{n+1})}                                                                               \\
         & U(x) = \frac{U^{(n+1)}(c_{n+1})}{(n+1)!}(x-x_0)^{n+1}                                                                                                                                                         \\
         & \text{Since $f^{(n+1)}(x) = Q^{(n+1)}(x) + U^{(n+1)}(x) = U^{(n+1)}(x)$, then}                                                                                                                                \\
         & U(x) = \frac{f^{(n+1)}(c_{n+1})}{(n+1)!}(x-x_0)^{n+1}
    \end{align*}
\end{proof}


\newpage

% % TeX root = ../Main.tex
% First argument to \section is the title that will go in the table of contents. Second argument is the title that will be printed on the page.
\section[Analysis Topics]{Analysis Topics}

\subsection{Metric Space}

\begin{definition}
	A set $X$ is said to be a \textbf{\textit{metric space}} if with any two points $p$ and $q$ of $X$, there is an associated real number $d(p,q)$ called the \textbf{\textit{distance}} from $p$ to $q$ such that:
	\begin{enumerate}
		\item $d(p,q) > 0$ if $p \neq q$ and $d(p,q) = 0$ if $p = q$,
		\item $d(p,q) = d(q,p)$, and
		\item $d(p,q) \leq d(p,r) + d(r,q)$ for any $r \in X$ (Triangle inequality).
	\end{enumerate}
\end{definition}

\begin{remark}
	Any function with these properties is called a \textit{\textbf{distance function}} or a \textit{\textbf{metric}}.
	
\end{remark}



% --- Bibliography ---

% Start a bibliography with one item.
% Citation example: "\cite{williams}".

\begin{thebibliography}{1}

% \bibitem{williams}
%    Williams, David.
%    \textit{Probability with Martingales}.
%    Cambridge University Press, 1991.
%    Print.

% Uncomment the following lines to include a webpage
% \bibitem{webpage1}
%   LastName, FirstName. ``Webpage Title''.
%   WebsiteName, OrganizationName.
%   Online; accessed Month Date, Year.\\
%   \texttt{www.URLhere.com}

\end{thebibliography}

% --- Document ends here ---

\end{document}
