% --- LaTeX Homework Template - S. Venkatraman ---

% --- Set document class and font size ---

\documentclass[letterpaper, 11pt]{article}

% --- Package imports ---

\usepackage{
  amsmath, amsthm, amssymb, mathtools, dsfont,	  % Math typesetting
  graphicx, wrapfig, subfig, float,                  % Figures and graphics formatting
  listings, color, inconsolata, pythonhighlight,     % Code formatting
  fancyhdr, sectsty, hyperref, enumerate, enumitem } % Headers/footers, section fonts, links, lists

% --- Page layout settings ---

% Set page margins
\usepackage[left=1.35in, right=1.35in, bottom=1in, top=1.1in, headsep=0.2in]{geometry}

% Anchor footnotes to the bottom of the page
\usepackage[bottom]{footmisc}

% Set line spacing
\renewcommand{\baselinestretch}{1}

% Set spacing between paragraphs
\setlength{\parskip}{1.5mm}

% Allow multi-line equations to break onto the next page
\allowdisplaybreaks

% Enumerated lists: make numbers flush left, with parentheses around them
\setlist[enumerate]{wide=0pt, leftmargin=21pt, labelwidth=0pt, align=left}
\setenumerate[1]{label={(\arabic*)}}

% --- Page formatting settings ---

% Set link colors for labeled items (blue) and citations (red)
\hypersetup{colorlinks=true, linkcolor=blue, citecolor=red}

% Make reference section title font smaller
\renewcommand{\refname}{\large\bf{References}}

% --- Settings for printing computer code ---

% Define colors for green text (comments), grey text (line numbers),
% and green frame around code
\definecolor{greenText}{rgb}{0.5, 0.7, 0.5}
\definecolor{greyText}{rgb}{0.5, 0.5, 0.5}
\definecolor{codeFrame}{rgb}{0.5, 0.7, 0.5}

% Define code settings
\lstdefinestyle{code} {
  frame=single, rulecolor=\color{codeFrame},            % Include a green frame around the code
  numbers=left,                                         % Include line numbers
  numbersep=8pt,                                        % Add space between line numbers and frame
  numberstyle=\tiny\color{greyText},                    % Line number font size (tiny) and color (grey)
  commentstyle=\color{greenText},                       % Put comments in green text
  basicstyle=\linespread{1.1}\ttfamily\footnotesize,    % Set code line spacing
  keywordstyle=\ttfamily\footnotesize,                  % No special formatting for keywords
  showstringspaces=false,                               % No marks for spaces
  xleftmargin=1.95em,                                   % Align code frame with main text
  framexleftmargin=1.6em,                               % Extend frame left margin to include line numbers
  breaklines=true,                                      % Wrap long lines of code
  postbreak=\mbox{\textcolor{greenText}{$\hookto$}\space} % Mark wrapped lines with an arrow
}

% Set all code listings to be styled with the above settings
\lstset{style=code}

% --- Math/Statistics commands ---

% Add a reference number to a single line of a multi-line equation
% Usage: "\numberthis\label{labelNameHere}" in an align or gather environment
\newcommand\numberthis{\addtocounter{equation}{1}\tag{\theequation}}

% Shortcut for bold text in math mode, e.g. $\b{X}$
\let\b\mathbf

% Shortcut for bold Greek letters, e.g. $\bg{\beta}$
\let\bg\boldsymbol

% Shortcut for calligraphic script, e.g. %\mc{M}$
\let\mc\mathcal

% \mathscr{(letter here)} is sometimes used to denote vector spaces
\usepackage[mathscr]{euscript}

% Convergence: right arrow with optional text on top
% E.g. $\converge[w]$ for weak convergence
\newcommand{\converge}[1][]{\xto{#1}}

% Normal distribution: arguments are the mean and variance
% E.g. $\normal{\mu}{\sigma}$
\newcommand{\normal}[2]{\mathcal{N}\left(#1,#2\right)}

% Uniform distribution: arguments are the left and right endpoints
% E.g. $\unif{0}{1}$
\newcommand{\unif}[2]{\text{Uniform}(#1,#2)}

% Independent and identically distributed random variables
% E.g. $ X_1,...,X_n \iid \normal{0}{1}$
\newcommand{\iid}{\stackrel{\smash{\text{iid}}}{\sim}}

% Equality: equals sign with optional text on top
% E.g. $X \equals[d] Y$ for equality in distribution
\newcommand{\equals}[1][]{\stackrel{\smash{#1}}{=}}

% Math mode symbols for common sets and spaces. Example usage: $\R$
\newcommand{\R}{\mathbb{R}}   % Real numbers
\newcommand{\C}{\mathbb{C}}   % Complex numbers
\newcommand{\Q}{\mathbb{Q}}   % Rational numbers
\newcommand{\Z}{\mathbb{Z}}   % Integers
\newcommand{\N}{\mathbb{N}}   % Natural numbers
\newcommand{\F}{\mathcal{F}}  % Calligraphic F for a sigma algebra
\newcommand{\El}{\mathcal{L}} % Calligraphic L, e.g. for L^p spaces

% Math mode symbols for probability
\newcommand{\pr}{\mathbb{P}}    % Probability measure
\newcommand{\E}{\mathbb{E}}     % Expectation, e.g. $\E(X)$
\newcommand{\var}{\text{Var}}   % Variance, e.g. $\var(X)$
\newcommand{\cov}{\text{Cov}}   % Covariance, e.g. $\cov(X,Y)$
\newcommand{\corr}{\text{Corr}} % Correlation, e.g. $\corr(X,Y)$
\newcommand{\B}{\mathcal{B}}    % Borel sigma-algebra

% Other miscellaneous symbols
\newcommand{\tth}{\text{th}}	% Non-italicized 'th', e.g. $n^\tth$
\newcommand{\Oh}{\mathcal{O}}	% Big-O notation, e.g. $\O(n)$
\newcommand{\1}{\mathds{1}}	% Indicator function, e.g. $\1_A$

% Additional commands for math mode
\DeclareMathOperator*{\argmax}{argmax}    % Argmax, e.g. $\argmax_{x\in[0,1]} f(x)$
\DeclareMathOperator*{\argmin}{argmin}    % Argmin, e.g. $\argmin_{x\in[0,1]} f(x)$
\DeclareMathOperator*{\spann}{Span}       % Span, e.g. $\spann\{X_1,...,X_n\}$
\DeclareMathOperator*{\bias}{Bias}        % Bias, e.g. $\bias(\hat\theta)$
\DeclareMathOperator*{\ran}{ran}          % Range of an operator, e.g. $\ran(T) 
\DeclareMathOperator*{\dv}{d\!}           % Non-italicized 'with respect to', e.g. $\int f(x) \dv x$
\DeclareMathOperator*{\diag}{diag}        % Diagonal of a matrix, e.g. $\diag(M)$
\DeclareMathOperator*{\trace}{trace}      % Trace of a matrix, e.g. $\trace(M)$

% Numbered theorem, lemma, etc. settings - e.g., a definition, lemma, and theorem appearing in that 
% order in Section 2 will be numbered Definition 2.1, Lemma 2.2, Theorem 2.3. 
% Example usage: \begin{theorem}[Name of theorem] Theorem statement \end{theorem}
\theoremstyle{definition}
\newtheorem{theorem}{Theorem}[section]
\newtheorem{proposition}[theorem]{Proposition}
\newtheorem{lemma}[theorem]{Lemma}
\newtheorem{corollary}[theorem]{Corollary}
\newtheorem{definition}[theorem]{Definition}
\newtheorem{example}[theorem]{Example}
\newtheorem{remark}[theorem]{Remark}

% Un-numbered theorem, lemma, etc. settings
% Example usage: \begin{lemma*}[Name of lemma] Lemma statement \end{lemma*}
\newtheorem*{theorem*}{Theorem}
\newtheorem*{proposition*}{Proposition}
\newtheorem*{lemma*}{Lemma}
\newtheorem*{corollary*}{Corollary}
\newtheorem*{definition*}{Definition}
\newtheorem*{example*}{Example}
\newtheorem*{remark*}{Remark}
\newtheorem*{claim}{Claim}

% --- Left/right header text (to appear on every page) ---

% Include a line underneath the header, no footer line
\pagestyle{fancy}
\renewcommand{\footrulewidth}{0pt}
\renewcommand{\headrulewidth}{0.4pt}

% Left header text: course name/assignment number
\lhead{MATHEMATICS - Assignment}

% Right header text: your name
\rhead{Zian Gong}

% --- Document starts here ---

\begin{document}
\subsection*{Problem 1}

Solution:

\begin{align*}
  \frac{d f(x)}{dx} & = \lim_{h \to 0} \frac{f(x+h)-f(x)}{h}                                                                                         \\
                    & =\lim_{h \to 0} \frac{(x+h)^{3}-(x+h)^{2}+(x+h)-1 - (x ^{3} - x ^{2} + x -1)}{h}                                               \\
                    & = \lim_{h \to 0} \frac{x ^{3} + 2 x ^{2}h + 3xh ^{2} + h ^{3} - x ^{2} - 2xh - h ^{2} + x + h - 1 - x ^{3} + x ^{2} - x +1}{h} \\
                    & = \lim_{h \to 0} \frac{h ^{3} + (3x-1)h ^{2}+(3x ^{2} - 2x + 1)h}{h}                                                           \\
                    & = 3x ^{2} - 2x + 1
\end{align*}

\subsection*{Problem 2}

Solution:

a)

\begin{align*}
  f'(x)  & = -1 - 4x + 3x ^{2} \\
  f''(x) & = -4 + 6x
\end{align*}

b)

\begin{align*}
  f'(x) < 0 \Longrightarrow 0 < x < \frac{\sqrt{7}+2}{3} \\
  f'(x) > 0 \Longrightarrow x > \frac{\sqrt{7}+2}{3}
\end{align*}

So $f(x)$ is increasing on $\{x: x \in (\frac{\sqrt{7}+2}{3}, \infty )\}$, decreasing on $\{x: x \in (0, \frac{\sqrt{7}+2}{3})\}$.

c)

\begin{align*}
  f''(x) < 0 \Longrightarrow x < \frac{2}{3} \\
  f''(x) > 0 \Longrightarrow x > \frac{2}{3}
\end{align*}

So $f(x)$ is convex on $\{x: x \in (\frac{2}{3}, \infty )\}$, concave on $\{x: x \in (0, \frac{2}{3})\}$.

\subsection*{Problem 3}

\begin{align*}
  \int_{0}^{1} x ^{2}e ^{-2x} \, dx & = -\frac{1}{2}\int_{0}^{1} x ^{2} \, d(e ^{-2x})                                                               \\
                                    & = -\frac{1}{2}\left\{\left[x ^{2}e ^{-2x}\right]_{0}^{1} - \int_{0}^{1} e ^{-2x} \, d(x ^{2})\right\}          \\
                                    & =-\frac{1}{2} \left\{e ^{-2} - \int_{0}^{1} 2xe ^{-2x} \, dx\right\}                                           \\
                                    & =-\frac{1}{2}e ^{-2} + \int_{0}^{1} x e ^{-2x} \, dx                                                           \\
                                    & = -\frac{1}{2}e ^{-2} -\frac{1}{2}\int_{0}^{1} x \, d(e ^{-2x})                                                \\
                                    & = -\frac{1}{2}e ^{-2} - \frac{1}{2}\left\{\left[xe ^{-2x}\right]_{0}^{1} - \int_{0}^{1} e ^{-2x} \, dx\right\} \\
                                    & = -\frac{1}{2}e ^{-2} - \frac{1}{2}\left(e ^{-2} - \int_{0}^{1} e ^{-2x} \, dx\right)                          \\
                                    & = -e ^{-2} + \frac{1}{2}\int_{0}^{1} e ^{-2x} \, dx                                                            \\
                                    & = -e ^{-2} - \frac{1}{4}\int_{0}^{1} e ^{-2x} \, d(-2x)                                                        \\
                                    & = -e ^{-2} - \frac{1}{4}\left[e ^{-2x}\right]_{0}^{1}                                                          \\
                                    & = -e ^{-2} - \frac{1}{4}(e ^{-2} - 1)                                                                          \\
                                    & = -\frac{5}{4}e ^{-2} + \frac{1}{4}
\end{align*}

\subsection*{Problem 4}

Solution:

a)

Using elementary row tranform:
\begin{align*}
  (A|I) = \left(\begin{array}{ccc|ccc}
                  1 & 0 & 3  & 1 & 0 & 0 \\
                  0 & 2 & 2  & 0 & 1 & 0 \\
                  3 & 2 & -1 & 0 & 0 & 1
                \end{array}\right)                              \\
  \to \left(\begin{array}{ccc|ccc}
              1 & 0 & 3   & 1  & 0 & 0 \\
              0 & 2 & 2   & 0  & 1 & 0 \\
              0 & 2 & -10 & -3 & 0 & 1
            \end{array}\right)                                \\
  \to \left(\begin{array}{ccc|ccc}
              1 & 0 & 3   & 1  & 0 & 0 \\
              0 & 2 & 2   & 0  & 1 & 0 \\
              0 & 2 & -10 & -3 & 0 & 1
            \end{array}\right)                                \\
  \to \left(\begin{array}{ccc|ccc}
              1 & 0 & 3   & 1  & 0  & 0 \\
              0 & 2 & 2   & 0  & 1  & 0 \\
              0 & 0 & -12 & -3 & -1 & 1
            \end{array}\right)                               \\
  \to \left(\begin{array}{ccc|ccc}
              1 & 0 & 3 & 1           & 0            & 0             \\
              0 & 1 & 1 & 0           & \frac{1}{2}  & 0             \\
              0 & 0 & 1 & \frac{1}{4} & \frac{1}{12} & -\frac{1}{12}
            \end{array}\right)  \\
  \to \left(\begin{array}{ccc|ccc}
              1 & 0 & 3 & 1            & 0            & 0             \\
              0 & 1 & 0 & -\frac{1}{4} & \frac{5}{12} & \frac{1}{12}  \\
              0 & 0 & 1 & \frac{1}{4}  & \frac{1}{12} & -\frac{1}{12}
            \end{array}\right) \\
  \to \left(\begin{array}{ccc|ccc}
              1 & 0 & 0 & \frac{1}{4}  & -\frac{1}{4} & \frac{1}{4}   \\
              0 & 1 & 0 & -\frac{1}{4} & \frac{5}{12} & \frac{1}{12}  \\
              0 & 0 & 1 & \frac{1}{4}  & \frac{1}{12} & -\frac{1}{12}
            \end{array}\right) \\
\end{align*}

So the $A ^{-1} = \begin{pmatrix}
    \frac{1}{4}  & -\frac{1}{4} & \frac{1}{4}   \\
    -\frac{1}{4} & \frac{5}{12} & \frac{1}{12}  \\
    \frac{1}{4}  & \frac{1}{12} & -\frac{1}{12}
  \end{pmatrix}$.

b)

First step, calculate the eigenvalues and eigenvectors. $Ax = \lambda x \to (A - \lambda I)x = 0$.

\begin{align*}
  \det(A - \lambda I) & = \begin{vmatrix}
                            1-\lambda & 0         & 3          \\
                            0         & 2-\lambda & 2          \\
                            3         & 2         & -1-\lambda
                          \end{vmatrix} = (2-\lambda)(\lambda+1)(\lambda-1) - 9(2-\lambda) -4 (1-\lambda) \\
                      & = -\lambda ^{3} + 2 \lambda ^{2} + 14 \lambda - 24                                \\
                      & =(\lambda-4)(-\lambda ^{2}-2\lambda+6) 0                                          \\
  \lambda             & = \left\{\begin{array}{l}
                                   \lambda_1 = 4             \\
                                   \lambda_2 = \sqrt{7} - 1  \\
                                   \lambda_3 = -\sqrt{7} - 1 \\
                                 \end{array}\right.
\end{align*}

While $\lambda = 4$, $A - \lambda I = \begin{pmatrix}
    -3 & 0  & 3  \\
    0  & -2 & 2  \\
    3  & 2  & -5
  \end{pmatrix} \to \begin{pmatrix}
    1 & 0 & -1 \\
    0 & 1 & -1 \\
    0 & 0 & 0
  \end{pmatrix}$, a specific eigenvector is $x_1 = \begin{pmatrix}
    1 \\ 1 \\ 1
  \end{pmatrix}$.

While $\lambda = \sqrt{7} - 1$, $A - \lambda I = \begin{pmatrix}
    2-\sqrt{7} & 0          & 3         \\
    0          & 3-\sqrt{7} & 2         \\
    3          & 2          & -\sqrt{7}
  \end{pmatrix} \to \begin{pmatrix}
    1 & 0 & -2-\sqrt{7} \\
    0 & 1 & 3+\sqrt{7}  \\
    0 & 0 & 0
  \end{pmatrix}$, a specific eigenvector is $x_2 = \begin{pmatrix}
    2+\sqrt{7} \\ -3-\sqrt{7} \\ 1
  \end{pmatrix}$.

While $\lambda = -\sqrt{7} - 1$, $A - \lambda I = \begin{pmatrix}
    2+\sqrt{7} & 0          & 3        \\
    0          & 3+\sqrt{7} & 2        \\
    3          & 2          & \sqrt{7}
  \end{pmatrix} \to \begin{pmatrix}
    1 & 0 & \sqrt{7}-2   \\
    0 & 1 & 3 - \sqrt{7} \\
    0 & 0 & 0
  \end{pmatrix}$, a specific eigenvector is $x_3 = \begin{pmatrix}
    2-\sqrt{7} \\ \sqrt{7} - 3\\ 1
  \end{pmatrix}$.

Therefore, the $D = \begin{pmatrix}
    4 & 0          & 0           \\
    0 & \sqrt{7}-1 & 0           \\
    0 & 0          & -\sqrt{7}-1
  \end{pmatrix}$, $P = \begin{pmatrix}
    1 & 2+\sqrt{7}  & 2-\sqrt{7} \\
    1 & -3-\sqrt{7} & \sqrt{7}-3 \\
    1 & 1           & 1
  \end{pmatrix}$.

Then calculate the inverse matrix of $P$:
\begin{align*}
  P ^{-1} & = \frac{1}{\det P} \begin{pmatrix}
                                 P_{11} & P_{21} & P_{31} \\
                                 P_{12} & P_{22} & P_{32} \\
                                 P_{13} & P_{23} & P_{33}
                               \end{pmatrix}                                                                                                                             \\
  \det P  & = -6 \sqrt{7}                                                                                                                                                           \\
  P_{11}  & = \begin{vmatrix}
                -3-\sqrt{7} & \sqrt{7}-3 \\
                1           & 1
              \end{vmatrix} = -2 \sqrt{7}                                                                                                                                           \\
  P_{12}  & = -\begin{vmatrix}
                 1 & \sqrt{7} -3 \\
                 1 & 1
               \end{vmatrix} = \sqrt{7} - 4                                                                                                                                         \\
  P_{13}  & = \begin{vmatrix}
                1 & -3-\sqrt{7} \\
                1 & 1
              \end{vmatrix} = 4+\sqrt{7}                                                                                                                                            \\
  P_{21}  & = - \begin{vmatrix}
                  2+\sqrt{7} & 2-\sqrt{7} \\
                  1          & 1
                \end{vmatrix} = -2 \sqrt{7}                                                                                                                                         \\
  P_{22}  & = \begin{vmatrix}
                1 & 2-\sqrt{7} \\
                1 & 1
              \end{vmatrix} = \sqrt{7} - 1                                                                                                                                          \\
  P_{23}  & = -\begin{vmatrix}
                 1 & 2+\sqrt{7} \\
                 1 & 1
               \end{vmatrix} = 1 + \sqrt{7}                                                                                                                                         \\
  P_{31}  & = \begin{vmatrix}
                2+\sqrt{7}  & 2-\sqrt{7} \\
                -3-\sqrt{7} & \sqrt{7}-3
              \end{vmatrix} = -2 \sqrt{7}                                                                                                                                           \\
  P_{32}  & = -\begin{vmatrix}
                 1 & 2-\sqrt{7} \\
                 1 & \sqrt{7}-3
               \end{vmatrix} = 5-2 \sqrt{7}                                                                                                                                         \\
  P_{33}  & = \begin{vmatrix}
                1 & 2+\sqrt{7}  \\
                1 & -3-\sqrt{7}
              \end{vmatrix} = -5-2 \sqrt{7}                                                                                                                                         \\
  P^{-1}  & = -\frac{1}{6 \sqrt{7}} \begin{pmatrix}
                                      -2 \sqrt{7} & -2 \sqrt{7}  & -2 \sqrt{7}   \\
                                      \sqrt{7}-4  & \sqrt{7} - 1 & 5-2 \sqrt{7}  \\
                                      4+\sqrt{7}  & 1 + \sqrt{7} & -5-2 \sqrt{7}
                                    \end{pmatrix} = \begin{pmatrix}
                                                      \frac{1}{3}               & \frac{1}{3}              & \frac{1}{3}                 \\
                                                      \frac{-7+4 \sqrt{7}}{42}  & \frac{-7+\sqrt{7}}{42}   & \frac{-5 \sqrt{7} + 14}{42} \\
                                                      \frac{-4 \sqrt{7} -7}{42} & \frac{-\sqrt{7} - 7}{42} & \frac{5 \sqrt{7} + 14}{42}
                                                    \end{pmatrix}
\end{align*}

Then $A = PDP ^{-1}$.

\subsection*{Problem 5}

Solution:

$Q(x) = ax_{1}^{2} + (ab+1)x_{1}x_{2}+bx_{2}^{2} = x'Ax$. Then \[
  A = \begin{pmatrix}
    a              & \frac{ab+1}{2} \\
    \frac{ab+1}{2} & b
  \end{pmatrix}
\]

$A$ is a symmetric matrix, so we can use eigenvalues to determine its definiteness.

\[
  \det(A-\lambda I) = \begin{vmatrix}
    a-\lambda      & \frac{ab+1}{2} \\
    \frac{ab+1}{2} & b-\lambda
  \end{vmatrix} = (a-\lambda)(b-\lambda)- \frac{(ab+1)^{2}}{4} = \lambda ^{2} - (a+b)\lambda-\frac{(ab-1)^{2}}{4} = 0
\]
Then $\left\{\begin{array}{l}
    \lambda_1 + \lambda_2 = a+b \\
    \lambda_1 \lambda_2 = -\frac{(ab-1)^{2}}{4}
  \end{array}\right.$

If $ab \neq 1$, then $\lambda_1\lambda_2 < 0$, matrix $A$ is indefinite.

If $ab = 1, a >0, b > 0 \Rightarrow \lambda_1 \geq 0, \lambda_2 \geq 0$, matrix $A$ is positive semi-definite.

If $ab = 1, a<0, b< 0 \Rightarrow \lambda_1 \leq 0, \lambda_2 \leq 0$, matrix $A$ is negative semi-definite.


\subsection*{Problem 6}

Solution:

a)

Let $L = \begin{pmatrix}
    a & 0 & 0 \\
    b & c & 0 \\
    d & e & f
  \end{pmatrix}$, then \[
  LL' = \begin{pmatrix}
    a & 0 & 0 \\
    b & c & 0 \\
    d & e & f
  \end{pmatrix} \begin{pmatrix}
    a & b & d \\
    0 & c & e \\
    0 & 0 & f
  \end{pmatrix} = \begin{pmatrix}
    a ^{2} & ab            & ad                   \\
    ab     & b ^{2}+c ^{2} & bd+ce                \\
    ad     & bd+ce         & d ^{2}+e ^{2}+f ^{2}
  \end{pmatrix}
\]

Since $LL'= \Sigma $, then
\[
  \left\{\begin{array}{l}
    a ^{2} = \sigma_{1}^{2}        \\
    ab = \rho\sigma_1\sigma_2      \\
    ad = \rho\sigma_1\sigma_3      \\
    b ^{2}+c ^{2} = \sigma_{2}^{2} \\
    bd+ce = \rho\sigma_2\sigma_3   \\
    d ^{2}+e ^{2} + f ^{2} = \sigma_{3}^{2}
  \end{array}\right. \Rightarrow \left\{\begin{array}{l}
    a = \sigma_1                                        \\
    b = \rho\sigma_2                                    \\
    c = \sqrt{1-\rho ^{2}}\sigma_2                      \\
    d = \rho\sigma_3                                    \\
    e = \frac{\rho\sigma_3(1-\rho)}{\sqrt{1-\rho ^{2}}} \\
    f = \sigma_3 \sqrt{1-\rho ^{2}-\rho ^{2}\frac{1-\rho}{1+\rho}}
  \end{array}\right.
\]

Then $L = \begin{pmatrix}
    \sigma_1     & 0                                               & 0                                                          \\
    \rho\sigma_2 & \sqrt{1-\rho ^{2}}\sigma_2                      & 0                                                          \\
    \rho\sigma_3 & \frac{\sigma_3\rho(1-\rho)}{\sqrt{1-\rho ^{2}}} & \sigma_3 \sqrt{1-\rho ^{2}-\rho ^{2}\frac{1-\rho}{1+\rho}}
  \end{pmatrix}$, we have $\Sigma = LL'$.

b)

Using leading minors:
\begin{align*}
  \det (M_1) & = \sigma_{1}^{2} > 0                                                                                                                                         \\
  \det(M_2)  & = \sigma_{1}^{2}\sigma_{2}^{2}-\rho ^{2}\sigma_{1}^{2}\sigma_{2}^{2} = (1-\rho ^{2})\sigma_1 ^{2}\sigma_2 ^{2} > 0  \Rightarrow  -1 < \rho < 1               \\
  \det (M_3) & = \sigma_{1}^{2}\sigma_{2}^{2}\sigma_{3}^{2}(1+2\rho ^{3}-3\rho ^{2}) > 0 \Rightarrow (\rho-1)^{2}(2\rho+1) > 0 \Rightarrow \rho > -\frac{1}{2}, \rho \neq 1
\end{align*}

Therefore, if $-\frac{1}{2}< \rho < 1$ then $\Sigma$ is positive definite.


\subsection*{Problem 7}

Solution:

We have $A ^{T} = A = A ^{*}$ and eigenvalues function: $Ax=\lambda x , x ^{*} A ^{*}= \bar{\lambda} x ^{*}$.
\begin{align*}
  Ax                   & = \lambda x            \\
  x ^{*}Ax             & = \lambda x ^{*}x      \\
  x ^{*}A ^{*} x       & = \lambda x ^{*}x      \\
  x ^{*}A ^{*} x       & = \bar{\lambda}x ^{*}x \\
  \bar{\lambda}x ^{*}x & = \lambda x ^{*}x      \\
  \bar{\lambda}        & = \lambda
\end{align*}
So, $\lambda$ is a real number. Then all the eigenvalues are real.




% --- Document ends here ---

\end{document}

