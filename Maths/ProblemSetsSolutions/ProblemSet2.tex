% --- LaTeX Homework Template - S. Venkatraman ---

% --- Set document class and font size ---

\documentclass[letterpaper, 11pt]{article}

% --- Package imports ---

\usepackage{
  amsmath, amsthm, amssymb, mathtools, dsfont,	  % Math typesetting
  graphicx, wrapfig, subfig, float,                  % Figures and graphics formatting
  listings, color, inconsolata, pythonhighlight,     % Code formatting
  fancyhdr, sectsty, hyperref, enumerate, enumitem } % Headers/footers, section fonts, links, lists

% --- Page layout settings ---

% Set page margins
\usepackage[left=1.35in, right=1.35in, bottom=1in, top=1.1in, headsep=0.2in]{geometry}

% Anchor footnotes to the bottom of the page
\usepackage[bottom]{footmisc}

% Set line spacing
\renewcommand{\baselinestretch}{1}

% Set spacing between paragraphs
\setlength{\parskip}{1.5mm}

% Allow multi-line equations to break onto the next page
\allowdisplaybreaks

% Enumerated lists: make numbers flush left, with parentheses around them
\setlist[enumerate]{wide=0pt, leftmargin=21pt, labelwidth=0pt, align=left}
\setenumerate[1]{label={(\arabic*)}}

% --- Page formatting settings ---

% Set link colors for labeled items (blue) and citations (red)
\hypersetup{colorlinks=true, linkcolor=blue, citecolor=red}

% Make reference section title font smaller
\renewcommand{\refname}{\large\bf{References}}

% --- Settings for printing computer code ---

% Define colors for green text (comments), grey text (line numbers),
% and green frame around code
\definecolor{greenText}{rgb}{0.5, 0.7, 0.5}
\definecolor{greyText}{rgb}{0.5, 0.5, 0.5}
\definecolor{codeFrame}{rgb}{0.5, 0.7, 0.5}

% Define code settings
\lstdefinestyle{code} {
  frame=single, rulecolor=\color{codeFrame},            % Include a green frame around the code
  numbers=left,                                         % Include line numbers
  numbersep=8pt,                                        % Add space between line numbers and frame
  numberstyle=\tiny\color{greyText},                    % Line number font size (tiny) and color (grey)
  commentstyle=\color{greenText},                       % Put comments in green text
  basicstyle=\linespread{1.1}\ttfamily\footnotesize,    % Set code line spacing
  keywordstyle=\ttfamily\footnotesize,                  % No special formatting for keywords
  showstringspaces=false,                               % No marks for spaces
  xleftmargin=1.95em,                                   % Align code frame with main text
  framexleftmargin=1.6em,                               % Extend frame left margin to include line numbers
  breaklines=true,                                      % Wrap long lines of code
  postbreak=\mbox{\textcolor{greenText}{$\hookto$}\space} % Mark wrapped lines with an arrow
}

% Set all code listings to be styled with the above settings
\lstset{style=code}

% --- Math/Statistics commands ---

% Add a reference number to a single line of a multi-line equation
% Usage: "\numberthis\label{labelNameHere}" in an align or gather environment
\newcommand\numberthis{\addtocounter{equation}{1}\tag{\theequation}}

% Shortcut for bold text in math mode, e.g. $\b{X}$
\let\b\mathbf

% Shortcut for bold Greek letters, e.g. $\bg{\beta}$
\let\bg\boldsymbol

% Shortcut for calligraphic script, e.g. %\mc{M}$
\let\mc\mathcal

% \mathscr{(letter here)} is sometimes used to denote vector spaces
\usepackage[mathscr]{euscript}

% Convergence: right arrow with optional text on top
% E.g. $\converge[w]$ for weak convergence
\newcommand{\converge}[1][]{\xto{#1}}

% Normal distribution: arguments are the mean and variance
% E.g. $\normal{\mu}{\sigma}$
\newcommand{\normal}[2]{\mathcal{N}\left(#1,#2\right)}

% Uniform distribution: arguments are the left and right endpoints
% E.g. $\unif{0}{1}$
\newcommand{\unif}[2]{\text{Uniform}(#1,#2)}

% Independent and identically distributed random variables
% E.g. $ X_1,...,X_n \iid \normal{0}{1}$
\newcommand{\iid}{\stackrel{\smash{\text{iid}}}{\sim}}

% Equality: equals sign with optional text on top
% E.g. $X \equals[d] Y$ for equality in distribution
\newcommand{\equals}[1][]{\stackrel{\smash{#1}}{=}}

% Math mode symbols for common sets and spaces. Example usage: $\R$
\newcommand{\R}{\mathbb{R}}   % Real numbers
\newcommand{\C}{\mathbb{C}}   % Complex numbers
\newcommand{\Q}{\mathbb{Q}}   % Rational numbers
\newcommand{\Z}{\mathbb{Z}}   % Integers
\newcommand{\N}{\mathbb{N}}   % Natural numbers
\newcommand{\F}{\mathcal{F}}  % Calligraphic F for a sigma algebra
\newcommand{\El}{\mathcal{L}} % Calligraphic L, e.g. for L^p spaces

% Math mode symbols for probability
\newcommand{\pr}{\mathbb{P}}    % Probability measure
\newcommand{\E}{\mathbb{E}}     % Expectation, e.g. $\E(X)$
\newcommand{\var}{\text{Var}}   % Variance, e.g. $\var(X)$
\newcommand{\cov}{\text{Cov}}   % Covariance, e.g. $\cov(X,Y)$
\newcommand{\corr}{\text{Corr}} % Correlation, e.g. $\corr(X,Y)$
\newcommand{\B}{\mathcal{B}}    % Borel sigma-algebra

% Other miscellaneous symbols
\newcommand{\tth}{\text{th}}	% Non-italicized 'th', e.g. $n^\tth$
\newcommand{\Oh}{\mathcal{O}}	% Big-O notation, e.g. $\O(n)$
\newcommand{\1}{\mathds{1}}	% Indicator function, e.g. $\1_A$

% Additional commands for math mode
\DeclareMathOperator*{\argmax}{argmax}    % Argmax, e.g. $\argmax_{x\in[0,1]} f(x)$
\DeclareMathOperator*{\argmin}{argmin}    % Argmin, e.g. $\argmin_{x\in[0,1]} f(x)$
\DeclareMathOperator*{\spann}{Span}       % Span, e.g. $\spann\{X_1,...,X_n\}$
\DeclareMathOperator*{\bias}{Bias}        % Bias, e.g. $\bias(\hat\theta)$
\DeclareMathOperator*{\ran}{ran}          % Range of an operator, e.g. $\ran(T) 
\DeclareMathOperator*{\dv}{d\!}           % Non-italicized 'with respect to', e.g. $\int f(x) \dv x$
\DeclareMathOperator*{\diag}{diag}        % Diagonal of a matrix, e.g. $\diag(M)$
\DeclareMathOperator*{\trace}{trace}      % Trace of a matrix, e.g. $\trace(M)$

% Numbered theorem, lemma, etc. settings - e.g., a definition, lemma, and theorem appearing in that 
% order in Section 2 will be numbered Definition 2.1, Lemma 2.2, Theorem 2.3. 
% Example usage: \begin{theorem}[Name of theorem] Theorem statement \end{theorem}
\theoremstyle{definition}
\newtheorem{theorem}{Theorem}[section]
\newtheorem{proposition}[theorem]{Proposition}
\newtheorem{lemma}[theorem]{Lemma}
\newtheorem{corollary}[theorem]{Corollary}
\newtheorem{definition}[theorem]{Definition}
\newtheorem{example}[theorem]{Example}
\newtheorem{remark}[theorem]{Remark}

% Un-numbered theorem, lemma, etc. settings
% Example usage: \begin{lemma*}[Name of lemma] Lemma statement \end{lemma*}
\newtheorem*{theorem*}{Theorem}
\newtheorem*{proposition*}{Proposition}
\newtheorem*{lemma*}{Lemma}
\newtheorem*{corollary*}{Corollary}
\newtheorem*{definition*}{Definition}
\newtheorem*{example*}{Example}
\newtheorem*{remark*}{Remark}
\newtheorem*{claim}{Claim}

% --- Left/right header text (to appear on every page) ---

% Include a line underneath the header, no footer line
\pagestyle{fancy}
\renewcommand{\footrulewidth}{0pt}
\renewcommand{\headrulewidth}{0.4pt}

% Left header text: course name/assignment number
\lhead{MATHEMATICS - Problem Set 2}

% Right header text: your name
\rhead{Zian Gong}

% --- Document starts here ---

\begin{document}
\subsection*{Problem 1}

Solution:

a)

\begin{align*}
    \det A            & = 4- 5 = -1                                    \\
    A'                & = \begin{pmatrix}
                              4 & 1 \\
                              5 & 1
                          \end{pmatrix}                               \\
    \det A'           & = 4 - 5 = -1                                   \\
    \det B            & = 3 - 4 = -1                                   \\
    AB                & = \begin{pmatrix}
                              4 & 5 \\
                              1 & 1
                          \end{pmatrix} \begin{pmatrix}
                                            3 & 4 \\
                                            1 & 1
                                        \end{pmatrix} = \begin{pmatrix}
                                                            17 & 21 \\
                                                            4  & 5
                                                        \end{pmatrix} \\
    \det (AB)         & = 85 - 84 = 1                                  \\
    \det (A) \det (B) & = -1 \times -1 = 1                             \\
    A + B             & = \begin{pmatrix}
                              7 & 9 \\
                              2 & 2
                          \end{pmatrix}                               \\
    \det (A+B)        & = -4                                           \\
    \det (A+B)        & \neq \det (A) + \det (B)
\end{align*}


b)

\begin{align*}
    \det A            & = 24                     \\
    A'                & = \begin{pmatrix}
                              1 & 0 & 0 \\
                              2 & 4 & 0 \\
                              3 & 5 & 6
                          \end{pmatrix}         \\
    \det A'           & = 24                     \\
    \det B            & = 18                     \\
    AB                & = \begin{pmatrix}
                              17 & 21 & 18 \\
                              28 & 37 & 30 \\
                              24 & 30 & 36
                          \end{pmatrix}         \\
    \det (AB)         & = 432                    \\
    \det (A) \det (B) & = 24 \times 18 = 432     \\
    A + B             & =  \begin{pmatrix}
                               2 & 2 & 3  \\
                               2 & 7 & 5  \\
                               4 & 5 & 12
                           \end{pmatrix}        \\
    \det (A+B)        & = 56                     \\
    \det (A+B)        & \neq \det (A) + \det (B)
\end{align*}

c)

\begin{align*}
    \det A            & = ad-bc                                       \\
    A'                & = \begin{pmatrix}
                              a & c \\
                              b & d
                          \end{pmatrix}                              \\
    \det A'           & = ad-bc                                       \\
    \det B            & = eh-fg                                       \\
    AB                & = \begin{pmatrix}
                              a & b \\
                              c & d
                          \end{pmatrix}\begin{pmatrix}
                                           e & f \\
                                           g & h
                                       \end{pmatrix} = \begin{pmatrix}
                                                           ae+bg & af+bh \\
                                                           ce+dg & cf+dh
                                                       \end{pmatrix} \\
    \det (AB)         & = (ae+bg)(cf+dh)- (af+bh)(ce+dg)              \\
                      & = bcfg + adeh - adfg - bceh                   \\
    \det (A) \det (B) & = (ad-bc)(eh-fg) = adeh - bceh - adfg + bcfg  \\
    A + B             & = \begin{pmatrix}
                              a+e & b+f \\
                              c+g & d+h
                          \end{pmatrix}                              \\
    \det (A+B)        & = (a+e)(d+h) - (b+f)(c+g)                     \\
    \det (A+B)        & \neq \det (A) + \det (B)
\end{align*}

\subsection*{Problem 2}

\begin{align*}
    D(x) & = \begin{vmatrix}
                 b & b & b & b \\
                 b & a & a & a \\
                 b & a & b & b \\
                 b & a & b & c
             \end{vmatrix} \\ & =\begin{vmatrix}
        b & 0    & 0   & 0   \\
        b & a -b & a-b & a-b \\
        b & a-b  & 0   & 0   \\
        b & a-b  & 0   & c-b
    \end{vmatrix} \\& =\begin{vmatrix}
        b & 0   & 0   & 0   \\
        b & 0   & a-b & 0   \\
        b & a-b & 0   & 0   \\
        b & a-c & 0   & c-b
    \end{vmatrix} \\& = - \begin{vmatrix}
        b & 0    & 0   & 0   \\
        b & a -b & 0   & 0   \\
        b & 0    & a-b & 0   \\
        b & 0    & a-c & c-b
    \end{vmatrix} = -b(a-b)^{2}(c-b)
\end{align*}

\subsection*{Problem 3}

% TODO:?

a)

\subsection*{Problem 4}

Rewrite this linear system as $Ax=b$, \begin{align*}
    \begin{bmatrix}
        1 & 1 & 1   \\
        a & 1 & 1   \\
        2 & 2 & a-1
    \end{bmatrix} \begin{bmatrix}
                      x \\ y \\ z
                  \end{bmatrix} = \begin{bmatrix}
                                      0 \\ b \\ 0
                                  \end{bmatrix}
\end{align*}

Then the augmented matrix $[A|b]$ is \begin{align*}
    \left[\begin{array}{ccc|c}
                  1 & 1 & 1   & 0 \\
                  a & 1 & 1   & b \\
                  2 & 2 & a-1 & 0
              \end{array}\right] \to \left[\begin{array}{ccc|c}
                                               1 & 1 & 1   & 0 \\
                                               a & 1 & 1   & b \\
                                               0 & 0 & a-3 & 0
                                           \end{array}\right] \to \left[\begin{array}{ccc|c}
                                                                            1    & 1 & 1   & 0 \\
                                                                            a -1 & 0 & 0   & b \\
                                                                            0    & 0 & a-3 & 0
                                                                        \end{array}\right]
\end{align*}

If $a = 3$, $rank(A)=rank(A|b) = 2< 3$, the linear system has infinite solutions.

If $a = 1$ and $b = 0$, $rank(A)=rank(A|b) =1 < 3$, the linear system has infinite solutions.

If $a = 1$ and $b \neq 0$, $rank(A) < rank(A|b)$, the linear system has no solution.

If $a \neq 1$ and $a \neq 3$,  $rank(A)=rank(A|b) = 3$, the linear system has one solution.


\subsection*{Problem 5}

Solution:

The augmented matrix can be written as following:
\begin{align*}
    \left[\begin{array}{ccc|c}
                  -1 & 3  & -6 & 1 \\
                  2  & -5 & 10 & 0 \\
                  3  & -8 & 17 & 0
              \end{array}\right] \to  \left[\begin{array}{ccc|c}
                                                -1 & 3 & 6  & 1 \\
                                                0  & 1 & -2 & 2 \\
                                                0  & 1 & -1 & 3
                                            \end{array}\right] \to \left[\begin{array}{ccc|c}
                                                                             -1 & 3 & 6  & 1 \\
                                                                             0  & 1 & -2 & 2 \\
                                                                             0  & 0 & 1  & 1
                                                                         \end{array}\right]
\end{align*}

,then we have \begin{equation*}
    \left\{\begin{array}{l}
        x = 5 \\
        y = 4 \\
        z = 1 \\
    \end{array}\right.
\end{equation*}


\subsection*{Problem 6}

Solution:

a)

\begin{align*}
    Ax                   & =\lambda x                     \\
    (A-\lambda I) x      & = 0                            \\
    \det (A - \lambda I) & =\begin{vmatrix}
                                1- \lambda & 2         \\
                                5          & 4-\lambda
                            \end{vmatrix}        \\
                         & = \lambda ^{2} - 5 \lambda - 6 \\
                         & = (\lambda-6)(\lambda + 1)
\end{align*}

We have eigenvalues: \begin{equation*}
    \left\{\begin{array}{l}
        \lambda_{1} = 6  \\
        \lambda_{2} = -1 \\
    \end{array}\right.
\end{equation*}

When $\lambda=6$, the augmented matrix of the linear equations are:
\begin{align*}
     & \left[\begin{array}{cc|c}
                     -5 & 2  & 0 \\
                     5  & -2 & 0
                 \end{array}\right] \to \left[\begin{array}{cc|c}
                                                  -5 & 2 & 0 \\
                                                  0  & 0 & 0
                                              \end{array}\right]                   \\
     & x = (2k , 5k)'\quad \text{one unit vector solution is} \quad x = (2,5)'
\end{align*}

When $\lambda=-1$, the augmented matrix of the linear equations are:
\begin{align*}
     & \left[\begin{array}{cc|c}
                     2 & 2 & 0 \\
                     5 & 5 & 0
                 \end{array}\right] \to \left[\begin{array}{cc|c}
                                                  1 & 1 & 0 \\
                                                  0 & 0 & 0
                                              \end{array}\right]                   \\
     & x = (k , -k)'\quad \text{one unit vector solution is} \quad x = (1,-1)'
\end{align*}

Therefore, we have \begin{align*}
    P         & = \begin{pmatrix}
                      2 & 1  \\
                      5 & -1
                  \end{pmatrix}             \\
    P'        & = \begin{pmatrix}
                      \frac{1}{7} & \frac{1}{7}  \\
                      \frac{5}{7} & -\frac{2}{7}
                  \end{pmatrix} \\
    D         & = \begin{pmatrix}
                      6 & 0  \\
                      0 & -1
                  \end{pmatrix}             \\
    PDP ^{-1} & =  A
\end{align*}


b)

\begin{align*}
    A                  & = \begin{pmatrix}
                               2  & 1 \\
                               -1 & 0
                           \end{pmatrix}                                      \\
    \det (A-\lambda I) & = \begin{vmatrix}
                               2-\lambda & 1        \\
                               -1        & -\lambda
                           \end{vmatrix}= \lambda ^{2}-2\lambda+1              \\
    \lambda            & = \left\{\begin{array}{l}
                                      \lambda_{1} = 1 \\
                                      \lambda_{2} = 1 \\
                                  \end{array}\right.                           \\
                       & \left(\begin{array}{cc|c}
                                       1  & 1  & 0 \\
                                       -1 & -1 & 0
                                   \end{array}\right) \to \left(\begin{array}{cc|c}
                                                                    1 & 1 & 0 \\
                                                                    0 & 0 & 0
                                                                \end{array}\right)
\end{align*}

Matrix $A$ has only one set of linear independent eigenvectors. So it cannot be diagonalized.

c)

\begin{align*}
    A                    & = \begin{pmatrix}
                                 0 & -1 \\
                                 2 & 2
                             \end{pmatrix}                            \\
    \det (A - \lambda I) & = \begin{vmatrix}
                                 -\lambda & -1        \\
                                 2        & 2-\lambda
                             \end{vmatrix} = \lambda ^{2} - 2\lambda+2 \\
    \lambda              & = \left\{\begin{array}{l}
                                        \lambda_{1} = 1+i \\
                                        \lambda_{2} = 1-i \\
                                    \end{array}\right.
\end{align*}

When $\lambda = 1+i$, the augmented matrix of the linear equations are: \begin{align*}
     & \left[\begin{array}{cc|c}
                     -1-i & -1  & 0 \\
                     2    & 1-i & 0
                 \end{array}\right] \to \left[\begin{array}{cc|c}
                                                  2 & 1-i & 0 \\
                                                  0 & 0   & 0
                                              \end{array}\right] \\
     & x=(i-1,2)
\end{align*}

When $\lambda = 1-i$, the augmented matrix of the linear equations are: \begin{align*}
     & \left[\begin{array}{cc|c}
                     -1+i & -1  & 0 \\
                     2    & 1+i & 0
                 \end{array}\right] \to \left[\begin{array}{cc|c}
                                                  2 & 1+i & 0 \\
                                                  0 & 0   & 0
                                              \end{array}\right] \\
     & x=(i+1,-2)
\end{align*}

Then \begin{align*}
    P         & = \begin{pmatrix}
                      i-1 & i+1 \\
                      2   & -2
                  \end{pmatrix}                 \\
    D         & = \begin{pmatrix}
                      1+i & 0   \\
                      0   & 1-i
                  \end{pmatrix}                 \\
    P ^{-1}   & = \begin{pmatrix}
                      \frac{1}{2i}  & \frac{i+1}{4i} \\
                      -\frac{1}{2i} & \frac{1-i}{4i}
                  \end{pmatrix} \\
    PDP ^{-1} & =  A
\end{align*}

d)

\begin{align*}
    \det (A-\lambda I) & = \begin{vmatrix}
                               -\lambda & 0         & 1        \\
                               0        & 1-\lambda & 0        \\
                               1        & 0         & -\lambda
                           \end{vmatrix} = -(\lambda-1)^{2}(\lambda+1) \\
    \lambda            & = \left\{\begin{array}{l}
                                      \lambda_{1} = -1  \\
                                      \lambda_{2,3} = 1 \\
                                  \end{array}\right.
\end{align*}

When $\lambda = -1$, the augmented matrix of the linear equations are: \begin{align*}
     & \left(\begin{array}{ccc|c}
                     1 & 0 & 1 & 0 \\
                     0 & 2 & 0 & 0 \\
                     1 & 0 & 1 & 0
                 \end{array}\right) \to \left(\begin{array}{ccc|c}
                                                  1 & 0 & 1 & 0 \\
                                                  0 & 2 & 0 & 0 \\
                                                  0 & 0 & 0 & 0
                                              \end{array}\right) \\
     & x_{1} = (1,0,-1)'
\end{align*}

When $\lambda = 1$, the augmented matrix of the linear equations are: \begin{align*}
     & \left(\begin{array}{ccc|c}
                     -1 & 0 & 1  & 0 \\
                     0  & 0 & 0  & 0 \\
                     1  & 0 & -1 & 0
                 \end{array}\right) \to \left(\begin{array}{ccc|c}
                                                  -1 & 0 & 1 & 0 \\
                                                  0  & 0 & 0 & 0 \\
                                                  0  & 0 & 0 & 0
                                              \end{array}\right) \\
     & x_{2} = (1,0,1)'                                       \\
     & x_{3} = (0,1,0)'                                       \\
\end{align*}

Then \begin{align*}
    P         & = \begin{pmatrix}
                      1  & 1 & 0 \\
                      0  & 0 & 1 \\
                      -1 & 1 & 0
                  \end{pmatrix}                 \\
    D         & = \begin{pmatrix}
                      -1 & 0 & 0 \\
                      0  & 1 & 0 \\
                      0  & 0 & 1
                  \end{pmatrix}                 \\
    P ^{-1}   & = \begin{pmatrix}
                      \frac{1}{2} & 0 & -\frac{1}{2} \\
                      \frac{1}{2} & 0 & \frac{1}{2}  \\
                      0           & 1 & 0
                  \end{pmatrix} \\
    PDP ^{-1} & =  A
\end{align*}

e)

\begin{align*}
    \det (A-\lambda I) & = \begin{vmatrix}
                               1-\lambda & 0          & 1         \\
                               0         & -1-\lambda & 0         \\
                               0         & 0          & 2-\lambda
                           \end{vmatrix} \\
    \lambda            & = \left\{\begin{array}{l}
                                      \lambda_{1} = 1  \\
                                      \lambda_{2} = -1 \\
                                      \lambda_{3} = 2
                                  \end{array}\right.
\end{align*}

When $\lambda = 1$, the augmented matrix of the linear equations is \begin{align*}
     & \left(\begin{array}{ccc|c}
                     0 & 0  & 1 & 0 \\
                     0 & -2 & 0 & 0 \\
                     0 & 0  & 1 & 0
                 \end{array}\right) \\
     & x_{1} = (1,0,0)'
\end{align*}

When $\lambda=-1$, the augmented matrix of the linear equations is \begin{align*}
     & \left(\begin{array}{ccc|c}
                     2 & 0 & 1 & 0 \\
                     0 & 0 & 0 & 0 \\
                     0 & 0 & 3 & 0
                 \end{array}\right) \\
     & x_{2} = (0,1,0)'
\end{align*}

When $\lambda=2$, the augmented matrix of the linear equations is \begin{align*}
     & \left(\begin{array}{ccc|c}
                     -1 & 0  & 1 & 0 \\
                     0  & -3 & 0 & 0 \\
                     0  & 0  & 0 & 0
                 \end{array}\right) \\
     & x_{3} = (1,0,1)'
\end{align*}

Then \begin{align*}
    P = \begin{pmatrix}
            1 & 0 & 1 \\
            0 & 1 & 0 \\
            0 & 0 & 1
        \end{pmatrix} \\
    D = \begin{pmatrix}
            1 & 0  & 0 \\
            0 & -1 & 0 \\
            0 & 0  & 2
        \end{pmatrix} \\
    P ^{-1} = \begin{pmatrix}
                  1 & 0 & -1 \\
                  0 & 1 & 0  \\
                  0 & 0 & 1
              \end{pmatrix}
\end{align*}

\subsection*{Problem 7}
% TODO:
Solution:

\subsection*{Problem 8}

Solution:

a)

We have $A ^{n} = A, n \in \mathbb{N}$, then $A ^{3} = A$.

\begin{align*}
    (I_{n}-A ^{3})^{n} = (I_{n}-A)^{n} \\
\end{align*}

If $n = 2$, \begin{align*}
    (I_{n}-A ^{3})^{2} & = (I_{n}-A)^{2}              \\
                       & = I_{n}^{2}-2I_{n}A + A ^{2} \\
                       & = I_{n} - 2A+A               \\
                       & = I_{n}-A = I_{n}-A ^{3}
\end{align*}

If $n >=3$, \begin{align*}
    (I_{n}-A ^{3})^{n} & = (I_{n}-A)^{n}                                    \\
                       & = (I_{n}-A)^{2}(I_{n}-A)^{n-2}                     \\
                       & = (I_{n}-A)(I_{n}-A)^{n-2}                         \\
                       & = (I_{n}-A)^{n-1} = \dots = I_{n}-A = I_{n}-A ^{3}
\end{align*}

Q.E.D.

b)

% TODO:


\subsection*{Problem 9}
%TODO:

\subsection*{Problem 10}

Solution:

a)

$n=2$, let $A = \begin{pmatrix}
        a_{11} & a_{12} \\
        a_{21} & a_{22}
    \end{pmatrix}$, then \begin{align*}
    \det A             & = a_{11}a_{22} - a_{12}a_{21}            \\
    tr(A)              & = a_{11} + a_{22}                        \\
    \det (A-\lambda I) & = \begin{vmatrix}
                               a_{11} -\lambda & a_{12}          \\
                               a_{21}          & a_{22} -\lambda
                           \end{vmatrix} = \det A - \lambda tr(A)
\end{align*}

Similarly, $\det (B - \lambda I) = \det B - \lambda tr(B)$. Since $\det A = \det B, tr(A) = tr(B)$, then $\det (A-\lambda I)  = \det (B - \lambda I)$

Example when $n = 3$:

% TODO:

b)

Since $\det A = \prod_{i=0}^{n}\lambda_i$.

If $\exists \lambda_{j} = 0$, then $\det A = 0$.

If $\det A \neq 0$ then $\forall \lambda_{i} \neq =0$


\subsection*{Problem 11}

Solution:

a)

\begin{align*}
    Q_{2}(x) & =x'A_{2}x         \\
             & =x'\begin{pmatrix}
                      4 & 2 \\
                      2 & 1
                  \end{pmatrix}x \\
    A_{2}    & = \begin{pmatrix}
                     4 & 2 \\
                     2 & 1
                 \end{pmatrix}
\end{align*}

$A_{2} $ has two eigenvalues, $\lambda_{1} = 0, \lambda_{2}=5$. Therefore, $Q_{2}$ is semi-positive definite.

\begin{align*}
    Q_{1}(x) & =x'A_{1}x          \\
             & = x'\begin{pmatrix}
                       0   & 1/2 \\
                       1/2 & 0
                   \end{pmatrix}x \\
    A_{1}    & =\begin{pmatrix}
                    0   & 1/2 \\
                    1/2 & 0
                \end{pmatrix}
\end{align*}

$A_{1}$ has two eigenvalues, $\lambda_{1}=\lambda_{2}=\frac{1}{2}$. Therefore, $Q_{1}$ is positive definite.


b)

Assume $L = \begin{pmatrix}
        a & 0 \\
        b & c
    \end{pmatrix}$.
\begin{align*}
     & LL' = \begin{pmatrix}
                 a & 0 \\
                 b & c
             \end{pmatrix}\begin{pmatrix}
                              a & b \\
                              0 & c
                          \end{pmatrix} = \begin{pmatrix}
                                              a ^{2} & ab            \\
                                              ab     & b ^{2}+c ^{2}
                                          \end{pmatrix}        \\
     & \left\{\begin{array}{l}
                  a ^{2} = \sigma_{1}^{2}       \\
                  ab = \rho\sigma_{1}\sigma_{2} \\
                  b ^{2}+ c ^{2} = \sigma_{2}^{2}
              \end{array}\right.                            \\
     & \left\{\begin{array}{l}
                  a = \sigma_{1}     \\
                  b = \rho\sigma_{2} \\
                  c =\sqrt{(1-\rho ^{2})\sigma_{2}^{2}}
              \end{array}\right.
\end{align*}

Then $L = \begin{pmatrix}
        \sigma_{1}     & 0                                  \\
        \rho\sigma_{2} & \sqrt{(1-\rho ^{2})\sigma_{2}^{2}}
    \end{pmatrix}$.

% --- Document ends here ---

\end{document}

