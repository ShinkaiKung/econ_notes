% --- LaTeX Homework Template - S. Venkatraman ---

% --- Set document class and font size ---

\documentclass[letterpaper, 11pt]{article}

% --- Package imports ---

\usepackage{
  amsmath, amsthm, amssymb, mathtools, dsfont,	  % Math typesetting
  graphicx, wrapfig, subfig, float,                  % Figures and graphics formatting
  listings, color, inconsolata, pythonhighlight,     % Code formatting
  fancyhdr, sectsty, hyperref, enumerate, enumitem } % Headers/footers, section fonts, links, lists

% --- Page layout settings ---

% Set page margins
\usepackage[left=1.35in, right=1.35in, bottom=1in, top=1.1in, headsep=0.2in]{geometry}

% Anchor footnotes to the bottom of the page
\usepackage[bottom]{footmisc}

% Set line spacing
\renewcommand{\baselinestretch}{1}

% Set spacing between paragraphs
\setlength{\parskip}{1.5mm}

% Allow multi-line equations to break onto the next page
\allowdisplaybreaks

% Enumerated lists: make numbers flush left, with parentheses around them
\setlist[enumerate]{wide=0pt, leftmargin=21pt, labelwidth=0pt, align=left}
\setenumerate[1]{label={(\arabic*)}}

% --- Page formatting settings ---

% Set link colors for labeled items (blue) and citations (red)
\hypersetup{colorlinks=true, linkcolor=blue, citecolor=red}

% Make reference section title font smaller
\renewcommand{\refname}{\large\bf{References}}

% --- Settings for printing computer code ---

% Define colors for green text (comments), grey text (line numbers),
% and green frame around code
\definecolor{greenText}{rgb}{0.5, 0.7, 0.5}
\definecolor{greyText}{rgb}{0.5, 0.5, 0.5}
\definecolor{codeFrame}{rgb}{0.5, 0.7, 0.5}

% Define code settings
\lstdefinestyle{code} {
  frame=single, rulecolor=\color{codeFrame},            % Include a green frame around the code
  numbers=left,                                         % Include line numbers
  numbersep=8pt,                                        % Add space between line numbers and frame
  numberstyle=\tiny\color{greyText},                    % Line number font size (tiny) and color (grey)
  commentstyle=\color{greenText},                       % Put comments in green text
  basicstyle=\linespread{1.1}\ttfamily\footnotesize,    % Set code line spacing
  keywordstyle=\ttfamily\footnotesize,                  % No special formatting for keywords
  showstringspaces=false,                               % No marks for spaces
  xleftmargin=1.95em,                                   % Align code frame with main text
  framexleftmargin=1.6em,                               % Extend frame left margin to include line numbers
  breaklines=true,                                      % Wrap long lines of code
  postbreak=\mbox{\textcolor{greenText}{$\hookto$}\space} % Mark wrapped lines with an arrow
}

% Set all code listings to be styled with the above settings
\lstset{style=code}

% --- Math/Statistics commands ---

% Add a reference number to a single line of a multi-line equation
% Usage: "\numberthis\label{labelNameHere}" in an align or gather environment
\newcommand\numberthis{\addtocounter{equation}{1}\tag{\theequation}}

% Shortcut for bold text in math mode, e.g. $\b{X}$
\let\b\mathbf

% Shortcut for bold Greek letters, e.g. $\bg{\beta}$
\let\bg\boldsymbol

% Shortcut for calligraphic script, e.g. %\mc{M}$
\let\mc\mathcal

% \mathscr{(letter here)} is sometimes used to denote vector spaces
\usepackage[mathscr]{euscript}

% Convergence: right arrow with optional text on top
% E.g. $\converge[w]$ for weak convergence
\newcommand{\converge}[1][]{\xto{#1}}

% Normal distribution: arguments are the mean and variance
% E.g. $\normal{\mu}{\sigma}$
\newcommand{\normal}[2]{\mathcal{N}\left(#1,#2\right)}

% Uniform distribution: arguments are the left and right endpoints
% E.g. $\unif{0}{1}$
\newcommand{\unif}[2]{\text{Uniform}(#1,#2)}

% Independent and identically distributed random variables
% E.g. $ X_1,...,X_n \iid \normal{0}{1}$
\newcommand{\iid}{\stackrel{\smash{\text{iid}}}{\sim}}

% Equality: equals sign with optional text on top
% E.g. $X \equals[d] Y$ for equality in distribution
\newcommand{\equals}[1][]{\stackrel{\smash{#1}}{=}}

% Math mode symbols for common sets and spaces. Example usage: $\R$
\newcommand{\R}{\mathbb{R}}   % Real numbers
\newcommand{\C}{\mathbb{C}}   % Complex numbers
\newcommand{\Q}{\mathbb{Q}}   % Rational numbers
\newcommand{\Z}{\mathbb{Z}}   % Integers
\newcommand{\N}{\mathbb{N}}   % Natural numbers
\newcommand{\F}{\mathcal{F}}  % Calligraphic F for a sigma algebra
\newcommand{\El}{\mathcal{L}} % Calligraphic L, e.g. for L^p spaces

% Math mode symbols for probability
\newcommand{\pr}{\mathbb{P}}    % Probability measure
\newcommand{\E}{\mathbb{E}}     % Expectation, e.g. $\E(X)$
\newcommand{\var}{\text{Var}}   % Variance, e.g. $\var(X)$
\newcommand{\cov}{\text{Cov}}   % Covariance, e.g. $\cov(X,Y)$
\newcommand{\corr}{\text{Corr}} % Correlation, e.g. $\corr(X,Y)$
\newcommand{\B}{\mathcal{B}}    % Borel sigma-algebra

% Other miscellaneous symbols
\newcommand{\tth}{\text{th}}	% Non-italicized 'th', e.g. $n^\tth$
\newcommand{\Oh}{\mathcal{O}}	% Big-O notation, e.g. $\O(n)$
\newcommand{\1}{\mathds{1}}	% Indicator function, e.g. $\1_A$

% Additional commands for math mode
\DeclareMathOperator*{\argmax}{argmax}    % Argmax, e.g. $\argmax_{x\in[0,1]} f(x)$
\DeclareMathOperator*{\argmin}{argmin}    % Argmin, e.g. $\argmin_{x\in[0,1]} f(x)$
\DeclareMathOperator*{\spann}{Span}       % Span, e.g. $\spann\{X_1,...,X_n\}$
\DeclareMathOperator*{\bias}{Bias}        % Bias, e.g. $\bias(\hat\theta)$
\DeclareMathOperator*{\ran}{ran}          % Range of an operator, e.g. $\ran(T) 
\DeclareMathOperator*{\dv}{d\!}           % Non-italicized 'with respect to', e.g. $\int f(x) \dv x$
\DeclareMathOperator*{\diag}{diag}        % Diagonal of a matrix, e.g. $\diag(M)$
\DeclareMathOperator*{\trace}{trace}      % Trace of a matrix, e.g. $\trace(M)$

% Numbered theorem, lemma, etc. settings - e.g., a definition, lemma, and theorem appearing in that 
% order in Section 2 will be numbered Definition 2.1, Lemma 2.2, Theorem 2.3. 
% Example usage: \begin{theorem}[Name of theorem] Theorem statement \end{theorem}
\theoremstyle{definition}
\newtheorem{theorem}{Theorem}[section]
\newtheorem{proposition}[theorem]{Proposition}
\newtheorem{lemma}[theorem]{Lemma}
\newtheorem{corollary}[theorem]{Corollary}
\newtheorem{definition}[theorem]{Definition}
\newtheorem{example}[theorem]{Example}
\newtheorem{remark}[theorem]{Remark}

% Un-numbered theorem, lemma, etc. settings
% Example usage: \begin{lemma*}[Name of lemma] Lemma statement \end{lemma*}
\newtheorem*{theorem*}{Theorem}
\newtheorem*{proposition*}{Proposition}
\newtheorem*{lemma*}{Lemma}
\newtheorem*{corollary*}{Corollary}
\newtheorem*{definition*}{Definition}
\newtheorem*{example*}{Example}
\newtheorem*{remark*}{Remark}
\newtheorem*{claim}{Claim}

% --- Left/right header text (to appear on every page) ---

% Include a line underneath the header, no footer line
\pagestyle{fancy}
\renewcommand{\footrulewidth}{0pt}
\renewcommand{\headrulewidth}{0.4pt}

% Left header text: course name/assignment number
\lhead{MATHEMATICS - Problem Set 1}

% Right header text: your name
\rhead{Zian Gong}

% --- Document starts here ---

\begin{document}
\subsection*{Problem 1}

\subsubsection*{a)}
Solution:

i)
\begin{align*}
  1 - \ln|x| \geq 0 \\
  \ln|x| \leq 1     \\
\end{align*}

Therefore, $D = \{x \in \mathbb{R}: |x| \leq e, x \neq 0  \}$

ii)
\begin{align*}
  x(x^{2} - 4)^{-1} \geq 0 \\
  \frac{x}{x^2 - 4} \geq 0
\end{align*}

We need:
\begin{itemize}
  \item $x \geq 0, x^2 - 4 > 0,  \Rightarrow x > 2$,
  \item $x \leq 0, x^2 -4 < 0, \Rightarrow -2 < x \leq 0$
\end{itemize}

Therefore, $D = \{ x \in \mathbb{R}, x \in (-2, 0] \cup x \in (2, +\infty) \}$

  \subsubsection*{b)}

  Solution:

  i)

  While $x \to 2$, $x^2 - 3x + 2 \to 0$ and $x -2 \to 0 $, using L'Hopital rule, \begin{equation*}
    \lim_{x \to 2}\frac{x^2 - 3x + 2}{x - 2} = \lim_{x \to 2}\frac{2x-3}{1} = 1
  \end{equation*}

  ii)

  While $x \to -1$, $4 - \sqrt{x+17} \to 0$ and $2x+2 \to 0$, using L'Hopital rule, \begin{equation*}
    \lim_{x \to -2}\frac{4 - \sqrt{x+17}}{2x+2} =  \lim_{x \to -1}\frac{-\frac{1}{2}(x+17)^{-1/2}}{2} = \lim_{x \to -1}-\frac{1}{4 \sqrt{x+17}} = -1/16
  \end{equation*}

  \subsubsection*{c)}

  Solution:

  i)
  Trying to prove $ \forall \epsilon > 0, \exists \eta > 0$, assure \begin{equation*}
    |2x+1-11| < \epsilon,\quad \text{when}\, |x - 5| < \eta.
  \end{equation*}

  We have $|2x+1-11| = |2x-10|=2|x-5| < 2 \eta $,
  we can assign $\eta = \frac{1}{2} \epsilon$.

  Then $|2x+1-11| = |2x-10|=2|x-5| < 2 \eta = \epsilon$.

  Q.E.D.

  ii)

  Trying to prove $ \forall \epsilon > 0, \exists \eta > 0$, assure \begin{equation*}
    |x^{2} - 10x + 21| < \epsilon,\quad \text{when}\, |x - 3| < \eta.
  \end{equation*}

  We have $|x^{2} - 10x + 21| = |(x-3)(x-7)|=|(x-3)(x-3-4)| = |(x-3)^{2} - 4(x-3)| < |(x-3)^{2}| + 4 |x-3| < \eta ^{2} + 4 \eta $.

  We can assume $0 < \eta < 1$, then $\eta ^{2} + 4 \eta < \eta + 4\eta = 5\eta$, then we assign $\eta = \frac{1}{5}\epsilon$.
  Then $|x^{2} - 10x + 21| < 5\eta = \epsilon$

  Q.E.D.

  \subsection*{Problem 2}
  % TODO: Check the Proof
  Trying to prove $ \forall \epsilon > 0, \exists \eta > 0$, assure \begin{equation*}
    |f(x)g(x) - AB| < \epsilon,\quad \text{when}\, |x-a| < \eta.
  \end{equation*}

  \begin{align*}
    |f(x)g(x) - AB| & = |f(x)g(x) - B f(x) + B f(x) - AB|    \\
                    & =|f(x)(g(x)-B) + B(f(x)-A)|            \\
                    & \leq |f(x)(g(x)- B)| + |B(f(x)-A)|     \\
                    & = |f(x)||g(x)- B| + |B||f(x)-A|        \\
                    & = |f(x)-A + A||g(x)-B| + |B||f(x) - A| \\
    %TODO: continue 
  \end{align*}



  \subsection*{Problem 3}

  \subsubsection*{a)}

  Solution:

  i)
  \begin{align*}
    \frac{d f(x)}{dx} & = \lim_{h \to 0} \frac{f(x+h) - f(x)}{h}                            \\
                      & = \lim_{h \to 0} \frac{(x+h)^{2} - (x+h) + 2 - x ^{2} + x -2}{h}    \\
                      & = \lim_{h \to 0} \frac{x ^{2} + 2hx + h ^{2} - x -h - x ^{2} +x}{h} \\
                      & = \lim_{h \to 0} \frac{(2x-1)h}{h} = 2x-1
  \end{align*}

  ii)
  \begin{align*}
    \frac{d f(x)}{dx} & = \lim_{h \to 0} \frac{f(x+h) - f(x)}{h}                                                                        \\
                      & = \lim_{h \to 0} \frac{-2(x+h)^3 + (x+h) ^{2} + 2x ^{3}  - x ^{2}}{h}                                           \\
                      & = \lim_{h \to 0} \frac{-2 x ^{3} - 6hx ^{2} -6xh ^{2} - 2h ^{3} + x ^{2} + 2xh + h ^{2} + 2 x ^{3} - x ^{2}}{h} \\
                      & = \lim_{h \to 0} \frac{-6x ^{2} h + 2xh -6 xh ^{2} -2h ^{3}}{h} = -6 x ^{2} + 2x
  \end{align*}



  \subsubsection*{b)}

  Solution:

  i)
  \begin{align*}
    f'(x)  & = e ^{-3x ^{2}} (-6 x) = -6x e ^{-3 x^{2}} \\
    f''(x) & = -6e ^{-3 x^{2}} -6x (-6x e ^{-3 x^{2}})  \\
           & = 6(6x ^{2} - 1) e ^{-3 x ^{2}}
  \end{align*}

  ii)
  \begin{align*}
    f'(x)  & = \frac{2xe^{x} - x ^{2} e ^{x}}{e ^{2x}} = \frac{2x-x ^{2}}{e ^{x}}              \\
    f''(x) & = \frac{(2 - 2x)e ^{x}-(2x - x ^{2})e ^{x}}{e ^{2x}} = \frac{2-4x+x ^{2}}{e ^{x}}
  \end{align*}

  iii)
  \begin{align*}
    f'(x)  & = e ^{x}(\ln(x ^{2} + 2) + \frac{2x}{x ^{2} + 2})                                \\
    f''(x) & = f'(x) + e ^{x} (\frac{2x}{x ^{2} + 2} + \frac{-2x ^{2} + 4}{(x ^{2} + 2)^{2}}) \\
           & = f'(x) + e ^{x}\frac{2x ^{3} -2 x ^{2} + 4x + 4}{(x ^{2} + 2) ^{2}}
  \end{align*}

  iv)
  \begin{align*}
    f'(x)  & = -4x ^{3} e ^{-x} + x ^{4} e ^{-x} = (x ^{4} - 4x ^{3})e ^{-x} \\
    f''(x) & =(-x ^{4} +4x ^{3} +4x ^{3} =12 x ^{2})e ^{-x}                  \\
           & = (-x ^{4} + 8 x ^{3}- 12 x ^{2}) e ^{-x}
  \end{align*}

  v)
  \begin{align*}
    f'(x)  & = n(e ^{x} + x ^{2}) ^{n - 1} (e ^{x} + 2x)                                                    \\
    f''(x) & = n(n-1)(e ^{x} + x ^{2}) ^{n - 2} (e ^{x} + 2x)^{2} +n(e ^{x} + x ^{2}) ^{n - 1} (e ^{x} + 2)
  \end{align*}

  vi)
  \begin{align*}
    f'(x)  & = \frac{1}{2(x + x ^{\frac{1}{2}})}        \\
    f''(x) & = -\frac{2 + x ^{-1/2}}{4(x + \sqrt{x})^2}
  \end{align*}

  \subsection*{Problem 4}

  \subsubsection*{a)}

  Solution:

  \subsubsection*{b)}

  Solution:

  For $0.98 ^{20}$, $ x = -0.02, m = 20$, then $0.98 ^{20} \approx 1 + 20 \times (-0.02) = 0.6 $.

  For $\sqrt{37}$, we can calculate $a = \frac{\sqrt{37}}{6} = \sqrt{\frac{37}{36}}$ first.

  For $a$, $x = \frac{1}{36}, m = 0.5$, then $a \approx 1 + \frac{1}{72} = \frac{73}{72}$. Therefore, $\sqrt{37} \approx \frac{73}{12}$

  \subsubsection*{c)}

  Solution:

  i)

  \begin{align*}
    f' = F'2x + 2FF'
  \end{align*}

  ii)

  \begin{align*}
    g' = \frac{1}{2} (F + \sqrt{x})^{-1/2} (F' + \frac{1}{2}x ^{-1/2})
  \end{align*}

  iii)

  \begin{align*}
    h' = F'(x-2h) + 2F'(2x+t)
  \end{align*}


  \subsection*{Problem 5}

  \subsubsection*{a)}

  Solution:

  i)

  \begin{align*}
      & \lim_{x \to a} \frac{x ^{2} - a ^{2}}{ x - a} \qquad\text{($\frac{0}{0}$)} \\
    = & \lim_{x \to a} \frac{2x}{1} = 2a
  \end{align*}

  ii)

  \begin{align*}
      & \lim_{x \to 1} \frac{x ^{x} -x}{1-x + \ln x} \qquad\text{($\frac{0}{0}$)}                \\
    = & \lim_{x \to 1} \frac{x ^{x}(\ln x + 1) - 1}{-1+\frac{1}{x}}\qquad\text{($\frac{0}{0}$)}  \\
    = & \lim_{x \to 1} \frac{x ^{x}(\ln x + 1)^{2} + \frac{1}{x}x ^{x}}{- \frac{1}{x ^{2}}} = -2
  \end{align*}

  iii)

  \begin{align*}
     & \lim_{x \to 1} \frac{\ln x - x + 1}{(x - 1)^{2}} \qquad\text{($\frac{0}{0}$)}                                 \\
     & = \lim_{x \to 1} \frac{1/x - 1}{2(x - 1)} = \lim_{x \to 1} \frac{1 - x}{2x(x-1)} \qquad\text{($\frac{0}{0}$)} \\
     & = \lim_{x \to 1} \frac{-1}{4x - 2} = -1/2
  \end{align*}

  \subsubsection*{b)}

  Solution:

  i)

  If $\beta > 1$ and $ x \to \infty$,
  \begin{align*}
      & \lim_{x \to \infty} \frac{x ^{\alpha}}{\beta ^{x}} \qquad\text{($\frac{\infty}{\infty}$)}                               \\
    = & \lim_{x \to \infty}\frac{\alpha(\alpha-1)(\alpha-2)\dots(\alpha-n)}{(\ln \beta)^{n}} \frac{x ^{\alpha - n}}{\beta ^{x}}
  \end{align*}

  While $n > a$, $\frac{x ^{\alpha - n}}{\beta ^{x}} \to 0$, then $\lim_{x \to \infty} \frac{x ^{\alpha}}{\beta ^{x}} = 0$.


  If $0 < \beta < 1$ and $ x \to \infty$, $\lim_{x \to \infty} \frac{x ^{\alpha}}{\beta ^{x}} = \infty$.


  If $\beta = 1$ and $x \to \infty $, $\lim_{x \to \infty} \frac{x ^{\alpha}}{\beta ^{x}} = \infty$.


  ii)

  \begin{align*}
    \lim_{x \to 0 ^{+}} \frac{e ^{-1/x}}{x ^{\alpha}} = \lim_{x \to 0 ^{+}} \frac{1}{x ^{\alpha}e ^{1/x}} = 0
  \end{align*}


  \subsection*{Problem 6}

  \subsubsection*{a)}

  Solution:

  Let $A(x) = \big( \frac{\sin x}{x}\big)^{\frac{1}{x ^{2}}}$. Then $\ln A(x)  = \frac{1}{x ^{2}}\ln(\frac{\sin x}{x})$

  \begin{align*}
    \lim_{x \to 0 ^{+}} \ln A(x) & = \lim_{x \to 0 ^{+}}  \frac{\ln \frac{\sin x}{x}}{x ^{2}} \\
                                 & = \lim_{x \to 0 ^{+}} \frac{\sin x - x}{x ^{3}}            \\
                                 & =\lim_{x \to 0 ^{+}}  \frac{\cos x - 1}{3 x ^{2}}          \\
                                 & = \lim_{x \to 0 ^{+}} \frac{-\sin x}{6x}                   \\
                                 & = \lim_{x \to 0 ^{+}} -\frac{1}{6}
  \end{align*}

  Therefore, $\lim_{x \to 0 ^{+}} A(x) = e ^{-1/6}$

  \subsection*{Problem 7}

  \subsubsection*{a)}

  Solution:

  i)

  \begin{align*}
    f'(x)  & = 2(e ^{2x} + 4e ^{-x})(2e ^{2x} -4e ^{-x}) \\
           & = 4(e ^{2x} + 4e ^{-x})(e ^{2x} - 2e ^{-x}) \\
           & = 4 (e ^{4x} + 2e ^{x} - 8e ^{-2x})         \\
    f''(x) & = 4(4e ^{4x} + 2e ^{x} + 16e ^{-2x})        \\
           & = 8(2e ^{4x} + e ^{x} + 8e ^{-2x})
  \end{align*}

  ii)

  \begin{align*}
    f'(x)                         & = 0               \\
    e ^{4x} + 2e ^{x} - 8e ^{-2x} & = 0               \\
    e ^{6x} + 2e ^{3x} - 8        & =0                \\
    (e ^{3x} - 2)(e ^{3x} + 4)    & = 0               \\
    x                             & = \frac{1}{3}\ln2
  \end{align*}

  Since $\forall x \in \mathbb{R}, f'' > 0$, then $f'$ is increasing on domain $\mathbb{R}$ and $f(x)$ in convex;

  Thus, $f' < 0$, while $x < \frac{1}{3}\ln2$ and $f' > 0$ while $ x > \frac{1}{3}\ln2$.

  Therefore $f(x)$ in decreasing when $x \in (-\infty, \frac{1}{3}\ln2)$ and increasing when $x \in (\frac{1}{3}\ln2, +\infty)$.

  iii)

  From ii) we can know $f(x)$ has one only extreme point (minimum) while $x = \frac{1}{3}\ln2$.


  \subsubsection*{b)}

  Solution:

  Let $L(x)$ be the distance from a point on $y=x ^{2}$ to point $(2,1)$. Then
  \begin{align*}
    L(x) & = \sqrt{(y - 1) ^{2} + (x-2) ^{2}}                                 \\
         & = \sqrt{(x ^{2} - 1) ^{2} + (x-2) ^{2}}                            \\
         & = \sqrt{x ^{4} - x ^{2} -4x + 5}                                   \\
    L'   & = \frac{1}{2}\frac{4x ^{3} -2x -4}{\sqrt{x ^{4} - x ^{2} -4x + 5}}
  \end{align*}

  We need to calculate $L'=0$, then $2x ^{3}-x -2=0$. Let $f(x) = 2x ^{3}-x-2$.
  We have $f(1)=-1, f(2) = 12$.
  Using Taylor expansion at $x=1$:\begin{equation*}
    f(x) \approx f(1) + f'(1)(x-1) = 5x-6
  \end{equation*}
  Then $x = \frac{6}{5}$.

  \subsubsection*{c)}

  Solution:

  i)

  \begin{align*}
    5x + y                  & = 10 \\
    5dx + dy                & = 0  \\
    \frac{dy}{dx}           & = -5 \\
    \frac{d ^{2}y}{dx ^{2}} & = 0
  \end{align*}

  ii)

  \begin{align*}
    xy ^{3}                 & = 125                         \\
    y ^{3} dx + 3xy ^{2}dy  & = 0                           \\
    \frac{dy}{dx}           & = -\frac{y}{3x}               \\
    \frac{d ^{2}y}{dx ^{2}} & = - \frac{3xy' - 3y}{9x ^{2}} \\
                            & = \frac{4y}{9x ^{2}}
  \end{align*}


  iii)

  \begin{align*}
    e ^{2y}                 & = x ^{3}                               \\
    2e ^{2y} dy             & = 3x ^{2} dx                           \\
    \frac{dy}{dx}           & = \frac{3x ^{2}}{2e ^{2y}}             \\
    \frac{d ^{2}y}{dx ^{2}} & = \frac{3x - 3x ^{2}y'}{e ^{2y}}       \\
                            & = \frac{6xe ^{2y} - 9x ^{4}}{2e ^{4y}} \\
  \end{align*}

  \subsection*{Problem 8}

  Solution:

  a)

  \begin{align*}
    \int_{1}^{\infty } \frac{5}{x ^{5}} \, dx & = \Big[- \frac{5}{4}x ^{-4}\Big]_{1}^{\infty} \\
                                              & = 0-(-\frac{5}{4}) = \frac{5}{4}
  \end{align*}

  b)
  \begin{align*}
    \int_{0}^{1} x ^{3}(1 +x ^{4})^{4} dx & = \frac{1}{4}\int_{0}^{1}(1+x ^{4})^{4} d(x ^{4}+1)          \\
                                          & = \frac{1}{4} \frac{1}{5} \Big[(1+x ^{4} )^{5} \Big]_{0}^{1} \\
                                          & = \frac{1}{20}[2 ^{5} - 1]                                   \\
                                          & = \frac{31}{20}
  \end{align*}

  c)
  \begin{align*}
    \int_{}^{}\frac{2}{x + 5} dx = 2\ln (x+5) + C
  \end{align*}

  d)

  \begin{align*}
    \int_{}^{}xe ^{-2x}dx & = -\frac{1}{2}\int_{}^{}x de ^{-2x}                      \\
                          & = -\frac{1}{2}\Big[xe ^{-2x} - \int_{}^{}e ^{-2x}dx\Big] \\
                          & =-\frac{1}{2}\Big[xe ^{-2x} + \frac{1}{2}e ^{-2x}\Big]   \\
                          & = \frac{1-x}{2e ^{2x}}
  \end{align*}

  e)
  \begin{align*}
    \int_{}^{}x \ln(1+x ^{2}) dx & =\frac{1}{2} \int_{}^{}\ln(1+x ^{2}) d(x ^{2}+1)                                         \\
                                 & = \frac{1}{2}\Big[(1+x ^{2})\ln(1+x ^{2}) - \int_{}^{}(x ^{2} + 1)\dv \ln(1+x ^{2})\Big] \\
                                 & =\frac{1}{2}\Big[(1+x ^{2})\ln(1+x ^{2}) - \int_{}^{} 2x
    dx)\Big]                                                                                                                \\
                                 & =\frac{1}{2}\Big[(1+x ^{2})\ln(1+x ^{2}) - x ^{2})\Big] + C                              \\
  \end{align*}


  f)

  \begin{align*}
    \int_{}^{}(\ln x)^{2}dx & = \int_{}^{}t ^{2}de ^{t} \qquad (\text{Using $t$ replace $\ln x$}) \\
                            & =t ^{2}e ^{t} - \int_{}^{}2te ^{t}dt                                \\
                            & =t ^{2}e ^{t} - (2te ^{t} - 2e ^{t})                                \\
                            & =(t ^{2}-2t+2)e ^{t}                                                \\
                            & = [(\ln x)^{2}-2 \ln x + 2]x + C \qquad \text{(Replace back)}
  \end{align*}

  g)

  \begin{align*}
    \int_{}^{}\frac{\sin ^{3}x}{1+ \cos x}dx & = \int_{}^{}\frac{\sin x (1- \cos ^{2}x)}{1+\cos x}dx \\
                                             & = -\int_{}^{}\frac{1-\cos ^{2} x}{1+ \cos x} d \cos x \\
                                             & = \int_{}^{}\cos x - 1 d \cos x                       \\
                                             & = \frac{1}{2}\cos ^{2} x - \cos x + C
  \end{align*}


  h)
  \begin{align*}
    \int_{}^{}\frac{1}{x \ln x}dx & = \int_{}^{}\frac{1}{\ln x}d \ln x \\
                                  & =\ln(\ln x) +C
  \end{align*}

  i)
  \begin{align*}
    \int_{}^{}(1+\frac{1}{x})^{3}\frac{1}{x ^{2}} dx & = -\int_{}^{}(1 + \frac{1}{x})^{3}d(\frac{1}{x} + 1) \\
                                                     & = -\frac{1}{4} (1+\frac{1}{x})^{4} + C
  \end{align*}

  \subsection*{Problem 9}

  \subsubsection*{a)}

  Solution:

  i)

  \begin{align*}
    F(x)  & = \int_{x}^{0} t \cos t dt         \\
    F'(x) & = 0 - x \cos x + \int_{0}^{x} 0 dt \\
          & = -x\cos x
  \end{align*}

  ii)

  \begin{align*}
    F(x)  & = x\int_{x}^{0} t \cos t dt                                             \\
    F'(x) & = 0 - x ^{2}\cos x + \int_{x}^{0}t \cos t dt                            \\
          & = -x ^{2} \cos x + \int_{x}^{0}t d \sin t                               \\
          & = -x ^{2} \cos x + \Big[ t \sin t \Big]_{x}^{0} - \int_{x}^{0}\sin t dt \\
          & = -x ^{2} \cos x - x \sin x + (1 - \cos x)                              \\
          & = 1 - (1+x ^{2})\cos x - x \sin x
  \end{align*}

  iii)

  \begin{align*}
    F(x)  & = \int_{1}^{x}(t ^{2} - 2t + 5) dt \\
    F'(x) & = x ^{2} - 2x + 5 - 0+ 0           \\
          & = x ^{2} - 2x + 5
  \end{align*}

  iv)

  \begin{align*}
    F(x)  & = \int_{4}^{x}(\sqrt{u} + \frac{x}{\sqrt{u}}) du                 \\
    F'(x) & =\int_{4}^{x}(\sqrt{u}) du +\int_{4}^{x}( \frac{x}{\sqrt{u}}) du \\
          & = \sqrt{x} + \sqrt{x} - 0 + \int_{4}^{x}\frac{1}{\sqrt{u}} du    \\
          & = 2 \sqrt{x} + \Big[ 2 u ^{1/2} \Big]_{4}^{x}                    \\
          & = 4 \sqrt{x} - 4
  \end{align*}

  \subsubsection*{b)}

  Solution:

  Let $f(x) = (e ^{x} + 1) e ^{-x}$, then
  \begin{align*}
    \int_{}^{} f(x) dx & = \int_{}^{} 1 + e ^{-x} dx \\
                       & = x - e ^{-x} + C
  \end{align*}

  When $x = 0$, we have \begin{align*}
    0 - 1 + C & = 1 \\
    C         & = 2
  \end{align*}

  Therefore, the antiderivative is $F(x) = x - e ^{ -x } + 2$

  \subsection*{Problem 10}

  \subsubsection*{a)}

  Solution:

  \begin{align*}
    \lim_{x \to 0}\frac{\int_{0}^{x}\sin ^{2} t dt}{x ^{2}} & =\lim_{x \to 0} \frac{\sin ^{2} x}{2x} = \lim_{x \to 0} \frac{x}{2} = 0
  \end{align*}

  \subsubsection*{b)}

  \begin{align*}
    f'  & = \frac{1}{2 \sqrt{x}} - 1 \\
    f'' & < 0
  \end{align*}

  Let $f' = 0$, $x = \frac{1}{4}$. Therefore, $f(x)$ has a maximum when $x = \frac{1}{4}$.


  \subsection*{Problem 11}

  Solution:

  a)

  \begin{align*}
    f'  & = 4 \frac{2 \ln t - t (\ln t) ^{2}}{t ^{3}} = \frac{8 \ln t}{t ^{2}} - \frac{4 \ln ^{2} t}{t ^{2}} \\
    f'' & = \frac{8 - 24 \ln t + 8 \ln ^{2} t}{t ^{3}}
  \end{align*}

  b)

  Let $f' = 0$, $\ln t = 0$ or $\ln t = 2$, then $t = 1, e ^{2}$.

  When $ t = 1$, $f'' > 0$, $f$ has a local minimum.

  When $ t = e ^{2}$, $f '' < 0$, $f$ has a local maximum.

  And $f' < 0$ when $t \in (0, 1) \cup (e ^{2}, +\infty)$, $f' > 0$ when $t \in (1, e ^{2})$.

$f(1) = 0$ and $f(e ^{2}) = \frac{16}{e ^{2}}$

  % TODO: Graph

  c)

  Let $S = \int_{1}^{e ^{2}} f(t)dt$, then
  \begin{align*}
    S & = 4\int_{1}^{e ^{2}} \frac{ln ^{2} t}{t} dt         \\
      & = 4 \int_{1}^{e ^{2}} \ln ^{2} t d(\ln t)           \\
      & = 4 \Big[ \frac{1}{3} \ln ^{3} t \Big]_{1}^{e ^{2}} \\
      & = \frac{4}{3}(8 - 0) = \frac{32}{3}
  \end{align*}


  \subsection*{Problem 12}

  Solution:

  a)

  It's easy to prove $F' = f$.

  \begin{align*}
    \lim_{x \to -\infty} F(x) & = \lim_{x \to -\infty} \frac{\alpha}{e ^{- \lambda x} + \alpha} = 0 \\
    \lim_{x \to +\infty} F(x) & =\lim_{x \to +\infty} \frac{\alpha}{e ^{- \lambda x} + \alpha} = 1
  \end{align*}

  b)

  \begin{align*}
    \int_{-\infty }^{x} f(t) dt = \Big[ F(t) \Big]_{- \infty }^{x} = F(x) -0 = F(x) \\
  \end{align*}

  Since $  F'(x) = f(x) >0$, $F(x)$ is strictly increasing.

  c)

$F''(x) = f'(x)$, Then

  \begin{align*}
    F''(x) = f'(x) & = \frac{df(u)}{du} \frac{du}{dx} \qquad \text{(using $u(x) = e ^{-\lambda x}$)}                          \\
                   & = \alpha \lambda u' \frac{\alpha - u}{(\alpha + u)^{3}}                                                  \\
                   & = \alpha \lambda ^{2} e ^{- \lambda x} \frac{e ^{- \lambda x} - \alpha}{(\alpha + e ^{- \lambda x})^{3}}
  \end{align*}

  Let $F'' = 0$, then $x_{0} = -\frac{\ln \alpha}{\lambda}, F(x_{0}) = 1/2$.

% TODO: Graph

% --- Document ends here ---

\end{document}

